%!TEX TS-program = xelatex
\documentclass[12pt, a4paper, oneside]{article}

%%%%%%%%%% Внешний вид страницы %%%%%%%%%%

\usepackage[paper=a4paper, top=20mm, bottom=15mm,left=20mm,right=15mm]{geometry}
\usepackage{indentfirst}    % установка отступа в первом абзаце главы

\usepackage{setspace}
\setstretch{1.15}  % межстрочный интервал
\setlength{\parskip}{4mm}   % Расстояние между абзацами
% Разные длины в LaTeX: https://en.wikibooks.org/wiki/LaTeX/Lengths

% свешиваем пунктуацию
% теперь знаки пунктуации могут вылезать за правую границу текста, при этом текст выглядит ровнее
\usepackage{microtype}

% \flushbottom                            % Эта команда заставляет LaTeX чуть растягивать строки, чтобы получить идеально прямоугольную страницу
\righthyphenmin=2                       % Разрешение переноса двух и более символов
\widowpenalty=300                     % Небольшое наказание за вдовствующую строку (одна строка абзаца на этой странице, остальное --- на следующей)
\clubpenalty=3000                     % Приличное наказание за сиротствующую строку (омерзительно висящая одинокая строка в начале страницы)
\tolerance=10000     % Ещё какое-то наказание.
\usepackage[english, russian]{babel} % выбор языка для документа

% меняю оформление секций 
\usepackage{titlesec}
\usepackage{sectsty}

% выбрасываю нумерацию страниц и колонтитулы 
\pagestyle{empty}

% синие круглые бульпоинты в списках itemize 
\usepackage{enumitem}


\usepackage[colorlinks=true,linkcolor=blue,urlcolor=black,bookmarksopen=true]{hyperref}

\usepackage{listings}
% для кода 
\documentclass[working]{article}

\begin{document}

\section{Quantitative Methods and Econometrics  }
\subsection{ Measures of central tendency: population mean, sample mean, arithmetic mean, geometric mean, harmonic mean }
The population mean (or the expected value) is a measure of central tendency that represents the average of all observations in a population. It is denoted as $\mu$ and is calculated as follows:

$\mu = \frac{\sum_{i=1}^{N}x_i}{N}$

where $x_i$ is the $i^{th}$ observation and $N$ is the total number of observations in the population.

The sample mean is a measure of central tendency that represents the average of a sample drawn from a population. It is denoted as $\bar{x}$ and is calculated as follows:

$\bar{x} = \frac{\sum_{i=1}^{n}x_i}{n}$

where $x_i$ is the $i^{th}$ observation in the sample and $n$ is the total number of observations in the sample.

The arithmetic mean is the same as the sample mean or the population mean. It is calculated by summing up all the observations and dividing by the number of observations.

The geometric mean is a measure of central tendency that represents the average of the logarithms of all observations. It is denoted as $G$ and is calculated as follows:

$G = (\prod_{i=1}^{n}x_i)^{\frac{1}{n}}$

The harmonic mean is a measure of central tendency that represents the average of the reciprocals of all observations. It is denoted as $H$ and is calculated as follows:

$H = \frac{n}{\sum_{i=1}^{n}\frac{1}{x_i}}$

Note: The harmonic mean is only applicable for positive observations and is generally smaller than the other measures of central tendency.
\subsection{ Measures of location and dispersion: quantile, mean absolute deviation, sample variance and standard deviation }
Measures of location and dispersion are used to describe the distribution of a set of data. They provide information about where the data is centered and how spread out the data is. There are several measures of location and dispersion, including quantile, mean absolute deviation, sample variance, and standard deviation.

A quantile is a value that separates the data into equal parts. The $p^{th}$ quantile, denoted as $Q_p$, is the value such that $p$ percent of the observations are less than or equal to $Q_p$.

The mean absolute deviation (MAD) is a measure of dispersion that represents the average absolute deviation of the observations from the mean. It is denoted as $MAD$ and is calculated as follows:

$MAD = \frac{\sum_{i=1}^{n}|x_i - \bar{x}|}{n}$

where $x_i$ is the $i^{th}$ observation in the sample and $\bar{x}$ is the sample mean.

The sample variance, denoted as $s^2$, is a measure of dispersion that represents the average squared deviation of the observations from the sample mean. It is calculated as follows:

$s^2 = \frac{\sum_{i=1}^{n}(x_i - \bar{x})^2}{n-1}$

The standard deviation, denoted as $s$, is the square root of the sample variance. It is calculated as follows:

$s = \sqrt{s^2}$

Note: The sample variance and standard deviation are only applicable for a sample drawn from a population and not for the entire population.
\subsection{ Skewness, kurtosis, and correlation  }
Skewness, kurtosis, and correlation are measures that describe the shape, peakness, and relationship between two variables, respectively.

Skewness is a measure of the asymmetry of a distribution. It is calculated as the third standardized moment of a distribution and can be either positive, negative, or zero. Positive skewness indicates that the tail of the distribution extends to the right, negative skewness indicates that the tail extends to the left, and zero skewness indicates a symmetric distribution. The formula for skewness is given by:

$Skewness = \frac{\frac{1}{n} \sum_{i=1}^{n} (\frac{x_i - \bar{x}}{s})^3}{(\frac{1}{n} \sum_{i=1}^{n} (\frac{x_i - \bar{x}}{s})^2)^{3/2}}$

where $x_i$ is the $i^{th}$ observation, $\bar{x}$ is the sample mean, $s$ is the sample standard deviation, and $n$ is the sample size.

Kurtosis is a measure of the peakedness of a distribution. It is calculated as the fourth standardized moment of a distribution and can be either positive, negative, or zero. Positive kurtosis indicates a more peaked distribution than a normal distribution, negative kurtosis indicates a flatter distribution than a normal distribution, and zero kurtosis indicates a normal distribution. The formula for kurtosis is given by:

$Kurtosis = \frac{\frac{1}{n} \sum_{i=1}^{n} (\frac{x_i - \bar{x}}{s})^4}{(\frac{1}{n} \sum_{i=1}^{n} (\frac{x_i - \bar{x}}{s})^2)^2} - 3$

where $x_i$ is the $i^{th}$ observation, $\bar{x}$ is the sample mean, $s$ is the sample standard deviation, and $n$ is the sample size.

Correlation measures the relationship between two variables. It is calculated as the covariance between two variables divided by the product of their standard deviations. The formula for Pearson's correlation coefficient, denoted as $\rho$, is given by:

$\rho = \frac{\frac{1}{n} \sum_{i=1}^{n} (x_i - \bar{x})(y_i - \bar{y})}{s_x s_y}$

where $x_i$ and $y_i$ are the $i^{th}$ observations of two variables, $\bar{x}$ and $\bar{y}$ are the sample means of the two variables, $s_x$ and $s_y$ are the sample standard deviations of the two variables, and $n$ is the sample size.
\subsection{ Expected value, variance and covariance }

Expected value, variance, and covariance are important concepts in probability and statistics. They are used to describe the distribution of a random variable and the relationship between two random variables.

The expected value, also known as the mean or the first moment, of a random variable is a measure of its central tendency. It is the weighted average of all possible values that the random variable can take, where the weights are the probabilities of each value. The formula for the expected value of a discrete random variable X, denoted as $E(X)$, is given by:

$E(X) = \sum_{i} x_i P(X = x_i)$

where $x_i$ is the $i^{th}$ possible value of the random variable X and $P(X = x_i)$ is the probability of the random variable taking the value $x_i$.

Variance is a measure of the spread of a random variable. It is the average of the squared deviations of the random variable from its expected value. The formula for the variance of a discrete random variable X, denoted as $Var(X)$, is given by:

$Var(X) = \sum_{i} (x_i - E(X))^2 P(X = x_i)$

Covariance measures the relationship between two random variables. It is the average of the product of the deviations of the two random variables from their expected values. The formula for the covariance between two random variables X and Y, denoted as $Cov(X, Y)$, is given by:

$Cov(X, Y) = \sum_{i} (x_i - E(X))(y_i - E(Y))P(X = x_i, Y = y_i)$

Note: If the two random variables are uncorrelated, their covariance is zero. If the two random variables are positively correlated, their covariance is positive, and if they are negatively correlated, their covariance is negative.
\subsection{ Confidence interval }
Confidence interval is a range of values that is used to estimate an unknown population parameter with a certain degree of confidence. In other words, it is a measure of the precision of an estimate. Given a sample of data, a confidence interval provides an interval of values that is likely to contain the true value of the population parameter with a specified probability, called the confidence level.

The formula for a confidence interval for the population mean $\mu$ of a normally distributed population, given a sample mean $\bar{x}$ and sample standard deviation $s$, with a confidence level of $1 - \alpha$ is given by:

$\bar{x} \pm z_{\frac{\alpha}{2}} \frac{s}{\sqrt{n}}$

where $z_{\frac{\alpha}{2}}$ is the critical value from the standard normal distribution with a cumulative probability of $\frac{\alpha}{2}$ to the left of it and $n$ is the sample size.

In general, a larger sample size and a lower confidence level result in a narrower confidence interval, which means that the estimate is more precise. Confidence intervals provide a useful way to communicate the uncertainty around an estimate and are widely used in statistical analysis and hypothesis testing.
\subsection{ Normal distribution. Standard normal distribution. Lognormal distribution }
The normal distribution, also known as the Gaussian distribution, is a continuous probability distribution that is commonly used to model real-valued random variables. It is characterized by its mean $\mu$ and variance $\sigma^2$. A normal distribution with mean $\mu$ and variance $\sigma^2$ is denoted as $N(\mu, \sigma^2)$.

The standard normal distribution is a normal distribution with a mean of 0 and a variance of 1. It is denoted as $N(0, 1)$. The standard normal distribution is important because many real-world variables can be transformed into standard normal variables through a technique called standardization, which makes it possible to use tables and software to find probabilities and make inferences.

The lognormal distribution is a continuous probability distribution that models the distribution of the logarithm of a random variable. It is often used to model variables that are positively skewed, such as financial returns, income, and stock prices. The lognormal distribution is characterized by its mean $\mu$ and standard deviation $\sigma$. A lognormal distribution with mean $\mu$ and standard deviation $\sigma$ is denoted as $LN(\mu, \sigma)$.

In summary, the normal distribution is widely used to model real-valued random variables, while the standard normal distribution is a standardized version of the normal distribution and is used in many statistical procedures. The lognormal distribution is used to model positively skewed variables and is often used in finance and economics.
\subsection{ Student’s t-distribution }
The Student's t-distribution is a continuous probability distribution that is used to make inferences about the mean of a population when the sample size is small or when the population variance is unknown. It is named after William Sealy Gosset, who wrote under the pseudonym "Student."

The t-distribution has heavier tails than the normal distribution, which means that it allows for more extreme values than the normal distribution. The shape of the t-distribution is determined by its degrees of freedom (df), which is the number of independent observations minus the number of parameters estimated from the sample. The larger the degrees of freedom, the closer the t-distribution approximates the standard normal distribution.

The formula for the probability density function (pdf) of the t-distribution with df degrees of freedom is:

$f(x) = \frac{\Gamma(\frac{df + 1}{2})}{\sqrt{df \pi} \Gamma(\frac{df}{2})} (1 + \frac{x^2}{df})^{-\frac{df + 1}{2}}$

where $\Gamma$ is the gamma function.

The t-distribution is widely used in hypothesis testing and confidence interval estimation, especially when the sample size is small or the population variance is unknown. It provides a useful way to account for the uncertainty associated with small sample sizes and allows for more accurate inferences than would be possible with the normal distribution alone.
\subsection{ Chi-square distribution }
The chi-square distribution is a continuous probability distribution that is used to model the sum of the squares of independent standard normal random variables. It is commonly used in hypothesis testing to assess the goodness of fit of a model, to test independence in contingency tables, and to test for homogeneity in variance.

The chi-square distribution has one parameter, degrees of freedom (df), which is equal to the number of independent standard normal random variables that are squared and summed. The formula for the probability density function (pdf) of the chi-square distribution with df degrees of freedom is:

$f(x) = \frac{1}{2^{\frac{df}{2}} \Gamma(\frac{df}{2})} x^{\frac{df}{2} - 1} e^{-\frac{x}{2}}$

where $\Gamma$ is the gamma function.

The chi-square distribution is symmetric and positively skewed, and its mean and variance are equal to its degrees of freedom. It is widely used in statistical analysis and is an important distribution in many statistical procedures, including hypothesis testing and model fitting.
\subsection{ Monte Carlo simulation }
Monte Carlo simulation is a statistical method that involves generating random values from a specified probability distribution to simulate real-world problems and estimate outcomes. The term "Monte Carlo" refers to the use of random sampling, just as the famous casino in Monaco uses random sampling to determine the outcome of games.

In Monte Carlo simulation, we generate a large number of random values, perform a series of calculations using these random values, and analyze the resulting outcomes to obtain an estimate of the desired quantity. For example, Monte Carlo simulation can be used to estimate the value of a financial derivative, to evaluate the risk associated with an investment, or to calculate the expected cost of a manufacturing process.

One of the main advantages of Monte Carlo simulation is its ability to account for uncertainty and variability in real-world problems. This makes it a useful tool for solving complex problems where traditional analytical methods may be inadequate. Monte Carlo simulation is widely used in various fields, including finance, engineering, physics, and biology.
\subsection{ Probability sampling methods: systematic sampling, stratified random sampling, cluster sampling }
Probability sampling methods are techniques used to select a sample from a population in such a way that each member of the population has a known, non-zero probability of being included in the sample. These methods allow for the estimation of population parameters from a smaller, more manageable sample, and provide a means for making generalizations about the population based on the sample results.

Systematic sampling: Systematic sampling involves selecting every kth member of the population to be included in the sample, where k is the sampling interval. The first member of the sample is selected randomly from the first k members of the population. The systematic sampling formula is as follows:
$$ X_i = X_{(i-1)k + R} $$

where X is the population, i is the sample number, k is the sampling interval, and R is a random number between 1 and k.

Stratified random sampling: Stratified random sampling involves dividing the population into subgroups, or strata, based on some criterion, and then randomly selecting a sample from each stratum. The objective is to ensure that the sample reflects the composition of the population with respect to the criterion used to form the strata.

Cluster sampling: Cluster sampling involves dividing the population into groups, or clusters, and selecting a sample of clusters to be included in the sample. All members of the selected clusters are then included in the sample. The objective is to reduce the cost of sampling by reducing the number of units that need to be measured.

These methods are widely used in various applications of sampling, including market research, health surveys, and population studies. The choice of method depends on the characteristics of the population and the goals of the study.

\subsection{ Non-probability sampling methods: convenience sampling, judgmental sampling }
Non-probability sampling methods are sampling techniques where the sample units are selected without considering the probability of each unit being selected. Some of the commonly used non-probability sampling methods are:

Convenience Sampling: It is a non-probability sampling method where the sample units are selected based on the ease of accessibility. This method is usually used in exploratory research when the researcher is looking for initial insights into the population. This method is quick and easy, but the sample may not be representative of the population and the results may not be generalizable.

Judgmental Sampling: It is a non-probability sampling method where the sample units are selected based on the judgment of the researcher. This method is used when the researcher has a specific knowledge of the population and wants to select a sample that best represents the population. The sample units are selected based on specific criteria such as age, education level, gender, etc. This method is subjective and may not provide representative results.

Overall, non-probability sampling methods are less reliable compared to probability sampling methods, but they can still provide useful insights into a population.






\subsection{ Resampling: bootstrap, jackknife }
Resampling is a statistical technique that involves taking repeated random samples of data from a population in order to obtain additional information about the population. Resampling is often used when the data is limited or when the researcher wants to obtain a more robust estimate of population parameters.

Bootstrap:
Bootstrapping is a resampling method that involves taking multiple samples from the original dataset with replacement. Each sample is of the same size as the original dataset, and it allows the estimation of population parameters based on repeated sampling. The bootstrap is particularly useful in cases where the population distribution is unknown and the sample size is small.

Let's say we have a dataset of size N, and we want to obtain the mean of the population. Instead of estimating the mean based on the entire sample, we can obtain multiple bootstrap samples of the same size N and compute the mean for each of them. We repeat this process many times, typically hundreds or thousands of times, to obtain a large number of estimates of the population mean. Finally, we compute the average of all these estimates to obtain a robust estimate of the population mean.

Mathematically, the bootstrap estimate of a population parameter $\theta$ is given by:

$\hat{\theta}^{} = \frac{1}{B} \sum_{i=1}^{B} \theta^{}_{i}$

where B is the number of bootstrap samples, and $\theta^{*}_{i}$ is the estimate of $\theta$ based on the i-th bootstrap sample.

Jackknife:
The jackknife is a resampling method similar to the bootstrap, but instead of sampling with replacement, the jackknife samples without replacement. In the jackknife, each sample is created by leaving out one observation from the original sample. The jackknife can be used to obtain robust estimates of population parameters and to test hypotheses.

Mathematically, the jackknife estimate of a population parameter $\theta$ is given by:

$\hat{\theta}^{j} = \frac{n-1}{n} \sum_{i=1}^{n} \theta_{(-i)}$

where n is the sample size, $\theta_{(-i)}$ is the estimate of $\theta$ based on the sample obtained by leaving out the i-th observation. The average of all the jackknife estimates, $\hat{\theta}^{j}$, is a robust estimate of $\theta$.








\subsection{ Biases: data snooping, sample selection, survivorship, look-ahead, time-period }
Biases are systematic errors that can occur in statistical analysis, leading to incorrect or misleading results. There are several types of biases that can occur in the data collection and analysis process.

Data Snooping Bias: This occurs when a researcher looks at the data multiple times and continues to make adjustments to the analysis until a significant result is obtained. This can result in overfitting the data and producing results that are not representative of the population.

Sample Selection Bias: This occurs when the sample is not representative of the population, leading to incorrect conclusions. For example, if a sample is chosen only from individuals who live in a specific area, it may not accurately represent the population as a whole.

Survivorship Bias: This occurs when the analysis only includes individuals who have "survived" to a specific point in time, ignoring those who did not. This can lead to incorrect conclusions about the population as a whole.

Look-Ahead Bias: This occurs when information from future observations is used in the analysis, leading to incorrect conclusions. For example, if an analyst uses data from future periods to predict future outcomes, the results may be biased.

Time-Period Bias: This occurs when the analysis is limited to a specific time period, ignoring other important factors that may have an impact on the results. This can lead to incorrect conclusions about the population as a whole.






\subsection{ Central limit theorem }
The Central Limit Theorem (CLT) is a fundamental concept in probability and statistics that states that for a large enough sample size, the distribution of the sum (or average) of independent and identically distributed (i.i.d.) random variables will approach a normal distribution, regardless of the shape of the original distribution. Mathematically, the CLT can be stated as follows:

Let $X_1, X_2, ..., X_n$ be i.i.d. random variables with mean μ and standard deviation σ. Then, the distribution of the sample mean,

\frac{X_1 + X_2 + ... + X_n}{n}

will converge to a normal distribution with mean μ and standard deviation

\frac{\sigma}{\sqrt{n}}

as n approaches infinity. In other words, the larger the sample size, the more closely the distribution of the sample mean will approximate a normal distribution.

\subsection{ Hypothesis tests: null vs. alternative hypothesis, one-tail vs. two-tail hypothesis test }
A hypothesis test is a statistical procedure used to make a decision about an unknown population parameter based on a sample of data. In a hypothesis test, there are two types of hypotheses: the null hypothesis, denoted as $H_0$, and the alternative hypothesis, denoted as $H_a$.

$H_0$ represents the assumption that there is no difference or relationship between variables, or that a population parameter has a specific value. For example, $H_0$ might be that the population mean is equal to a certain value, or that two groups have the same mean.

$H_a$ represents the opposite of the null hypothesis and states that there is a difference or relationship between variables, or that the population parameter does not have the value specified in $H_0$. For example, $H_a$ might be that the population mean is not equal to a certain value, or that two groups have different means.

The goal of a hypothesis test is to either reject or fail to reject the null hypothesis based on the sample data. There are two types of hypothesis tests: one-tail and two-tail hypothesis tests.

In a one-tail hypothesis test, the alternative hypothesis is specified in one direction, either greater than or less than a specified value. For example, $H_0$ might be that the population mean is equal to a certain value, and $H_a$ might be that the population mean is greater than that value.

In a two-tail hypothesis test, the alternative hypothesis is specified in both directions, either greater than or less than a specified value. For example, $H_0$ might be that the population mean is equal to a certain value, and $H_a$ might be that the population mean is not equal to that value.
\subsection{ Type I and Type II errors }

Вы беременны - но я же мужчина. Извините - ошибка второго рода
\subsection{ Statistical significance and interpretation. P-value }
The concept of statistical significance is used to assess whether the results of a hypothesis test are unlikely to have occurred by chance. The statistical significance of a result is typically measured by the P-value, which is the probability of observing the sample data (or more extreme data) assuming the null hypothesis is true.

In hypothesis testing, a small P-value (typically less than 0.05) is interpreted as evidence against the null hypothesis, and supports the alternative hypothesis. In other words, a small P-value indicates that the sample data are not consistent with the null hypothesis, and suggests that the true population parameter is not equal to the value specified in $H_0$.

A large P-value (typically greater than 0.05), on the other hand, indicates that the sample data are consistent with the null hypothesis, and fails to provide evidence against it. In this case, the null hypothesis cannot be rejected and the results are considered statistically insignificant.

Mathematically, the P-value is calculated as the probability of observing a test statistic as extreme or more extreme than the one calculated from the sample data, assuming the null hypothesis is true. It can be calculated using various methods, including analytical methods or simulation techniques such as Monte Carlo simulation.

The P-value can be used as a guide for making decisions in hypothesis testing, but it should not be interpreted as the probability that the null hypothesis is true, or that the alternative hypothesis is false.
\subsection{ t-test, z-test and chi-square test }

The t-test is a hypothesis test that is used to compare the means of two samples. It is often used when the sample size is small, or the population standard deviation is unknown. The t-test is based on the t-distribution, which is a distribution of the difference between two means standardized by the estimated standard error of the difference.

The z-test is a hypothesis test that is used to compare the means of two samples when the population standard deviation is known. The z-test is based on the standard normal distribution, and the test statistic is calculated as the difference between the sample mean and the population mean, standardized by the population standard deviation.

The chi-square test is a hypothesis test that is used to determine if there is a relationship between two categorical variables. The chi-square test calculates the difference between the observed and expected frequencies in a contingency table and measures if this difference is significant. The test statistic follows a chi-square distribution, which is used to determine the level of significance of the test.

Let's denote the sample mean as $\overline{x}$, population mean as $\mu$, sample standard deviation as $s$, population standard deviation as $\sigma$, and sample size as $n$.

For the t-test, the test statistic is given by:

$$t = \frac{\overline{x} - \mu}{\frac{s}{\sqrt{n}}}$$

For the z-test, the test statistic is given by:

$$z = \frac{\overline{x} - \mu}{\frac{\sigma}{\sqrt{n}}}$$

For the chi-square test, the test statistic is given by:

$$\chi^2 = \sum_{i=1}^{k} \frac{(O_i - E_i)^2}{E_i}$$

where $O_i$ is the observed frequency in cell $i$, $E_i$ is the expected frequency in cell $i$, and $k$ is the number of cells in the contingency table.

\subsection{ Simple linear regression model: sum of squared errors (SSE), slope coefficient interpretation }
In the simple linear regression model, the goal is to model the relationship between a dependent variable and an independent variable. The model is represented by a linear equation:

$y = b_0 + b_1x$

where $y$ is the dependent variable, $x$ is the independent variable, $b_0$ is the intercept, and $b_1$ is the slope coefficient. The slope coefficient represents the change in the dependent variable for a unit change in the independent variable.

To estimate the parameters of the model, the method of least squares is used, which minimizes the sum of squared errors (SSE) between the observed and predicted values. The SSE is defined as:

$SSE = \sum_{i=1}^n (y_i - \hat{y}_i)^2$

where $y_i$ is the observed value of the dependent variable, $\hat{y}_i$ is the predicted value of the dependent variable, and $n$ is the number of observations.

Once the parameters are estimated, the model can be used to make predictions about the dependent variable based on a given independent variable. The interpretation of the slope coefficient is that it represents the change in the dependent variable for a unit change in the independent variable, holding all other variables constant.
\subsection{ Homoskedasticity vs. heteroskedasticity  }
Homoskedasticity and heteroskedasticity are terms used to describe the constant variance of residuals in a regression model.

Homoskedasticity refers to the assumption that the variance of residuals (i.e. the differences between the observed and predicted values) is constant and equal across all values of the independent variable. In other words, the scatter of residuals is the same across all points on the plot of residuals versus the independent variable. This assumption is important for many statistical tests, such as t-tests, that are based on the normal distribution.

Heteroskedasticity, on the other hand, refers to the situation where the variance of residuals is not constant and changes as a function of the independent variable. In other words, the scatter of residuals is different across different points on the plot of residuals versus the independent variable. This assumption can cause problems with many statistical tests, leading to incorrect results or biased coefficients.

To check for homoskedasticity, one can plot residuals against the independent variable and look for a pattern. A constant scatter of residuals indicates homoskedasticity, while a pattern of increasing or decreasing variance indicates heteroskedasticity.



\subsection{ Analysis of variance (ANOVA): total sum of squares (SST), regression sum of squares (RSS), sum of squared errors (SSE) }
The Analysis of Variance (ANOVA) is a statistical technique used to determine the variability of a response variable based on one or more predictor variables. ANOVA provides an effective way of testing the difference in means of multiple groups, and it is a generalization of the two-sample t-test for independent samples.

In ANOVA, we partition the total sum of squares (SST) into two components: the regression sum of squares (RSS) and the sum of squared errors (SSE).

The regression sum of squares (RSS) is a measure of the variation explained by the regression model, which represents the sum of squared differences between the predicted values of the response variable and its mean. The formula for the RSS is:

$$RSS = \sum_{i=1}^{n} (\hat{y}_i - \bar{y})^2$$

where $\hat{y}_i$ is the predicted value of the response variable for the i-th observation and $\bar{y}$ is the mean of the response variable.

The sum of squared errors (SSE) is a measure of the residual variation, which represents the sum of squared differences between the observed values of the response variable and its predicted values. The formula for the SSE is:

$$SSE = \sum_{i=1}^{n} (y_i - \hat{y}_i)^2$$

where $y_i$ is the observed value of the response variable for the i-th observation.

The total sum of squares (SST) is the sum of the variation of the response variable and can be calculated as the sum of the regression sum of squares and the sum of squared errors:

$$SST = RSS + SSE$$

\subsection{ Standard error of estimate (SEE), coefficient of determination (R-square), F-statistics }
The standard error of estimate (SEE) is a measure of the variability of the residuals or error terms around the regression line. It is calculated as the square root of the mean squared error (MSE), which is the average of the squared differences between the predicted and actual values. Mathematically, the SEE can be expressed as:

$$SEE = \sqrt{\frac{1}{n-2}\sum_{i=1}^{n}(y_i - \hat{y}_i)^2}$$

where $n$ is the number of observations, $y_i$ is the actual value for the $i^{th}$ observation, and $\hat{y}_i$ is the predicted value for the $i^{th}$ observation.

The coefficient of determination, also known as $R^2$, measures the proportion of the variance in the dependent variable that is explained by the independent variables in the regression model. It ranges from 0 to 1, where 1 indicates that the model explains 100% of the variance in the dependent variable. Mathematically, $R^2$ is defined as:

$$R^2 = 1 - \frac{\sum_{i=1}^{n}(y_i - \hat{y}i)^2}{\sum{i=1}^{n}(y_i - \bar{y})^2}$$

where $\bar{y}$ is the mean of the dependent variable.

The F-statistic is used to test the overall significance of the regression model, including all the independent variables in the model. It is calculated as the ratio of the explained variance to the residual variance, and follows an F-distribution with degrees of freedom equal to the number of independent variables and the number of observations minus the number of independent variables. The F-statistic can be used to determine whether any of the independent variables in the regression model have a significant effect on the dependent variable. Mathematically, the F-statistic can be expressed as:

$$F = \frac{R^2 / k}{(1-R^2) / (n-k-1)}$$

where $k$ is the number of independent variables in the regression model.







\subsection{ Multiple linear regression model. Adjusted R-square, Dummy variables }
In a multiple linear regression model, multiple independent variables are used to explain the variation in the dependent variable. The goal of multiple linear regression is to build a model that accurately predicts the dependent variable based on the values of the independent variables.

The multiple linear regression model can be written as:

$y = \beta_0 + \beta_1x_1 + \beta_2x_2 + ... + \beta_kx_k + \epsilon$

where $y$ is the dependent variable, $\beta_0$ is the intercept, $\beta_1, \beta_2, ..., \beta_k$ are the slope coefficients, $x_1, x_2, ..., x_k$ are the independent variables, and $\epsilon$ is the error term.

Adjusted R-square: In addition to the R-square, which is a measure of goodness of fit, there is also an adjusted R-square that takes into account the number of independent variables in the model. The adjusted R-square is a more accurate measure of the model fit as it penalizes the model for adding irrelevant independent variables. The formula for adjusted R-square is:

$R_{adj}^2 = 1 - (1 - R^2) \frac{n - 1}{n - k - 1}$

where $R^2$ is the R-square, $n$ is the sample size, and $k$ is the number of independent variables in the model.

Dummy Variables: Dummy variables are binary independent variables used in multiple linear regression to account for categorical data. They allow the regression to capture the effect of a categorical variable on the dependent variable. A categorical variable with $k$ categories will be transformed into $k-1$ dummy variables, with one category serving as the reference category. The effect of each category on the dependent variable is captured by its corresponding dummy variable, with the reference category serving as the baseline comparison. The coefficients of the dummy variables in the regression model represent the difference in the dependent variable between the respective category and the reference category.

\subsection{ Serial correlation. Durbin-Watson statistic }
Serial correlation refers to the presence of correlation between the errors or residuals of a time series model. The Durbin-Watson statistic is a test for serial correlation in the residuals of a linear regression model.

The Durbin-Watson statistic is a measure of the autocorrelation of the residuals and ranges from 0 to 4. If the Durbin-Watson statistic is close to 2, it indicates the absence of positive serial correlation. If it is less than 2, it indicates positive serial correlation, and if it is greater than 2, it indicates negative serial correlation. The exact critical values of the Durbin-Watson statistic depend on the sample size, but values close to 0 or 4 are usually considered indicative of significant serial correlation.

In practice, the Durbin-Watson statistic is often used to diagnose and correct for serial correlation in regression models. If significant serial correlation is present, it can lead to incorrect standard errors and hypothesis tests, and it can also impact the accuracy of the predictions made by the model.

Mathematically, the Durbin-Watson statistic is defined as:

$$d = \frac{\sum_{t=2}^T (e_t - e_{t-1})^2}{\sum_{t=1}^T e_t^2}$$

where $e_t$ is the residual at time $t$, and $T$ is the total number of observations in the time series.



\subsection{ Multicollinearity problem }
Multicollinearity is a problem in multiple linear regression when two or more independent variables are highly correlated with each other. This high correlation can cause problems when interpreting the coefficients of the regression model, as the effect of each variable on the dependent variable can be hard to distinguish. In other words, if two or more independent variables are strongly related, it becomes difficult to determine which one is having the largest effect on the dependent variable.

The problem of multicollinearity can lead to unstable coefficient estimates and increase the standard errors, making the coefficients more difficult to interpret. Moreover, it can also increase the risk of overfitting the model, leading to poor predictions.

To diagnose multicollinearity, various methods can be used such as the Variance Inflation Factor (VIF), tolerance, and the condition index. If the problem of multicollinearity is severe, it may require removing one or more of the highly correlated variables, combining them into a single composite variable, or finding alternative independent variables.

In summary, multicollinearity is a common problem in multiple linear regression and must be addressed to ensure the validity and reliability of the results.

In statistics, functions play an important role in understanding relationships between variables. One commonly used function in regression analysis is the linear function, which can be represented as:

$$y = \beta_0 + \beta_1x_1 + \beta_2x_2 + ... + \beta_px_p$$

where $y$ is the response or dependent variable, $\beta_0$ is the intercept, $x_1, x_2, ..., x_p$ are the predictor or independent variables, and $\beta_1, \beta_2, ..., \beta_p$ are the coefficients associated with each predictor variable.

The coefficients represent the change in the response variable for a one unit change in the predictor variable, while holding all other predictor variables constant. The goal in regression analysis is to estimate the coefficients that minimize the differences between the observed values of the response variable and the values predicted by the regression equation.

Another commonly used function in statistical analysis is the logistic function, which is used in logistic regression to model the relationship between a binary response variable and one or more predictor variables. The logistic function can be represented as:

$$p = \frac{1}{1 + e^{-(\beta_0 + \beta_1x_1 + \beta_2x_2 + ... + \beta_px_p)}}$$

where $p$ is the probability of the response variable being equal to 1, and the coefficients $\beta_0, \beta_1, \beta_2, ..., \beta_p$ are estimated in the same way as in linear regression.

Functions can also be used to describe the distribution of a random variable. The probability density function (pdf) describes the continuous distribution of a random variable, while the probability mass function (pmf) describes the discrete distribution of a random variable. For example, the normal distribution is described by the pdf:

$$f(x) = \frac{1}{\sqrt{2\pi\sigma^2}}e^{-\frac{(x - \mu)^2}{2\sigma^2}}$$

where $\mu$ is the mean and $\sigma^2$ is the variance of the distribution. Similarly, the Bernoulli distribution is described by the pmf:

$$f(k) = P(X = k) = p^k(1-p)^{1-k}$$

where $p$ is the probability of success and $k$ is either 0 or 1.









\subsection{ Time-series analysis: linear and log-linear trend models }
Time-series analysis is a statistical method used to analyze and model the behavior of variables over time. A linear trend model assumes that the relationship between the independent variable (time) and the dependent variable is linear. The general form of a linear trend model is:

$y_t = \beta_0 + \beta_1 t + \epsilon_t$

where $y_t$ is the dependent variable, $t$ is time, $\beta_0$ is the intercept, $\beta_1$ is the slope, and $\epsilon_t$ is the error term.

A log-linear trend model, also known as a logarithmic trend model, assumes that the relationship between the logarithms of the independent and dependent variables is linear. The general form of a log-linear trend model is:

$\log(y_t) = \beta_0 + \beta_1 t + \epsilon_t$

where $\log(y_t)$ is the logarithm of the dependent variable, $t$ is time, $\beta_0$ is the intercept, $\beta_1$ is the slope, and $\epsilon_t$ is the error term.

Linear and log-linear trend models are used in time-series analysis to capture and model trends in the data over time. They are simple models that are easy to interpret and estimate, and they provide a useful starting point for more complex models that can be used to analyze and model more complex relationships between variables over time.

\subsection{ Autocorrelations and autoregressive time-series models (AR) }
Autocorrelation is a statistical measure of the dependence between the values of a time series at different time lags. It is used to assess the persistence of the time series and to identify patterns in the data over time.

Autoregressive (AR) time-series models are models that describe the behavior of a time series by regressing it on its own lagged values. The general form of an AR model is:

$y_t = \beta_0 + \beta_1 y_{t-1} + \beta_2 y_{t-2} + ... + \beta_p y_{t-p} + \epsilon_t$

where $y_t$ is the dependent variable, $\beta_0, \beta_1, ..., \beta_p$ are parameters to be estimated, $y_{t-1}, y_{t-2}, ..., y_{t-p}$ are lagged values of the dependent variable, and $\epsilon_t$ is the error term. The integer $p$ is the order of the AR model and determines the number of lagged values of the dependent variable that are used in the regression.

AR models are used in time-series analysis to capture the persistence of the time series, to explain its behavior over time, and to make predictions. They are widely used in many fields, including economics, finance, and engineering, and they provide a simple and intuitive way to model time-series data.

\subsection{ Unit root test of nonstationarity }
A unit root test is a statistical test used to determine if a time series is stationary or non-stationary. Stationarity is an important property of a time series because it means that its mean, variance, and autocorrelation structure do not change over time. Non-stationary time series can be more difficult to model and forecast than stationary time series.

The most widely used unit root test is the Augmented Dickey-Fuller (ADF) test. The ADF test uses regression analysis to test the hypothesis that a time series has a unit root, which is a characteristic of a non-stationary time series. The general form of the regression used in the ADF test is:

$\Delta y_t = \beta_0 + \beta_1 y_{t-1} + \beta_2 y_{t-2} + ... + \beta_p y_{t-p} + \epsilon_t$

where $\Delta y_t = y_t - y_{t-1}$ is the first difference of the time series, $y_t$ is the dependent variable, $\beta_0, \beta_1, ..., \beta_p$ are parameters to be estimated, $y_{t-1}, y_{t-2}, ..., y_{t-p}$ are lagged values of the dependent variable, and $\epsilon_t$ is the error term.

If the null hypothesis that a time series has a unit root is rejected, then the time series is considered to be stationary. If the null hypothesis cannot be rejected, then the time series is considered to be non-stationary.

The ADF test is widely used in time-series analysis to test for stationarity and to determine the appropriate modeling and forecasting techniques to use for non-stationary time series.




\subsection{ Moving-average time-series models (MA) }
Moving-average (MA) time-series models are models that describe the behavior of a time series by regressing it on the errors or residuals from a lagged moving average of the series. The general form of an MA model is:

$y_t = \mu + \epsilon_t + \theta_1 \epsilon_{t-1} + ... + \theta_q \epsilon_{t-q}$

where $y_t$ is the dependent variable, $\mu$ is the mean of the series, $\epsilon_t$ is the error term, and $\theta_1, ..., \theta_q$ are parameters to be estimated. The integer $q$ is the order of the MA model and determines the number of lagged error terms that are used in the regression.

MA models are used in time-series analysis to capture short-term fluctuations or irregular patterns in the data, such as spikes, dips, and outliers. They are particularly useful when the time series is non-stationary and has a non-constant mean and variance over time.

MA models are widely used in many fields, including economics, finance, and engineering, and they provide a simple and intuitive way to model time-series data. When combined with AR models, they form the class of Autoregressive Moving-Average (ARMA) models, which are widely used for modeling and forecasting time-series data.
\subsection{ Seasonality in time-series models }
Seasonality in time-series models refers to the systematic variation in a time series data that occurs at regular intervals, such as daily, weekly, monthly, or yearly. Seasonality is often caused by patterns in demand, production, or consumption that repeat over time, such as holidays, seasons, or monthly payments.

In time-series analysis, seasonality is typically modeled using either additive or multiplicative models, depending on the nature of the data. An additive seasonality model has the form:

$y_t = T_t + S_t + \epsilon_t$

where $y_t$ is the dependent variable, $T_t$ is the trend component, $S_t$ is the seasonality component, and $\epsilon_t$ is the error term. In an additive seasonality model, the seasonal effect is constant over time and is added to the trend component to form the observed value.

A multiplicative seasonality model has the form:

$y_t = T_t \cdot S_t \cdot \epsilon_t$

where $y_t$ is the dependent variable, $T_t$ is the trend component, $S_t$ is the seasonality component, and $\epsilon_t$ is the error term. In a multiplicative seasonality model, the seasonal effect is proportional to the trend component and the observed value is obtained by multiplying the trend and seasonal components.

In both cases, the seasonal component can be modeled using periodic functions, such as sine and cosine, or using a set of dummy variables that represent the different seasons. The choice of model and the number of seasons to include depend on the nature of the data and the goals of the analysis.

Seasonality modeling is important in time-series analysis because it allows for the capture and understanding of the regular patterns in the data, and provides a more accurate representation of the underlying behavior of the series. Seasonality models are widely used in many fields, including economics, finance, and marketing, to analyze and forecast time-series data.
\subsection{ Autoregressive moving-average models (ARMA) and autoregressive conditional heteroskedasticity models (ARCH)  }

Autoregressive Moving Average (ARMA) models are time series models that combine both autoregressive (AR) and moving average (MA) models to describe the evolution of a time series over time. ARMA models are used to analyze and predict time series data by taking into account both the past values of the series and the residuals (the errors) from a moving average model.

An ARMA(p, q) model has the following form:

$y_t = \phi_1 y_{t-1} + \phi_2 y_{t-2} + ... + \phi_p y_{t-p} + \epsilon_t + \theta_1 \epsilon_{t-1} + \theta_2 \epsilon_{t-2} + ... + \theta_q \epsilon_{t-q}$

where $y_t$ is the dependent variable at time $t$, $\epsilon_t$ is the error term, $\phi_1, \phi_2, ..., \phi_p$ are the autoregressive coefficients, and $\theta_1, \theta_2, ..., \theta_q$ are the moving average coefficients.

Autoregressive Conditional Heteroskedasticity (ARCH) models are a class of models used to model time series data with time-varying volatility. ARCH models are used to capture the dynamics of the volatility of a time series and to model the conditional variance of the error term in a time series model.

An ARCH(p) model has the following form:

$\sigma_t^2 = \omega + \alpha_1 \epsilon_{t-1}^2 + \alpha_2 \epsilon_{t-2}^2 + ... + \alpha_p \epsilon_{t-p}^2$

where $\sigma_t^2$ is the conditional variance at time $t$, $\omega$ is a constant, $\epsilon_t$ is the error term, and $\alpha_1, \alpha_2, ..., \alpha_p$ are the ARCH coefficients.

ARMA and ARCH models are widely used in finance, economics, and other fields to analyze and predict time-series data, particularly in the context of volatility forecasting. These models provide a flexible and comprehensive framework for modeling the complex dynamics of time series data, and can be used in combination with other models, such as GARCH (Generalized Autoregressive Conditional Heteroskedasticity), to improve the accuracy of time series predictions.
\subsubsection{GARCH}
Generalized Autoregressive Conditional Heteroskedasticity (GARCH) models are an extension of ARCH models that allow for the modeling of both the mean and the volatility of a time series. GARCH models are used to capture the dynamics of both the expected value and the volatility of a time series, and are widely used in finance, economics, and other fields to analyze and predict time-series data, particularly in the context of volatility forecasting.

A GARCH(p, q) model has the following form:

$y_t = \mu + \epsilon_t$

$\sigma_t^2 = \omega + \alpha_1 \epsilon_{t-1}^2 + \alpha_2 \epsilon_{t-2}^2 + ... + \alpha_p \epsilon_{t-p}^2 + \beta_1 \sigma_{t-1}^2 + \beta_2 \sigma_{t-2}^2 + ... + \beta_q \sigma_{t-q}^2$

where $y_t$ is the dependent variable at time $t$, $\mu$ is the mean, $\epsilon_t$ is the error term, $\sigma_t^2$ is the conditional variance at time $t$, $\omega$ is a constant, $\alpha_1, \alpha_2, ..., \alpha_p$ are the ARCH coefficients, $\beta_1, \beta_2, ..., \beta_q$ are the GARCH coefficients, and $\sigma_{t-1}^2, \sigma_{t-2}^2, ..., \sigma_{t-q}^2$ are the past conditional variances.

GARCH models provide a flexible framework for modeling the complex dynamics of time series data, and can be used in combination with other models, such as ARMA and ARCH, to improve the accuracy of time series predictions. GARCH models have been widely used in finance and economics to analyze financial time-series data, including stock returns, exchange rates, and interest rates, among others.

\subsection{ Cointegrated series. (Engle-Granger) Dickey-Fuller test }
In economics, cointegrated series are time series variables that have a long-run relationship between them, meaning their difference is stationary (i.e., mean-reverting). The Engle-Granger method is a way of testing for cointegration between two or more time series.

The Dickey-Fuller test is a statistical test for checking the stationarity of a time series. It tests the null hypothesis that the time series has a unit root (i.e., it is non-stationary) against the alternative hypothesis that the time series has no unit root (i.e., it is stationary). The test statistic from the Dickey-Fuller test can be used to decide whether or not to reject the null hypothesis, and if it is rejected, the time series is considered to be cointegrated.

The Engle-Granger test is a two-step procedure to test for cointegration between two non-stationary time series.

The first step involves regressing one series on the other to obtain the residuals, which are then tested for stationarity.

The test equation is as follows:

$\Delta y_t = \beta_0 + \beta_1 x_t + u_t$

where $\Delta y_t$ is the first difference of $y_t$, $x_t$ is another time series, and $u_t$ is the residual.

The second step involves testing the residuals for stationarity using the Dickey-Fuller test:

$\Delta u_t = \alpha_0 + \alpha_1 \Delta u_{t-1} + \epsilon_t$

where $\epsilon_t$ is the error term. If the residuals are found to be stationary, it can be concluded that the two series are cointegrated.

The coefficients $\beta_1$ and $\alpha_0$ are estimated using OLS and the resulting t-statistics are used to determine the level of significance. If the t-statistic for $\beta_1$ is significantly different from 0, it can be concluded that there is a long-run relationship between the two series, and if the t-statistic for $\alpha_0$ is significantly different from 0, it can be concluded that the residuals are stationary.




\subsection{ Panel data series} 
Panel data series, also known as cross-sectional time-series data or longitudinal data, is a type of data that combines both cross-sectional and time-series data. It consists of data collected on multiple individuals or entities over multiple time periods. In a panel data series, the same individuals or entities are observed repeatedly over time, and their characteristics and behaviors are recorded at each point in time.

Panel data is used in many different fields, including economics, sociology, marketing, and health care, among others. It provides rich information about individuals or entities and allows researchers to examine the changes in their characteristics and behaviors over time. This makes it possible to study the impact of time-varying variables, such as economic conditions, on individual or entity behavior.

Panel data can be analyzed using various statistical techniques, including fixed effects models, random effects models, and instrumental variables models. The choice of model depends on the research question, the nature of the data, and the assumptions of the model.

Panel data, also known as cross-sectional time-series data, is a type of data structure that combines both cross-sectional and time-series dimensions. It includes multiple observations of individuals, firms, or countries over time.

In a panel data context, the regression model can be represented as:

$$y_{it} = \beta_0 + \beta_1 x_{1it} + \beta_2 x_{2it} + \dots + \beta_k x_{kit} + \epsilon_{it}$$

where $y_{it}$ is the dependent variable for the i-th unit (individual, firm, or country) at time t, $x_{1it}$, $x_{2it}$, ..., $x_{kit}$ are the independent variables, $\beta_0$, $\beta_1$, $\beta_2$, ..., $\beta_k$ are the coefficients to be estimated, and $\epsilon_{it}$ is the error term.

In panel data, we can control for both time-invariant and time-varying variables, as well as deal with issues such as omitted variables bias and endogeneity. The model can be estimated using fixed effects or random effects method.

Example:

Suppose we want to examine the relationship between firm size (represented by the log of total assets) and firm profitability (represented by the return on assets) for a sample of firms over time. The panel data regression model can be represented as:

$$ROA_{it} = \beta_0 + \beta_1 \log(TA_{it}) + \epsilon_{it}$$

where $ROA_{it}$ is the return on assets for the i-th firm at time t, $\log(TA_{it})$ is the log of total assets for the i-th firm at time t, and $\epsilon_{it}$ is the error term. The coefficients $\beta_0$ and $\beta_1$ can be estimated using fixed effects or random effects method.




\section{Calculus }
\subsection{
Limits and derivatives

 }
Limits and derivatives are two fundamental concepts in calculus that play a crucial role in understanding and modeling various phenomena in mathematics, physics, and many other fields.

A limit is a value that a function approaches as the input approaches a particular value. The concept of a limit is used to study the behavior of a function near a specific point, and it is an essential tool for understanding the continuity and differentiability of a function. The limit of a function can be calculated using algebraic and graphical methods.

The derivative of a function is a measure of the rate of change of the function at a specific point. It is calculated by taking the limit of the ratio of the change in the function to the change in the input as the change in the input approaches zero. The derivative of a function provides important information about the local behavior of the function, such as its slope and curvature, and it is used in many applications, such as optimization and finding maximum and minimum values.

The derivative of a function can be calculated using various methods, including the power rule, the product rule, the quotient rule, and the chain rule. In addition, there are also many special functions, such as the trigonometric, exponential, and logarithmic functions, that have well-known derivatives, making it easier to find the derivative of their compositions.

In summary, limits and derivatives are two fundamental concepts in calculus that provide a way to study the behavior of functions near specific points and to measure the rate of change of functions. They play a crucial role in understanding and modeling various phenomena in mathematics and many other fields.





\subsection{ L’Hospital’s rule }
L'Hôpital's rule is a method in calculus used to evaluate the limit of an indeterminate fraction of the form 0/0 or ∞/∞. The rule states that, if the limit of the fraction approaches 0/0 or ∞/∞, and if the derivative of the numerator and denominator exist and are not equal to zero at that point, then the limit of the fraction is equal to the limit of the derivative of the numerator divided by the derivative of the denominator.

L'Hôpital's rule is named after the French mathematician Guillaume de l'Hôpital, who introduced it in his book "Analyse des Infiniment Petits pour l'Intelligence des Lignes Courbes" (1696).

L'Hôpital's rule can be stated as follows:

If lim x→c f(x) = lim x→c g(x) = 0/0 or ∞/∞, and if f'(x) ≠ 0 and g'(x) ≠ 0 in a neighborhood of x = c, then:

lim x→c f(x)/g(x) = lim x→c f'(x)/g'(x)

L'Hôpital's rule is used to evaluate limits that cannot be found using other methods. However, it should be used with caution, as it is not applicable in all cases and may not give the correct answer for some indeterminate forms.

In conclusion, L'Hôpital's rule is a useful tool in calculus for evaluating the limit of indeterminate fractions, but it should be used with care, as it is not always applicable and may not give the correct answer.
\subsection{ Integration }
Integration is a fundamental concept in calculus that deals with finding the area under a curve or the total change in a function over a specified interval. It is the inverse operation of differentiation and plays a crucial role in solving problems in physics, engineering, economics, and many other fields.

The most common way of calculating an integral is by using the Riemann Sum, which approximates the area under the curve as a series of rectangles. The definite integral of a function f(x) over an interval [a,b] is defined as the limit of the Riemann Sum as the width of the rectangles approaches zero:

∫_a^bf(x)dx = lim n→∞ Σ_i=1^nf(x_i)(b-a)/n

where x_i is the right endpoint of the i-th rectangle.

There are various techniques for evaluating integrals, including substitution, integration by parts, and partial fraction decomposition. In addition, there are also many special functions, such as the trigonometric, exponential, and logarithmic functions, that have well-known antiderivatives, making it easier to evaluate their integrals.

The definite integral can be used to calculate various quantities, such as the area under a curve, the average value of a function, and the total change in a function over a specified interval. The indefinite integral, also known as the antiderivative, provides a general formula for the function that has a given derivative.

In summary, integration is a fundamental concept in calculus that provides a way to calculate the total change in a function over a specified interval and to find the general formula for a function that has a given derivative.




\subsection{ Partial derivatives and multiple integrals }
Partial derivatives are derivatives taken with respect to one variable, while holding the other variables constant. They are used to describe the rate of change of a function with respect to one variable, while keeping the other variables fixed.

Let f(x,y) be a function of two variables, x and y. The partial derivative of f with respect to x, denoted as ∂f/∂x, is defined as the rate of change of f with respect to x, while keeping y constant. It can be calculated as follows:

∂f/∂x = limΔx→0 (f(x+Δx,y) - f(x,y))/Δx

Similarly, the partial derivative of f with respect to y, denoted as ∂f/∂y, is defined as the rate of change of f with respect to y, while keeping x constant. It can be calculated as follows:

∂f/∂y = limΔy→0 (f(x,y+Δy) - f(x,y))/Δy

Partial derivatives are used in various fields such as engineering, physics, and economics to describe the behavior of functions and to solve optimization problems.

Multiple integrals are integrals taken with respect to multiple variables. They are used to calculate the volume, area, or mass of a region in space, or to calculate the expected value of a random variable in probability theory.

There are two main types of multiple integrals: double integrals and triple integrals.

A double integral is an integral taken with respect to two variables, x and y. It can be written as:

∫∫f(x,y)dxdy

A triple integral is an integral taken with respect to three variables, x, y, and z. It can be written as:

∫∫∫f(x,y,z)dxdydz

Multiple integrals are used in various fields such as physics, engineering, and computer graphics to model and solve real-world problems.

\subsection{ Taylor’s series }
The Taylor series is a powerful tool in mathematical analysis and numerical analysis that provides a way to approximate a function as a polynomial. It is named after the mathematician Brook Taylor, who first introduced it in the early 18th century.

The Taylor series of a function f(x) about a point a is given by:

f(x) = f(a) + f'(a)(x-a) + (f''(a)/2!) (x-a)^2 + (f'''(a)/3!) (x-a)^3 + ... + (f^n(a)/n!) (x-a)^n + ...

where f^n(a) represents the nth derivative of f evaluated at a, and n! is the factorial of n.

The Taylor series provides a way to approximate a function by using its derivatives evaluated at a single point, a. By including more and more terms in the series, the approximation becomes more accurate. However, for many functions, the Taylor series may not converge to the actual function for all values of x.

The Taylor series is commonly used in numerical analysis for approximating functions, solving differential equations, and in other areas of mathematics and science. For example, it can be used to approximate the value of functions that are difficult to calculate directly, such as exponential, logarithmic, and trigonometric functions.




\subsection{ Newton’s method }
Newton's method is an iterative algorithm used to find the roots of a real-valued function. It is a fast and efficient method for finding the approximate roots of a function and is widely used in optimization and numerical analysis.

Here is how the method works:

Start with an initial guess for the root of the function, x0.

Compute the first derivative of the function, f'(x), at x0.

Compute the equation of the tangent line to the function at x0, which is given by: y = f'(x0)(x-x0) + f(x0)

Find the x-intercept of the tangent line, which is the root of the equation y = 0.

Update the guess for the root, x0, by setting it equal to the x-intercept of the tangent line.

Repeat steps 2 to 5 until the desired accuracy is achieved or a maximum number of iterations has been reached.

It's important to note that Newton's method can converge quickly to a root if the initial guess is close to the true root. However, it can also diverge if the initial guess is far from the root or if the function is not well-behaved (e.g., has a singularity or is not differentiable). In such cases, it's often necessary to choose a different method or a different initial guess.

\subsubsection{example}
Sure, here's an example of using Newton's method to find a root of the function f(x) = x^2 - 2.

Choose an initial guess, x0. For this example, let's use x0 = 1.

Compute the derivative of the function at x0, which is f'(x) = 2x. So, at x0 = 1, f'(1) = 2.

Compute the equation of the tangent line at x0, which is y = 2x - 1.

Find the x-intercept of the tangent line, which is the root of the equation y = 0. Solving for x, we find that x = 1/2.

Update the guess for the root by setting x0 = 1/2.

Repeat steps 2 to 5 until the desired accuracy is achieved or a maximum number of iterations has been reached. For this example, let's perform one more iteration.

At x0 = 1/2, f'(1/2) = 1. The equation of the tangent line at x0 = 1/2 is y = x + 1/2.

The x-intercept of the tangent line is x = -1/2. So, the updated guess for the root is x0 = -1/2.

After two iterations, the Newton's method has found an approximate root of x = -1/2 with a value of f(-1/2) = 2.25, which is very close to the true root of x = sqrt(2).

Note that in general, it is recommended to use a stopping criteria to determine when to stop the iterations, such as when the absolute value of the difference between two consecutive guesses is below a certain tolerance, or when the value of the function at the current guess is close enough to zero.





\subsection{ Lagrange multipliers}
Lagrange multipliers are a method for solving optimization problems with constraints. The basic idea is to form a new function, called the Lagrangian, which involves both the objective function and the constraints. The maxima and minima of the Lagrangian occur at the same points as the maxima and minima of the original objective function, subject to the constraints.

The Lagrangian is defined as:

$$L(x_1,x_2,\ldots,x_n,\lambda_1,\lambda_2,\ldots,\lambda_m) = f(x_1,x_2,\ldots,x_n) + \sum_{i=1}^m \lambda_i g_i(x_1,x_2,\ldots,x_n)$$

where $f(x_1,x_2,\ldots,x_n)$ is the objective function to be optimized, $g_i(x_1,x_2,\ldots,x_n)$ are the constraint functions, and $\lambda_i$ are the Lagrange multipliers.

The solution to the optimization problem is obtained by finding the critical points of the Lagrangian, i.e., the points where the partial derivatives with respect to each of the variables $x_1,x_2,\ldots,x_n,\lambda_1,\lambda_2,\ldots,\lambda_m$ are equal to zero. The critical points are then evaluated to determine which are the maximum or minimum values of the Lagrangian and hence the objective function, subject to the constraints.

The method of Lagrange multipliers is a powerful tool for solving a wide range of optimization problems and is widely used in many fields, including physics, engineering, and economics.




\section{Linear Algebra  }
\subsection{ Basic operations with matrices and vectors: addition, multiplication, transposition }
Matrices and vectors are mathematical objects that are used to represent and manipulate data. There are several basic operations that can be performed with matrices and vectors, including addition, multiplication, and transposition. In this article, we will explore these operations in detail.

Addition:

Addition is a simple operation that involves adding two matrices or vectors to obtain a new matrix or vector. The matrices or vectors that are being added must have the same dimensions, meaning that the number of rows and columns must be the same. In the case of matrices, the addition is performed element-wise, meaning that the elements in the same position in each matrix are added together to form a new matrix. For example, if we have two matrices A and B, their sum would be represented by C = A + B.

Multiplication:

Multiplication is a more complex operation that involves multiplying two matrices or vectors. In the case of matrices, the multiplication is performed using matrix multiplication rules. Matrix multiplication involves multiplying the elements of one matrix by the corresponding elements of another matrix, and then summing the products. For example, if we have two matrices A and B, their product would be represented by C = AB. In the case of vectors, the multiplication is performed using scalar multiplication, where a vector is multiplied by a scalar value to obtain a new vector.

Transposition:

Transposition is an operation that involves flipping a matrix or vector over its diagonal. In other words, the rows become columns and the columns become rows. This operation is represented by a superscript T and is used to convert row vectors into column vectors or vice versa. For example, if we have a matrix A, its transpose would be represented by A^T.

In conclusion, addition, multiplication, and transposition are basic operations that are used to manipulate matrices and vectors. These operations allow us to perform various mathematical tasks and are widely used in various fields, including engineering, computer science, and physics.
\subsection{ Linear dependence, basis and dimension }
Linear dependence and independence are important concepts in linear algebra that are used to describe the relationship between vectors. A set of vectors is said to be linearly dependent if there is a non-trivial linear combination of these vectors that results in the zero vector. On the other hand, a set of vectors is said to be linearly independent if no non-trivial linear combination of these vectors results in the zero vector.

Basis is a set of linearly independent vectors that span a vector space. In other words, any vector in the vector space can be represented as a linear combination of the basis vectors. The number of basis vectors is called the dimension of the vector space. For example, in a two-dimensional space, the basis vectors would be two linearly independent vectors that can span the entire space.

The dimension of a vector space is a measure of its size. For example, a two-dimensional vector space is smaller than a three-dimensional vector space. The dimension of a vector space can also be used to describe the number of degrees of freedom that a vector in that space has.

In mathematical terms, these concepts can be expressed as follows:

A set of vectors ${v_1, v_2, ..., v_n}$ is linearly dependent if there exist scalars ${c_1, c_2, ..., c_n}$, not all equal to zero, such that

$$\sum_{i=1}^n c_iv_i = \mathbf{0}$$

A set of vectors ${v_1, v_2, ..., v_n}$ is linearly independent if for every set of scalars ${c_1, c_2, ..., c_n}$, not all equal to zero, we have

$$\sum_{i=1}^n c_iv_i \ne \mathbf{0}$$

A set of vectors ${v_1, v_2, ..., v_n}$ is a basis for a vector space $V$ if

The set is linearly independent
The set spans $V$
The dimension of a vector space $V$ is the number of basis vectors in any basis for $V$.

\subsection{ Determinant. Properties and signs. Eigenvalues and vectors }
The determinant of a square matrix is a scalar value that describes certain properties of the matrix. It is a useful tool for solving linear equations, computing inverses, and determining the stability of linear systems. The determinant is denoted by the symbol $|A|$ for a matrix $A$.

The properties of the determinant include:

Linearity: If $A$ and $B$ are square matrices, then $|A + B| = |A| + |B|$ and $|AB| = |A| \cdot |B|$.

Multiplicative Identity: If $A$ is a square matrix and $I$ is the identity matrix of the same size, then $|AI| = |A|$.

Singularity: A square matrix is singular if and only if its determinant is equal to zero.

The sign of the determinant is related to the orientation of the vectors in the matrix. If the determinant is positive, the vectors are said to be oriented in a positive direction, while a negative determinant indicates that the vectors are oriented in a negative direction.

Eigenvalues and eigenvectors are also important concepts in linear algebra. An eigenvalue is

a scalar value that represents the amount of stretching or shrinking that occurs when a linear transformation is performed on a vector. An eigenvector is a non-zero vector that is stretched or shrunk by a factor equal to the corresponding eigenvalue when a linear transformation is performed on it.

Eigenvalues and eigenvectors can be used to diagonalize a matrix, which is a useful tool for solving systems of linear equations. If a matrix is diagonalizable, it can be transformed into a diagonal matrix whose diagonal entries are the eigenvalues of the original matrix.

The eigenvalue-eigenvector relationship can be expressed mathematically as follows:

If $A$ is a square matrix and $\lambda$ is an eigenvalue, then a non-zero vector $x$ is an eigenvector corresponding to $\lambda$ if

$$A x = \lambda x$$

In other words, the eigenvector $x$ is an eigenvector of $A$ with eigenvalue $\lambda$ if it satisfies the above equation. This relationship can be used to find the eigenvalues and eigenvectors of a matrix, which can then be used to solve various problems in linear algebra.

\subsection{ Inverse matrix, matrix invertibility criterion, inversion and transposition }
The inverse of a matrix is a matrix that when multiplied with the original matrix gives the identity matrix. If a matrix $A$ has an inverse, it is denoted as $A^{-1}$. The inverse of a matrix is an important concept in linear algebra as it can be used to solve linear equations and perform various other operations.

There is a criterion for checking the invertibility of a matrix known as the invertibility criterion. A square matrix $A$ is invertible if and only if its determinant is non-zero. In other words, if $|A| \ne 0$, then $A$ is invertible and has an inverse.

The process of finding the inverse of a matrix is known as inversion. The inverse of a matrix can be found using various methods such as the Gaussian elimination method, the Gauss-Jordan method, and the matrix adjugate method.

The transposition of a matrix is a process of flipping a matrix over its main diagonal. If $A$ is a matrix, its transpose is denoted as $A^T$. The transpose of a matrix is important in linear algebra as it can be used to find the transpose of a product of matrices, which is useful in solving systems of linear equations.

The transposition of a matrix can also be used to find the transpose of a vector, which is important in vector calculus. The transpose of a vector is often denoted as a row vector, while the original vector is denoted as a column vector. The transpose of a vector can be used to perform various operations such as taking the dot product or cross product of two vectors.





\subsection{ Moore-Penrose pseudoinverse matrix. Skeleton decomposition. Ordinary Least Squares. Singular value decomposition }
The Moore-Penrose pseudoinverse matrix, also known as the generalized inverse, is a matrix that can be used when a matrix is not invertible. It is a generalization of the concept of inverse matrices and can be used to solve linear equations when the original matrix is not invertible.

The skeleton decomposition is a decomposition of a matrix into a product of three matrices. It is a useful tool for solving linear equations and performing various other operations. The skeleton decomposition can be used to find the singular value decomposition of a matrix, which is an important tool in linear algebra.

The ordinary least squares (OLS) method is a method for finding the best-fitting line or curve to a set of data. It is often used in regression analysis and can be used to find the best-fitting line or curve that minimizes the sum of the squares of the residuals.

The singular value decomposition (SVD) is a factorization of a matrix into the product of three matrices. The SVD is an important tool in linear algebra and is used in various applications such as image compression, data compression, and recommendation systems.

The SVD can be used to find the rank of a matrix, which is the number of non-zero singular values of the matrix. The rank of a matrix is an important concept in linear algebra and is used to determine the dimensionality of a matrix. The SVD can also be used to find the null space of a matrix, which is the set of vectors that are mapped to the zero vector by the matrix.





\subsection{ Metrics and norms of vectors. Hölder norm. Euclidean norm }
A metric is a measure of the distance between two points. In linear algebra, metrics and norms are used to measure the size or magnitude of vectors. Metrics and norms are important concepts in linear algebra as they can be used to find the shortest distance between two points, the magnitude of a vector, and the angle between two vectors.

There are many types of norms, but two of the most commonly used norms are the Hölder norm and the Euclidean norm.

The Hölder norm is a type of norm that is used to measure the size of a vector. It is defined as

$$\left|x\right|p = \left(\sum\limits{i=1}^{n} \left|x_i\right|^p\right)^{1/p}$$

where $p$ is a real number greater than 0 and $x$ is a vector. The Hölder norm is often used in various applications such as image processing and signal processing.

The Euclidean norm, also known as the 2-norm, is a type of norm that is used to measure the magnitude of a vector. It is defined as

$$\left|x\right|2 = \sqrt{\sum\limits{i=1}^{n} x_i^2}$$

where $x$ is a vector. The Euclidean norm is often used in various applications such as linear regression, PCA, and SVM.

In addition to the Hölder norm and the Euclidean norm, there are many other types of norms that can be used to measure the size of a vector. The choice of norm depends on the specific application and the type of data being analyzed.
\subsection{ Linear spaces. Euclidean space. Scalar product. Orthogonal systems }
Linear spaces are mathematical objects that contain a set of vectors and a set of scalars. In linear spaces, vectors can be added and scalars can be multiplied with vectors. The sum of two vectors in a linear space is also a vector in the same space, and scalar multiplication of a vector preserves the structure of the space.

The Euclidean space is a linear space that is equipped with a scalar product, also known as the dot product. The scalar product is a mathematical operation that takes two vectors as input and produces a scalar as output. The scalar product is defined as

$$x\cdot y = \sum\limits_{i=1}^{n} x_i y_i$$

where $x$ and $y$ are two vectors. The scalar product is used to find the angle between two vectors, the projection of one vector onto another vector, and to find the length of a vector.

An orthogonal system is a set of vectors that are orthogonal to each other, meaning that the scalar product of any two vectors in the system is equal to zero. Orthogonal systems are used in various applications such as signal processing, image processing, and linear regression.

Orthogonal systems have many useful properties such as being linearly independent and spanning a subspace. A linearly independent system is a system in which no vector can be expressed as a linear combination of the other vectors in the system. A spanning system is a system that can be used to represent any vector in a given space.

Orthogonal systems are also used in the orthogonal decomposition of a vector. In this decomposition, a vector is expressed as the sum of two orthogonal vectors, one of which is the projection of the vector onto a subspace and the other of which is orthogonal to the subspace. This decomposition is used in various applications such as linear regression and principal component analysis.

\subsection{ Matrix norms and induced norms }
Matrix norms and induced norms are mathematical concepts that are used to measure the size of matrices. Matrix norms are a measure of the size of a matrix and induced norms are a measure of the size of a linear transformation.

Matrix norms are defined as

$$\left|A\right| = \sup\limits_{x\neq0} \frac{\left|Ax\right|}{\left|x\right|}$$

where $A$ is a matrix and $x$ is a vector. There are many different types of matrix norms, including the Frobenius norm, the spectral norm, and the induced norm.

The Frobenius norm is a type of matrix norm that is defined as

$$\left|A\right|F = \sqrt{\sum\limits{i=1}^{m}\sum\limits_{j=1}^{n} |a_{ij}|^2}$$

where $A$ is a matrix and $a_{ij}$ are the elements of the matrix. The Frobenius norm is used in various applications such as matrix factorization and singular value decomposition.

The spectral norm is a type of matrix norm that is defined as the largest singular value of a matrix. The spectral norm is used to find the largest eigenvalue of a matrix and is used in various applications such as numerical linear algebra and optimization.

Induced norms are a measure of the size of a linear transformation. Induced norms are defined as

$$\left|T\right|{\text{induced}} = \sup\limits{x\neq0} \frac{\left|T(x)\right|}{\left|x\right|}$$

where $T$ is a linear transformation and $x$ is a vector. Induced norms are used in various applications such as optimization and control theory.

In summary, matrix norms and induced norms are mathematical concepts that are used to measure the size of matrices and linear transformations. They are used in various applications such as numerical linear algebra, optimization, and control theory.





\subsection{ Singularity and spectral radius. Low-rank approximation}
Singularity and spectral radius are mathematical concepts that are used to describe the properties of matrices. Singularity refers to the property of a matrix that makes it impossible to find a unique solution to a system of linear equations. Spectral radius refers to the largest eigenvalue of a matrix.

A matrix is singular if its determinant is equal to zero. Singular matrices are not invertible, which means that it is impossible to find a unique solution to a system of linear equations when the coefficient matrix is singular. In other words, when a matrix is singular, there are infinitely many solutions to the system of linear equations.

The spectral radius of a matrix is the largest eigenvalue of the matrix. The spectral radius is used to find the rate of convergence of iterative algorithms and is used in various applications such as numerical linear algebra and control theory.

Low-rank approximation is a technique used to reduce the rank of a matrix while preserving its important properties. Low-rank approximation is used in various applications such as data compression, machine learning, and signal processing.

Low-rank approximation is performed by finding the singular value decomposition of a matrix and then keeping only the largest singular values. The resulting matrix is a low-rank approximation of the original matrix.

In summary, singularity, spectral radius, and low-rank approximation are mathematical concepts that are used to describe the properties of matrices. Singularity refers to the property of a matrix that makes it impossible to find a unique solution to a system of linear equations, spectral radius refers to the largest eigenvalue of a matrix, and low-rank approximation is a technique used to reduce the rank of a matrix while preserving its important properties.




\section{Probability Theory}
\subsection{ Basic probability definitions and set operations: outcome, sample space / probability space, event, mutually exclusive, exhaustive events, random variable  }
Probability is a mathematical framework used to quantify the uncertainty associated with events. In probability theory, several basic concepts are used to define and analyze random events and processes.

Outcome: An outcome is a possible result of an experiment or a random event. For example, the outcome of a coin flip can be either "heads" or "tails."

Sample Space: A sample space is a set of all possible outcomes of an experiment or a random event. For example, the sample space for the coin flip experiment is {heads, tails}.

Event: An event is a collection of one or more outcomes in the sample space. For example, the event "getting heads" in the coin flip experiment is {heads}.

Mutually Exclusive: Two events are said to be mutually exclusive if they have no outcomes in common. For example, the events "getting heads" and "getting tails" are mutually exclusive in the coin flip experiment.

Exhaustive Events: A set of events is said to be exhaustive if the sample space is completely covered by these events. For example, in the coin flip experiment, the events "getting heads" and "getting tails" are exhaustive.

Random Variable: A random variable is a function that maps an outcome in the sample space to a real number. For example, the random variable X that maps the outcome "heads" to the value 1 and the outcome "tails" to the value 0 is a binary random variable.

In summary, these basic probability definitions and set operations are used to define and analyze random events and processes. The outcome, sample space, event, mutually exclusive events, exhaustive events, and random variable are the key concepts in probability theory and are widely used in various areas, including statistics, data analysis, and machine learning.
\subsection{ Combinatorial analysis: permutations, combination, binomial theorem, Inclusion-Exclusion Principle }
Combinatorial analysis is a branch of mathematics concerned with counting, arranging, and analyzing objects in various combinations. The following are some of the key concepts in combinatorial analysis:

Permutations: A permutation is an ordered arrangement of objects from a set. For example, the set {1, 2, 3} has 6 permutations: (1, 2, 3), (1, 3, 2), (2, 1, 3), (2, 3, 1), (3, 1, 2), and (3, 2, 1). The number of permutations of a set of n objects is given by n! (n factorial), where n! = n × (n - 1) × (2) × ... × 3 × 2 × 1.

Combination: A combination is a selection of objects from a set without regard to their order. For example, the set {1, 2, 3} has 3 combinations of 2 objects: (1, 2), (1, 3), and (2, 3). The number of combinations of n objects taken k at a time is given by C(n, k) = n! / (k! (n - k)!).

Binomial Theorem: The binomial theorem states that for any real numbers a and b and any nonnegative integer n, the expression (a + b)^n can be expanded as a sum of terms of the form C(n, k) a^(n - k) b^k, where k = 0, 1, 2, ..., n.

Inclusion-Exclusion Principle: The Inclusion-Exclusion Principle is a combinatorial principle used to find the number of elements in the union of two or more sets. The principle states that the number of elements in the union of two sets A and B is equal to the sum of the number of elements in each set minus the number of elements in their intersection.

In summary, these concepts in combinatorial analysis are widely used in various areas, including mathematics, computer science, and engineering. They are essential tools for solving counting problems, analyzing patterns and structures in data, and designing algorithms.
\subsection{ Unconditional and conditional probability. Joint probability }
Unconditional and conditional probability are fundamental concepts in probability theory.

Unconditional Probability: Unconditional probability refers to the probability of an event occurring without any prior knowledge or information. For example, the probability of rolling a six on a fair six-sided die is 1/6.

Conditional Probability: Conditional probability is the probability of an event occurring given that another event has already occurred. For example, the probability of rolling a six on a fair six-sided die given that a 5 has already been rolled is 1/6, since the die is still fair and all outcomes are equally likely. The formula for conditional probability is P(A|B) = P(A and B) / P(B), where P(A|B) is the probability of event A given that event B has occurred, P(A and B) is the probability of both events A and B occurring, and P(B) is the probability of event B occurring.

Joint Probability: Joint probability refers to the probability of two or more events occurring simultaneously. For example, the joint probability of rolling a 5 and a 6 on a fair six-sided die is 0, since it is impossible for both of these events to occur at the same time. The formula for joint probability is P(A and B) = P(A|B)P(B) = P(B|A)P(A), where P(A and B) is the joint probability of events A and B, P(A|B) is the conditional probability of event A given that event B has occurred, and P(B|A) is the conditional probability of event B given that event A has occurred.

In summary, understanding these concepts is crucial for analyzing and making predictions about random events and for understanding the relationships between events in a probability model.




\subsection{ Law of total probability }
The law of total probability is a fundamental theorem in probability theory. It states that the total probability of all possible outcomes in a sample space is equal to 1.

Let's say we have a sample space S and a partition of S into disjoint events B1, B2, ..., Bn. The law of total probability states that for any event A in S:

P(A) = P(B1 and A) + P(B2 and A) + ... + P(Bn and A) = P(B1)P(A|B1) + P(B2)P(A|B2) + ... + P(Bn)P(A|Bn)

where P(A) is the probability of event A, P(B1 and A) is the probability of both events B1 and A occurring, P(B1) is the probability of event B1 occurring, and P(A|B1) is the conditional probability of event A given that event B1 has occurred.

The law of total probability is useful in many real-world applications, such as in Bayesian statistics, where it can be used to calculate the overall probability of a hypothesis by taking into account all possible sources of evidence.

In summary, the law of total probability provides a way to calculate the probability of an event by dividing the sample space into smaller, more manageable parts and taking into account the probabilities of these parts and their intersections with the event of interest.




\subsection{ Bayes’ formula }
Bayes' formula is a fundamental theorem in probability theory that relates the prior probability of an event to the likelihood of the event and the posterior probability of the event. It provides a way to update the probabilities of events in light of new information.

The formula is named after Thomas Bayes, an 18th-century statistician and theologian, who first developed the idea of conditional probability.

Bayes' formula is written as:

P(A|B) = (P(B|A) * P(A)) / P(B)

where:

P(A|B) is the posterior probability of event A given that event B has occurred.
P(B|A) is the likelihood of event B given that event A has occurred.
P(A) is the prior probability of event A.
P(B) is the probability of event B.
Bayes' formula can be used to calculate the probability of a hypothesis (represented by event A) given some observed data (represented by event B). The prior probability represents the belief in the hypothesis before taking the data into account, while the likelihood represents the degree to which the data supports the hypothesis. The posterior probability represents the updated belief in the hypothesis after taking the data into account.

Bayes' formula is widely used in many fields, including machine learning, natural language processing, and computer vision, where it is used to classify and predict outcomes based on data. It is also used in medical diagnosis, where it can be used to calculate the probability of a disease given a patient's symptoms, or in spam filtering, where it can be used to calculate the probability that an email is spam given its content.

In summary, Bayes' formula is a powerful tool for updating beliefs and making predictions based on new information, and it has many applications in a wide range of fields.




\subsection{ Discrete and continuous distributions: common function of random variables, discrete random variables, continuous random variables}
A random variable is a function that maps outcomes of a random experiment to real values. Depending on the nature of the experiment, a random variable can either be discrete or continuous.

Discrete random variables take on a countable number of values. For example, the number of heads in a series of coin flips is a discrete random variable. The probability distribution of a discrete random variable is represented by a probability mass function (pmf), which gives the probability of each value that the random variable can take on. The pmf must satisfy the following properties:

The probabilities are non-negative: p(x) >= 0 for all x.
The sum of the probabilities is 1: ∑x p(x) = 1.
Continuous random variables take on an uncountable number of values in a range. For example, the height of a person is a continuous random variable. The probability distribution of a continuous random variable is represented by a probability density function (pdf), which gives the density of the values that the random variable can take on. The pdf must satisfy the following properties:

The density is non-negative: f(x) >= 0 for all x.
The area under the density curve is 1: ∫x f(x) dx = 1.
Common functions of random variables describe the relationship between two or more random variables. For example, the sum or product of two random variables is a function of random variables. The probability distribution of a function of random variables can be found using the cumulative distribution function (cdf), which gives the cumulative probability of the values that the random variable can take on.

Discrete and continuous random variables are the two main types of random variables and each have their own distinct properties and uses. Understanding the properties and characteristics of these different types of random variables is fundamental to many areas of probability theory and statistics.





\section{Stochastic Process  }
\subsection{ Definition of a (stochastic) random process. Trajectory and finite-dimensional distribution of a random process }
A random process, also known as a stochastic process, is a mathematical model for a sequence of random variables defined over a common probability space. The idea behind a random process is to describe a situation where the outcome of a system is uncertain and is described by a set of random variables.

A random process can be thought of as a function from a set of indices (e.g., time) to a set of possible outcomes (e.g., real numbers). For example, a random process could be a sequence of daily temperatures at a specific location, where the index is time (day) and the possible outcome is temperature (real number).

A trajectory of a random process is a single, specific realization of the random process. It is a sequence of random variables defined over a set of indices. For example, if we consider the daily temperature random process, a trajectory could be a sequence of temperatures for a specific location over a period of time, such as one week.

The finite-dimensional distribution of a random process is a distribution that describes the joint probability of a finite number of random variables at specific indices. It is a mathematical representation of the distribution of a specific set of values of the random process. For example, the finite-dimensional distribution of the daily temperature random process could describe the joint probability of the temperatures at specific days, such as Monday, Tuesday, and Wednesday.

In summary, a random process is a mathematical model for a sequence of random variables defined over a common probability space. A trajectory of a random process is a single, specific realization of the random process, and the finite-dimensional distribution of a random process is a distribution that describes the joint probability of a finite number of random variables at specific indices. These concepts play a crucial role in the study of random phenomena and the modeling of uncertain systems in various fields, such as engineering, finance, and communication systems.




\subsection{ Markov chain }
A Markov chain is a type of random process that models a system where the future state of the system depends only on the current state, and not on any of the previous states. It is a mathematical model that is widely used in many fields, such as physics, economics, and computer science.

A Markov chain is defined by a set of states and transition probabilities between the states. The transition probabilities describe the probability of moving from one state to another in a single step. A Markov chain is said to have the Markov property if the probability of moving from one state to another depends only on the current state and not on any of the previous states.

The behavior of a Markov chain can be described by its transition matrix, which is a square matrix that describes the transition probabilities between all pairs of states. The eigenvalues and eigenvectors of the transition matrix play a crucial role in understanding the behavior of the Markov chain, such as its long-term behavior and stability.

In addition to the transition matrix, other important concepts in the study of Markov chains include stationary distributions, ergodicity, and absorbing states. Stationary distributions are probability distributions over the states that remain constant over time, regardless of the starting state. Ergodicity refers to the property of a Markov chain that, starting from any state, the average of the sequence of states over time converges to the stationary distribution. Absorbing states are states from which the system cannot leave, and they play a crucial role in the analysis of Markov chains with absorbing states, such as models of failure and repair systems.

In summary, a Markov chain is a type of random process that models a system where the future state of the system depends only on the current state. It is widely used in many fields and can be described by its transition matrix and other concepts, such as stationary distributions, ergodicity, and absorbing states.
\subsubsection{formulas}
A Markov chain can be described mathematically as follows:

Let S be the set of states, and let P be the transition probability matrix, where P_{ij} is the probability of transitioning from state i to state j. The transition probability matrix satisfies the following properties:

The entries in each row of the matrix are non-negative and sum to 1, since the probabilities must be valid and represent a complete distribution:
$$ \forall i \in S, \quad \sum_{j\in S} P_{ij} = 1 $$

The entries in each column of the matrix are non-negative:
$$ \forall i,j \in S, \quad P_{ij} \ge 0 $$

The long-term behavior of a Markov chain can be described by its stationary distribution, which is a probability distribution over the states that remains constant over time:

$$ \pi = \pi P $$

where $\pi$ is a row vector representing the stationary distribution and $P$ is the transition probability matrix.

A Markov chain is ergodic if it has a unique stationary distribution and starting from any state, the average of the sequence of states over time converges to the stationary distribution:

$$ \lim_{n\to\infty} \frac{1}{n} \sum_{k=0}^{n-1} \bold{x} P^k = \pi $$

where $\bold{x}$ is a row vector representing the initial state and $P^k$ is the transition probability matrix raised to the power of k.

In an absorbing Markov chain, there is a set of states called absorbing states, from which the system cannot leave. The probability of reaching an absorbing state from any state can be calculated using the fundamental matrix, which is the inverse of the identity matrix minus the transition matrix for the non-absorbing states:

$$ N = (I - Q)^{-1} $$

where $I$ is the identity matrix, $Q$ is the transition matrix for the non-absorbing states, and $N$ is the fundamental matrix. The entries of the fundamental matrix represent the expected number of steps to reach an absorbing state starting from a given state.

In summary, these are some of the key mathematical concepts and formulas used to describe a Markov chain.




\subsection{ Martingale and random walk }
Martingale and random walk are related concepts in probability theory.

A martingale is a mathematical object that is used to model a sequence of random variables that satisfy a certain condition. Specifically, a sequence of random variables is said to be a martingale if the expected value of each variable given the history of all previous variables is equal to the value of the previous variable. That is,

$$ E[X_n | X_{n-1}, X_{n-2}, \dots, X_1] = X_{n-1} $$

for all $n \ge 1$. This property ensures that the expectation of the future variables in the sequence depends only on the current value and not on the past values. Martingales are widely used in finance, as they can be used to model the price of a stock or other financial asset over time.

A random walk is a special type of martingale that models a sequence of steps in which each step is taken in a randomly chosen direction. For example, a random walk on a grid can model a particle moving randomly in the North, South, East, or West direction at each step. Random walks are widely used in many fields, such as physics, economics, and computer science, to model the behavior of systems that exhibit randomness.

A simple random walk is a random walk where each step has the same length, and the direction of each step is equally likely to be any of the four directions. The simplest example of a random walk is a fair coin toss, where the outcome of each coin toss determines the direction of the next step in the random walk.

In summary, martingale and random walk are related concepts in probability theory that are used to model sequences of random variables and random processes. A martingale is a sequence of random variables that satisfies the condition that the expected value of each variable given the history of all previous variables is equal to the value of the previous variable. A random walk is a special type of martingale that models a sequence of steps in which each step is taken in a randomly chosen direction.




\subsection{ Brownian motion }
Brownian motion, also known as a Wiener process, is a type of random process that models the random movement of a particle due to the continuous collisions with its surrounding particles. This process was first described by the botanist Robert Brown in 1827, and later formalized mathematically by the mathematician Norbert Wiener in the early 20th century.

In mathematical terms, Brownian motion is a continuous-time stochastic process that models the random movement of a particle over time. It is defined as a Gaussian process, which means that its increments are normally distributed and have mean 0. This can be formalized as:

$$ W(t) - W(s) \sim \mathcal{N}(0,t-s) $$

where $W(t)$ is the value of the Brownian motion process at time t, and $\mathcal{N}(0,t-s)$ represents a normal distribution with mean 0 and variance $t-s$.

Brownian motion is widely used in many fields, such as finance, physics, and engineering, to model a wide range of random processes and phenomena, such as the diffusion of particles in a fluid, the random movement of stock prices, and the noise in electronic signals.

In summary, Brownian motion, also known as a Wiener process, is a type of random process that models the random movement of a particle due to the continuous collisions with its surrounding particles. It is defined as a Gaussian process, with increments that are normally distributed and have mean 0. Brownian motion is widely used in many fields to model a wide range of random processes and phenomena.
\subsection{ Weiner process }
The Wiener process, also known as Brownian motion, is a continuous-time stochastic process that models the random movement of a particle over time. It was first described by the botanist Robert Brown in 1827, and later formalized mathematically by the mathematician Norbert Wiener in the early 20th century.

A Wiener process is characterized by the following properties:

It is a continuous-time process, meaning that the value of the process can change continuously over time.

It is a Markov process, meaning that the future state of the process depends only on its current state and not on its past history.

The increments of the process are normally distributed with mean 0 and variance equal to the length of the time interval. That is,

$$ W(t) - W(s) \sim \mathcal{N}(0,t-s) $$

where $W(t)$ is the value of the Wiener process at time $t$, and $\mathcal{N}(0,t-s)$ represents a normal distribution with mean 0 and variance $t-s$.

The Wiener process has many important applications in many fields, such as finance, physics, and engineering. In finance, it is used to model the random movement of stock prices, interest rates, and other financial quantities. In physics, it is used to model the diffusion of particles in a fluid and the random movement of electrons in a conductor. In engineering, it is used to model the noise in electronic signals and other random processes.

In summary, the Wiener process, also known as Brownian motion, is a continuous-time stochastic process that models the random movement of a particle over time. It is a Markov process with normally distributed increments with mean 0 and variance equal to the length of the time interval. The Wiener process has many important applications in finance, physics, engineering, and other fields.
\subsubsection{function}
Sure! Here are some important formulas related to the Wiener process:

The definition of the Wiener process:
$$ W(t) - W(s) \sim \mathcal{N}(0,t-s) $$

where $W(t)$ is the value of the Wiener process at time $t$, and $\mathcal{N}(0,t-s)$ represents a normal distribution with mean 0 and variance $t-s$.

The mean and variance of the Wiener process:
$$ \mathbb{E}[W(t)] = 0 $$

$$ \text{Var}[W(t)] = t $$

where $\mathbb{E}[W(t)]$ is the expected value of the Wiener process at time $t$ and $\text{Var}[W(t)]$ is the variance of the Wiener process at time $t$.

The covariance of the Wiener process:
$$ \text{Cov}[W(t),W(s)] = \min(t,s) $$

where $\text{Cov}[W(t),W(s)]$ is the covariance between the Wiener process at times $t$ and $s$.

These are some of the most important formulas related to the Wiener process. I hope that helps!


\subsection{ Properties of random processes: stationarity (in the narrow and broad sense), ergodicity }
Random processes can have various properties that describe their behavior over time. Two important properties are stationarity and ergodicity.

Stationarity:

Stationarity, in the narrow sense, refers to the property of a random process where the mean and variance of the process are constant over time. That is, if $X(t)$ is a stationary random process, then for all $t$,

$$ \mathbb{E}[X(t)] = \mathbb{E}[X(t+\tau)] $$

$$ \text{Var}[X(t)] = \text{Var}[X(t+\tau)] $$

where $\mathbb{E}[X(t)]$ is the expected value of the random process at time $t$, $\text{Var}[X(t)]$ is the variance of the random process at time $t$, and $\tau$ is a constant time shift.

In the broad sense, stationarity refers to the property of a random process where the joint distribution of the process remains constant over time. That is, if $X(t_1),X(t_2),...,X(t_n)$ is a stationary random process in the broad sense, then for all $t_1,t_2,...,t_n$,

$$ f_{X(t_1),X(t_2),...,X(t_n)}(x_1,x_2,...,x_n) = f_{X(t_1+\tau),X(t_2+\tau),...,X(t_n+\tau)}(x_1,x_2,...,x_n) $$

where $f_{X(t_1),X(t_2),...,X(t_n)}(x_1,x_2,...,x_n)$ is the joint probability density function of the random process at times $t_1,t_2,...,t_n$.

Ergodicity:

Ergodicity refers to the property of a random process where the time average of the process converges to the ensemble average of the process. That is, if $X(t)$ is an ergodic random process, then for any function $g(X(t))$,

$$ \lim_{T \rightarrow \infty} \frac{1}{T} \int_0^T g(X(t))dt = \mathbb{E}[g(X(t))] $$

where $\mathbb{E}[g(X(t))]$ is the expected value of the function $g(X(t))$.

In other words, ergodicity means that the long-term behavior of the random process can be inferred from its statistical properties. This property is important in many applications, such as communication systems, signal processing, and control systems, where the behavior of the random process over long time periods is of interest.

In summary, stationarity and ergodicity are two important properties of random processes that describe their behavior over time. Stationarity refers to the property of a random process where the mean and variance (in the narrow sense) or joint distribution (in the broad sense) are constant over time. Ergodicity refers to the property of a random process where the time average of the process converges to the ensemble average of the process.


\subsection{ Ito’s lemma}
Ito's lemma is a mathematical result that provides a relationship between the rate of change of a stochastic process and its expected value. It is a fundamental result in stochastic calculus and is widely used in various areas, including finance, engineering, and physics.

Suppose that $X(t)$ is a stochastic process, and $f(X(t),t)$ is a differentiable function of $X(t)$ and $t$. Then, Ito's lemma states that:

$$ df(X(t),t) = \frac{\partial f}{\partial t}dt + \frac{\partial f}{\partial X}dX(t) + \frac{1}{2}\frac{\partial^2 f}{\partial X^2}(dX(t))^2 $$

where $\frac{\partial f}{\partial t}$, $\frac{\partial f}{\partial X}$, and $\frac{\partial^2 f}{\partial X^2}$ are the partial derivatives of $f(X(t),t)$ with respect to $t$, $X(t)$, and $X(t)$ respectively.

In this equation, the first term $\frac{\partial f}{\partial t}dt$ represents the rate of change of the function $f(X(t),t)$ with respect to $t$, while the second term $\frac{\partial f}{\partial X}dX(t)$ represents the rate of change of the function with respect to $X(t)$. The third term $\frac{1}{2}\frac{\partial^2 f}{\partial X^2}(dX(t))^2$ represents the correction term that takes into account the variation of the random process $X(t)$ over time.

Ito's lemma is widely used in various areas to model and analyze the behavior of stochastic processes. For example, in finance, it is used to model the evolution of stock prices, interest rates, and other financial quantities. In engineering and physics, it is used to model random signals and systems, such as communication signals, noise, and control systems.

In summary, Ito's lemma is a fundamental result in stochastic calculus that provides a relationship between the rate of change of a stochastic process and its expected value. The lemma is widely used in various areas, including finance, engineering, and physics, to model and analyze the behavior of stochastic processes.
\section{Machine Learning}
\subsection{ Supervised vs. unsupervised learning }
Supervised learning is a type of machine learning where the model is trained on a labeled dataset, where the desired output is already known. The model is trained to predict the output values based on the input features. The main goal of supervised learning is to learn a mapping function $f: X \rightarrow Y$ where X is the input feature space and Y is the output space.

Unsupervised learning, on the other hand, is a type of machine learning where the model is trained on an unlabeled dataset, and the goal is to find patterns and relationships in the data without any prior knowledge of the output. The main goal of unsupervised learning is to learn the structure of the data and to identify meaningful patterns and groups in it.

In summary, supervised learning requires labeled data and tries to predict an output given an input, while unsupervised learning works with unlabeled data and tries to find patterns and structure in the data.
\subsection{ Deep learning and reinforcement learning }
Deep learning is a subfield of machine learning that is inspired by the structure and function of the brain, known as artificial neural networks. It involves training artificial neural networks with multiple hidden layers on large amounts of data. Deep learning has achieved state-of-the-art performance on a variety of tasks, such as image and speech recognition, natural language processing, and others.

Reinforcement learning is a type of machine learning where the model learns to make decisions by taking actions in an environment to maximize a reward signal. The model interacts with the environment by taking actions, observing the resulting state, and receiving a reward. Based on this feedback, the model updates its policy to make better decisions in the future. Reinforcement learning has been successfully applied in fields such as gaming, robotics, and autonomous control.

In summary, deep learning is a subfield of machine learning that uses deep neural networks to learn from data, while reinforcement learning is a type of machine learning that focuses on decision-making by taking actions in an environment to maximize a reward signal.
\subsection{ Evaluating ML algorithm performance: generalization and overfitting }
Evaluating the performance of a machine learning algorithm is crucial in determining its effectiveness and reliability. Two important concepts in evaluating the performance of a machine learning algorithm are generalization and overfitting.

Generalization refers to the ability of a machine learning model to make accurate predictions on new, unseen data. The goal of a machine learning model is to generalize well, which means that it should be able to make accurate predictions on data that it has not seen before. A model that generalizes well has a low prediction error on new data.

Overfitting occurs when a machine learning model is too complex and fits the training data too well, capturing not only the true underlying relationships in the data but also the noise. As a result, the model is not able to generalize well and performs poorly on new, unseen data. Overfitting can be prevented by using simpler models, regularization techniques, and/or by using a larger training dataset.

In summary, generalization refers to the ability of a machine learning model to make accurate predictions on new, unseen data, while overfitting occurs when a model is too complex and fits the training data too well, leading to poor performance on new data. Evaluating the performance of a machine learning algorithm should take into account both generalization and overfitting to ensure that the model is effective and reliable.
\subsection{ Penalized regression }
Penalized regression is a type of regression analysis that adds a penalty term to the loss function to reduce the complexity of the model and prevent overfitting. The goal of penalized regression is to find a balance between fitting the training data well and having a simple and interpretable model.

The most common forms of penalized regression are Ridge Regression and Lasso Regression. In Ridge Regression, the penalty term is the L2-norm of the coefficients, which adds a penalty proportional to the magnitude of the coefficients. In Lasso Regression, the penalty term is the L1-norm of the coefficients, which has the effect of setting some coefficients to zero, effectively performing feature selection.

Penalized regression can be seen as a trade-off between goodness-of-fit and model simplicity. By adding a penalty term to the loss function, penalized regression tries to find a solution that is both a good fit to the training data and has a simple and interpretable model structure.

In summary, penalized regression is a type of regression analysis that adds a penalty term to the loss function to reduce the complexity of the model and prevent overfitting. The most common forms of penalized regression are Ridge Regression and Lasso Regression, which balance the goodness-of-fit and model simplicity.

Notes:

Lasso Regression and Ridge Regression are both forms of penalized regression that add a penalty term to the loss function to prevent overfitting.

The loss function in Lasso Regression is given by:

$$ J(\beta) = \frac{1}{N} \sum_{i=1}^{N} (y_i - \beta^T x_i)^2 + \lambda \sum_{j=1}^{p} |\beta_j| $$

where $y_i$ is the target value for the i-th sample, $\beta^T x_i$ is the predicted value for the i-th sample, $N$ is the number of samples, $p$ is the number of features, $\beta_j$ is the j-th coefficient, and $\lambda$ is a tuning parameter that controls the amount of regularization. The penalty term is the L1-norm of the coefficients, which has the effect of setting some coefficients to zero and performing feature selection.

The loss function in Ridge Regression is given by:

$$ J(\beta) = \frac{1}{N} \sum_{i=1}^{N} (y_i - \beta^T x_i)^2 + \lambda \sum_{j=1}^{p} \beta_j^2 $$

where the only difference with the Lasso Regression loss function is that the penalty term is the L2-norm of the coefficients, which adds a penalty proportional to the magnitude of the coefficients.

The L1-norm, also known as the Manhattan distance or the taxicab norm, is defined as:

$$ \left| x \right|1 = \sum{i=1}^{n} |x_i| $$

where $x$ is a vector and $x_i$ is the i-th element of the vector.

The L2-norm, also known as the Euclidean distance, is defined as:

$$ \left| x \right|2 = \sqrt{\sum{i=1}^{n} x_i^2} $$

where $x$ is a vector and $x_i$ is the i-th element of the vector.

In summary, Lasso Regression and Ridge Regression are forms of penalized regression that add a penalty term to the loss function to prevent overfitting. The penalty term in Lasso Regression is the L1-norm of the coefficients, while the penalty term in Ridge Regression is the L2-norm of the coefficients. The L1-norm is the sum of the absolute values of the elements of a vector, while the L2-norm is the square root of the sum of the squares of the elements of a vector.





\subsection{ Support vector machine }
Support Vector Machines (SVMs) is a type of supervised learning algorithm for classification and regression tasks. It is based on the idea of finding the maximum margin hyperplane that separates the data into different classes.

In the case of a two-class problem, the goal of SVM is to find the hyperplane with the largest margin between the two classes. The margin is defined as the distance between the hyperplane and the closest data points, also known as support vectors. These support vectors determine the position and orientation of the hyperplane, hence the name Support Vector Machines.

The optimization problem in SVM can be expressed as follows:

$$\min_{\beta, \beta_0} \frac{1}{2} \left| \beta \right|^2 + C \sum_{i=1}^{N} \xi_i$$

subject to

$$y_i (\beta^T x_i + \beta_0) \geq 1 - \xi_i$$

$$\xi_i \geq 0$$

where $\beta$ and $\beta_0$ are the parameters of the hyperplane, $C$ is a regularization parameter that controls the trade-off between maximizing the margin and minimizing the classification error, $y_i$ is the class label for the i-th sample, $x_i$ is the i-th feature vector, and $\xi_i$ is the slack variable that allows for misclassified points.

SVMs can also be used for regression by replacing the hinge loss with the squared error loss. In addition, SVMs can handle non-linearly separable data by transforming the data into a higher-dimensional space through the use of kernel functions.

In summary, Support Vector Machines (SVMs) is a type of supervised learning algorithm for classification and regression tasks. It finds the maximum margin hyperplane that separates the data into different classes, with the support vectors determining the position and orientation of the hyperplane. SVMs can handle non-linearly separable data through the use of kernel functions.


\subsection{ K-nearest neighbor (KNN) }
K-Nearest Neighbors (KNN) is a type of instance-based, or lazy, learning algorithm for classification and regression tasks. It is based on the idea of assigning a new data point to the class or value of its closest neighbors in the feature space.

The K in KNN represents the number of nearest neighbors used to make a prediction. For each new data point, the algorithm finds the K nearest points in the training data and aggregates their class labels or values to make a prediction. The most common aggregation method is to take a majority vote for classification or to take the average for regression.

The algorithm does not learn a model or make any assumptions about the distribution of the data. Instead, it stores the entire training data and makes predictions at run-time based on the proximity of new data points to the stored training data.

The prediction process in KNN can be formalized as follows:

Calculate the distance between the new data point and all training data points
Find the K nearest neighbors based on the distance metric
Assign the class label or value based on a majority vote or average of the K nearest neighbors
KNN has several important parameters, including the number of neighbors (K), the distance metric, and the weighting scheme. The choice of these parameters can affect the performance of the algorithm, so they should be carefully chosen through cross-validation or other performance evaluation techniques.

In summary, K-Nearest Neighbors (KNN) is a type of instance-based learning algorithm for classification and regression tasks. It makes predictions based on the proximity of new data points to the stored training data. The algorithm does not learn a model or make any assumptions about the data distribution, and the choice of parameters can affect its performance.


The prediction process in KNN can be expressed mathematically as follows:

Let $\mathbf{X} = \left{ \mathbf{x_1}, \mathbf{x_2}, \dots, \mathbf{x_N} \right}$ be the set of training data points with $\mathbf{x_i} \in \mathbb{R}^d$ and $y_i$ be the corresponding class labels or values. For a new data point $\mathbf{x}$, the KNN algorithm calculates the distances between $\mathbf{x}$ and all $\mathbf{x_i}$ using a distance metric $d(\mathbf{x}, \mathbf{x_i})$, and finds the K nearest neighbors $\mathbf{x_{i1}}, \mathbf{x_{i2}}, \dots, \mathbf{x_{iK}}$ based on the distances. The prediction for $\mathbf{x}$ is then given by:

For Classification:

$$\hat{y} = \text{majority} \left{ y_{i1}, y_{i2}, \dots, y_{iK} \right}$$

For Regression:

$$\hat{y} = \frac{1}{K} \sum_{j=1}^{K} y_{ij}$$

The complexity of the KNN algorithm is primarily determined by the distance calculations and the search for the nearest neighbors. The time complexity of finding the K nearest neighbors depends on the data structure used to store the training data, with the most common approach being a brute-force search that takes $O(Nd)$ time for each new data point, where N is the number of training data points and d is the dimensionality of the feature space.

In summary, the KNN algorithm predicts the class label or value for a new data point based on the majority vote or average of the K nearest neighbors. The complexity of the algorithm is determined by the distance calculations and the search for the nearest neighbors, with a brute-force search taking $O(Nd)$ time for each new data point.




\subsection{ Classification and regression tree (CART) }
Classification and Regression Tree (CART) is a type of decision tree algorithm used for both classification and regression tasks. It is a supervised machine learning algorithm that builds a tree-based model to make predictions.

The basic idea behind CART is to recursively split the feature space into smaller regions that contain similar class labels or values, and to make predictions based on the majority class or average value within each region. The algorithm starts with the entire feature space as the root node and splits it into two or more child nodes based on the best split that maximizes a criterion such as information gain or Gini impurity. The process is repeated for each child node until a stopping criterion is reached, such as a minimum number of samples in a leaf node or a maximum tree depth.

The final tree consists of a set of internal nodes that represent the splits and leaf nodes that represent the final regions. For each new data point, the algorithm traverses the tree from the root node to a leaf node and makes a prediction based on the majority class or average value in that region.

The prediction process in CART can be formalized as follows:

Start at the root node of the tree
Follow the split at the current node based on the value of the relevant feature for the new data point
Repeat step 2 until a leaf node is reached
Make a prediction based on the majority class or average value in the leaf node
CART has several important parameters, including the criterion for splitting, the stopping criterion, and the method for handling missing values or outliers. The choice of these parameters can affect the performance of the algorithm, so they should be carefully chosen through cross-validation or other performance evaluation techniques.

In summary, Classification and Regression Tree (CART) is a decision tree algorithm used for both classification and regression tasks. It builds a tree-based model by recursively splitting the feature space into smaller regions based on a criterion such as information gain or Gini impurity. The algorithm makes predictions based on the majority class or average value in the leaf nodes, and the choice of parameters can affect its performance.

Consider a binary classification problem with a two-dimensional feature space and two classes, $C_1$ and $C_2$. Let $\mathbf{X} = \left{ \mathbf{x_1}, \mathbf{x_2}, \dots, \mathbf{x_N} \right}$ be the set of training data points with $\mathbf{x_i} = [x_{i,1}, x_{i,2}]^T$ and $y_i \in \left{ C_1, C_2 \right}$ be the corresponding class labels. A simple example of a CART tree for this problem is shown below:

At the root node, the algorithm calculates the best split based on a criterion such as information gain or Gini impurity. For example, if the criterion is information gain, the split is chosen as the one that maximizes the following equation:

$$\text{Information Gain} = \text{Entropy}(T) - \sum_{j=1}^{m} \frac{N_j}{N} \text{Entropy}(T_j)$$

where $T$ is the root node, $T_j$ is the $j^{th}$ child node, $m$ is the number of child nodes, $N$ is the number of samples in $T$, and $N_j$ is the number of samples in $T_j$. The entropy is calculated as:

$$\text{Entropy}(T) = - \sum_{k=1}^{c} p_k \log_2 p_k$$

where $c$ is the number of classes, and $p_k$ is the proportion of samples in $T$ that belong to class $k$.

For the example CART tree above, the splits are based on the values of the two features, $x_{1}$ and $x_{2}$. For example, the split at the root node could be based on the value of $x_{1}$, with a threshold of 0.5, such that samples with $x_{1} < 0.5$ are sent to the left child node and samples with $x_{1} \ge 0.5$ are sent to the right child node. The splits at the child nodes could be based on the value of $x_{2}$, and so on.

Once the tree is built, the prediction for a new data point $\mathbf{x}$ is made by traversing the tree from the root node to a leaf node based on the values of the relevant features. For example, for the data point $\mathbf{x} = [0.2, 0.8]^T$, the algorithm would follow the path [Root] $\to$ [Split 1] $\to$ [Leaf 1] and make a prediction based on the majority class or average value in that region.

In summary, the CART algorithm builds a tree-based model by recursively splitting the feature space into smaller regions based on a criterion such as information gain or Gini impurity. The splits are based on the values of the relevant features, and the predictions are made based on the majority class or average value in the
corresponding region defined by the tree. The CART algorithm can handle both continuous and categorical variables, and is widely used for both classification and regression problems. The complexity of the CART algorithm is typically linear with respect to the number of samples and the number of features, making it a relatively efficient algorithm for small to medium-sized datasets. However, for very large datasets, more sophisticated algorithms such as random forests or gradient boosting may be preferred due to their ability to handle large amounts of data more efficiently.






\subsection{ Clustering: K-mean, hierarchical, agglomerative, divisive }

Clustering is a type of unsupervised learning that groups similar data points together into clusters. There are several different algorithms for clustering, including K-means, hierarchical, agglomerative, and divisive.

K-means is one of the most widely used clustering algorithms and is based on the idea of partitioning the data into $K$ clusters, where $K$ is a user-defined parameter. The algorithm starts by randomly initializing $K$ cluster centroids and then assigns each data point to the closest centroid. The centroids are then recomputed as the mean of all data points assigned to that cluster, and the data points are re-assigned to the closest centroid. These two steps are repeated until the cluster assignments no longer change or a maximum number of iterations is reached.

Hierarchical clustering is another popular clustering method that creates a hierarchy of clusters, starting with each data point as a single-point cluster and then merging or splitting clusters based on some criterion, such as distance between clusters. There are two main types of hierarchical clustering: agglomerative and divisive. In agglomerative clustering, the algorithm starts with individual data points and then merges clusters into larger clusters until a stopping criterion is reached. In divisive clustering, the algorithm starts with all data points in a single cluster and then splits the cluster into smaller clusters until a stopping criterion is reached.

The choice of clustering algorithm depends on the nature of the data and the desired properties of the clusters. For example, K-means is a fast and efficient algorithm for large datasets, but is sensitive to initial conditions and may produce suboptimal solutions. Hierarchical clustering is more flexible and can handle non-linear relationships, but can be slower and more computationally intensive.

The objective function for K-means clustering is to minimize the sum of squared distances between each data point and its assigned cluster centroid:

$$ J = \sum_{i=1}^K \sum_{x \in C_i} ||x - \mu_i||^2 $$

where $K$ is the number of clusters, $C_i$ is the set of data points assigned to cluster $i$, $\mu_i$ is the centroid of cluster $i$, and $||x - \mu_i||^2$ is the squared Euclidean distance between data point $x$ and centroid $\mu_i$.

In hierarchical clustering, the objective is to create a tree structure that summarizes the relationships between data points. One popular method for defining the tree structure is based on single linkage, where the distance between two clusters is defined as the minimum distance between any two data points in the two clusters. Another popular method is based on complete linkage, where the distance between two clusters is defined as the maximum distance between any two data points in the two clusters.

$$ d(C_i, C_j) = \min_{x \in C_i, y \in C_j} ||x - y|| $$

or

$$ d(C_i, C_j) = \max_{x \in C_i, y \in C_j} ||x - y|| $$

where $d(C_i, C_j)$ is the distance between clusters $C_i$ and $C_j$, and $||x - y||$ is the Euclidean distance between data points $x$ and $y$.

In agglomerative hierarchical clustering, the objective is to iteratively merge the two closest clusters until all data points are in a single cluster or a stopping criterion is reached. The merging process can be formalized as finding a pair of clusters $(C_i, C_j)$ such that the distance $d(C_i, C_j)$ is minimized:

$$ (C_i, C_j) = \arg \min_{C_p, C_q} d(C_p, C_q) $$

where $d(C_p, C_q)$ is the distance between clusters $C_p$ and $C_q$, as defined by either single linkage or complete linkage.

In divisive hierarchical clustering, the objective is to iteratively split the largest cluster into two smaller clusters until each data point is in its own cluster or a stopping criterion is reached. The splitting process can be formalized as finding a cluster $C_i$ and a threshold value $t$ such that the sum of squared distances between each data point and its assigned centroid is minimized:

$$ (C_i, t) = \arg \min_{C_p, t} \sum_{x \in C_p} ||x - \mu_{C_p}||^2 $$

where $\mu_{C_p}$ is the centroid of cluster $C_p$. The data points in cluster $C_p$ are then split into two new clusters based on their distances to $\mu_{C_p}$ relative to the threshold value $t$.
\subsection{ Neural networks  }
\subsubsection{Bla-bla-bla } 
Neural networks are a type of machine learning algorithm inspired by the structure and function of the human brain. They consist of interconnected nodes or artificial neurons, which are organized into layers. The input layer receives the raw data, and the output layer produces the predictions. The intermediate layers, known as hidden layers, perform transformations on the input data to generate intermediate representations that can capture complex relationships between the inputs and outputs.

Each artificial neuron in a neural network takes in a set of inputs, applies a weighted sum to the inputs, and passes the result through an activation function. The activation function determines the output of the neuron and can be used to introduce non-linearity into the network.

There are several different types of neural networks, including feedforward neural networks, recurrent neural networks, and convolutional neural networks. They can be used for various tasks, including classification, regression, and image recognition.

One common technique for training a neural network is backpropagation, which involves adjusting the weights of the network to minimize the difference between the predicted outputs and the true outputs. This is done by computing the gradient of the loss function with respect to the weights and updating the weights in the direction of the negative gradient.
\subsubsection{backpropagation}
Backpropagation is an algorithm for training neural networks that uses gradient descent to minimize a loss function. The goal of backpropagation is to find the optimal weights for the network such that the difference between the predicted outputs and the true outputs is minimized.

Backpropagation works by computing the gradient of the loss function with respect to the weights of the network, and then updating the weights in the direction of the negative gradient. The gradient computation is performed using the chain rule of calculus, which allows the gradient of the loss function with respect to the weights to be computed by recursively applying the chain rule to each layer of the network.

The basic steps of backpropagation are as follows:

Feedforward: The input data is fed into the network, and the activations of each layer are computed.

Loss computation: The loss function is computed based on the difference between the predicted outputs and the true outputs.

Backpropagation: The gradient of the loss function with respect to the weights of the network is computed using the chain rule.

Weight update: The weights are updated in the direction of the negative gradient, using a learning rate to control the size of the update.

Here is a mathematical example to illustrate the backpropagation algorithm. Let's consider a neural network with a single hidden layer and a single output neuron. The input to the network is a vector $x \in \mathbb{R}^n$, the weights connecting the input layer to the hidden layer are represented by a matrix $W_1 \in \mathbb{R}^{m \times n}$, the weights connecting the hidden layer to the output layer are represented by a vector $W_2 \in \mathbb{R}^m$, and the activation function is the sigmoid function $\sigma(z) = \frac{1}{1 + e^{-z}}$.

The activations of the hidden layer and the output layer are computed as follows:

$$ a_1 = \sigma(W_1x) $$
$$ a_2 = \sigma(W_2a_1) $$

The loss function is given by a mean squared error (MSE) between the predicted output $a_2$ and the true output $y$:

$$ L = \frac{1}{2} ||a_2 - y||^2 $$

The gradient of the loss function with respect to the weights can be computed using the chain rule:

$$ \frac{\partial L}{\partial W_2} = (a_2 - y) \odot a_1 \odot (1 - a_1) $$
$$ \frac{\partial L}{\partial W_1} = (a_2 - y) \odot W_2 \odot (1 - a_1) \odot a_1 \odot x $$

where $\odot$ denotes element-wise multiplication.

Finally, the weights are updated in the direction of the negative gradient:

$$ W_2 = W_2 - \alpha \frac{\partial L}{\partial W_2} $$
$$ W_1 = W_1 - \alpha \frac{\partial L}{\partial W_1} $$

where $\alpha$ is the learning rate. This process is repeated for multiple iterations until the loss function converges to a minimum or a stopping criterion is reached.
\subsubsection{Activation functions }
Activation functions are mathematical operations applied to the inputs of each neuron in a neural network to produce its output. They introduce non-linearities into the network, allowing it to learn complex relationships between the inputs and outputs.

Here are some of the most widely used activation functions, along with their formulas:

Sigmoid:
$$ \sigma(z) = \frac{1}{1 + e^{-z}} $$

The sigmoid activation function maps any input to the range [0, 1]. It is widely used in binary classification problems, where the output represents the probability of the positive class. The main disadvantage of the sigmoid function is that it can lead to vanishing gradients, which can make it difficult to train deep networks.

Hyperbolic tangent (tanh):
$$ \text{tanh}(z) = \frac{e^z - e^{-z}}{e^z + e^{-z}} $$

The tanh activation function maps any input to the range [-1, 1]. Like the sigmoid function, it is widely used in binary and multi-class classification problems. However, compared to the sigmoid function, the tanh function is less prone to vanishing gradients and has a faster convergence rate.

Rectified linear unit (ReLU):
$$ \text{ReLU}(z) = \max(0, z) $$

The ReLU activation function maps any input to either 0 or the input value itself. It is widely used in feedforward neural networks, where it has been shown to significantly improve the training speed and accuracy compared to other activation functions. The main disadvantage of the ReLU function is that it can result in dead neurons, which can lead to reduced network performance.

Leaky ReLU:
$$ \text{Leaky ReLU}(z) = \max(0.01z, z) $$

The Leaky ReLU activation function is a variant of the ReLU function that solves the problem of dead neurons. It maps any input to either a small negative value or the input value itself, ensuring that the gradient of the function is always positive.

These are some of the most widely used activation functions in neural networks. The choice of activation function depends on the specific problem being solved and the characteristics of the data. Each activation function has its own advantages and disadvantages, and the best choice is often determined through experimentation and trial-and-error.
\subsubsection{feedforward neural network }
A feedforward neural network is a type of artificial neural network in which the information flows only in one direction, from the input layer to the output layer, without looping back. The information is processed through a series of hidden layers, each of which applies a set of weights and biases to the inputs and passes the results through an activation function to produce the outputs.

The structure of a feedforward neural network can be represented as an input layer, one or more hidden layers, and an output layer. The input layer takes in the input data and passes it to the first hidden layer. Each hidden layer receives the outputs from the previous layer and processes them further to produce its own outputs, which are passed to the next hidden layer. The final hidden layer produces the outputs that are passed to the output layer, which produces the final prediction.

The goal of training a feedforward neural network is to learn the weights and biases that produce the best predictions for the given data. This is typically done using a supervised learning approach, in which the network is trained using a labeled dataset and an optimization algorithm such as stochastic gradient descent. During training, the optimization algorithm adjusts the weights and biases to minimize the difference between the network's predictions and the actual values in the training data.

Feedforward neural networks are widely used for various tasks, including image classification, speech recognition, natural language processing, and many others. They are particularly well-suited for problems in which the relationships between the inputs and outputs are complex and non-linear. With the ability to learn these relationships, feedforward neural networks can produce highly accurate predictions for unseen data.
\subsubsection{RNN}
Recurrent neural networks (RNNs) are a type of artificial neural network that are designed to handle sequential data, such as time series data, speech signals, and text. Unlike feedforward neural networks, which process information in a one-directional flow, RNNs have the ability to process inputs in a sequential manner, and to use information from previous time steps to inform the processing of current time steps. This allows RNNs to capture complex dependencies between inputs and outputs, making them well-suited for tasks such as language modeling, speech recognition, and music generation.

An RNN is composed of a set of recurrent units, each of which takes as input the output of the previous time step and the current input, and produces an output for the current time step. The recurrent units are connected in a cyclic loop, allowing the network to maintain information from previous time steps and to use it to inform its processing of the current time step.

The structure of an RNN can be represented as an input layer, one or more hidden layers, and an output layer. The input layer takes in the input data for a given time step, and the hidden layers process the data using a set of weights and biases and an activation function, similar to a feedforward neural network. The outputs from the hidden layers are passed to the output layer, which produces the final prediction for the current time step.

The goal of training an RNN is to learn the weights and biases that produce the best predictions for a given dataset. This is typically done using a supervised learning approach, in which the network is trained using a labeled dataset and an optimization algorithm such as stochastic gradient descent. During training, the optimization algorithm adjusts the weights and biases to minimize the difference between the network's predictions and the actual values in the training data.

RNNs are widely used for various tasks, including language modeling, speech recognition, and machine translation. They are particularly well-suited for problems in which the relationships between inputs and outputs are complex and depend on previous inputs and outputs. With the ability to capture these relationships, RNNs can produce highly accurate predictions for sequential data.
\subsubsection{CNN}
Convolutional neural networks (ConvNets or CNNs) are a type of artificial neural network designed for image classification and processing. Unlike feedforward and recurrent neural networks, which process information in a flat and sequential manner, respectively, ConvNets process information in a hierarchical and spatially organized manner. This allows ConvNets to effectively capture the spatial structure and relationships between pixels in an image, making them well-suited for image classification and other computer vision tasks.

A ConvNet consists of multiple layers, including an input layer, one or more convolutional layers, one or more pooling layers, and one or more fully-connected layers. The input layer takes in the image data, and the convolutional layers use filters to scan the image and detect features such as edges, corners, and textures. The pooling layers then reduce the spatial dimensions of the feature maps, which helps to reduce the amount of computation required and to make the network more robust to small translations in the input. The fully-connected layers then use the feature maps to produce the final classification or prediction.

The filters in the convolutional layers are learned during the training process, which is typically done using a supervised learning approach, in which the network is trained using a labeled dataset and an optimization algorithm such as stochastic gradient descent. During training, the optimization algorithm adjusts the filters to minimize the difference between the network's predictions and the actual labels in the training data.

ConvNets are widely used for various tasks in computer vision, such as image classification, object detection, and segmentation. They have been very successful in achieving high accuracy on benchmark datasets and are widely used in industry and academia for various computer vision applications. With their ability to effectively capture the spatial structure of images, ConvNets have revolutionized the field of computer vision and have opened up new avenues for research and applications.
\subsubsection{Diferencec in structure}
The structure of Convolutional Neural Networks (ConvNets), Recurrent Neural Networks (RNNs), and Feedforward Neural Networks (FFNNs) are each designed to handle different types of input data and tasks.

ConvNets are designed to handle image and other grid-structured data. They consist of an input layer, one or more convolutional layers, one or more pooling layers, and one or more fully-connected layers. The convolutional layers use filters to scan the image and detect features such as edges, corners, and textures. The pooling layers reduce the spatial dimensions of the feature maps and make the network more robust to small translations in the input. The fully-connected layers use the feature maps to produce the final classification or prediction.

RNNs are designed to handle sequential data, such as time series, text, and speech. They consist of an input layer, one or more recurrent layers, and one or more fully-connected layers. The recurrent layers use the hidden state of the network from the previous time step to process the current input and produce the hidden state for the current time step. This allows the network to capture dependencies between the inputs across time. The fully-connected layers use the hidden state to produce the final prediction.

FFNNs are designed to handle any type of input data and are the simplest type of neural network. They consist of an input layer, one or more hidden layers, and an output layer. The hidden layers use weighted connections to transform the input into a higher-level representation that is used to produce the final prediction.

Here is an example of the mathematical formulation of a simple feedforward neural network:

\begin{equation}
z_1 = W_1 x + b_1 \
a_1 = \sigma(z_1) \
z_2 = W_2 a_1 + b_2 \
a_2 = \sigma(z_2) \
\cdots \
z_n = W_n a_{n-1} + b_n \
\hat{y} = \sigma(z_n)
\end{equation}

where $x$ is the input, $W_i$ and $b_i$ are the weights and biases of the $i$-th layer, $\sigma$ is the activation function, and $\hat{y}$ is the prediction.

For a more in-depth treatment of ConvNets, RNNs, and FFNNs, including their architectures and mathematical formulations, I recommend reading the original research papers or taking a course in deep learning. Some good resources include the book "Deep Learning" by Ian Goodfellow, Yoshua Bengio, and Aaron Courville, and the online course "Deep Learning Specialization" by Andrew Ng on Coursera.

\section{Python}
\subsection{ Python as an object-oriented programming language. Functions isinstance( ), type( ) }
Python is an object-oriented programming language, which means that in Python, everything is an object. This includes functions, integers, strings, lists, dictionaries, etc. An object has properties and methods, which are like variables and functions, respectively, associated with the object.

The isinstance() function is used to determine if an object is an instance of a certain class or a subclass of that class. This is useful in testing the type of an object in a program. Here's an example:

\begin{lstlisting}[language=Python, frame=single]
def add_numbers(a, b):
    if isinstance(a, (int, float)) and isinstance(b, (int, float)):
        return a + b
    return "Inputs must be numbers"

print(add_numbers(2, 3)) # 5
print(add_numbers(2, "3")) # "Inputs must be numbers"
\end{lstlisting}
The type() function is used to determine the type of an object. This is useful in determining the class of an object in a program. Here's an example:

\begin{lstlisting}[language=Python, frame=single]
def add_numbers(a, b):
    if type(a) == int and type(b) == int:
        return a + b
    return "Inputs must be integers"


print(add_numbers(2, 3)) # 5
print(add_numbers(2, "3")) # "Inputs 
must be integers"
\end{lstlisting}
These functions are useful when working with objects in Python, as they allow you to determine the type of an object and ensure that the inputs to your functions are the correct types.

\subsection{ Python basic data types: numbers, strings, booleans, tuples, lists, dictionaries, sets. Basic methods and properties of basic data structures: iterable, ordered, mutable, hashable, etc. }

section{Python basic data types}

In Python, there are several basic data types that are used to store data and information. These include:

\begin{itemize}
\item Numbers: This includes integers and floating-point numbers, which are used to store numerical values.
\item Strings: This includes sequences of characters, which are used to store text data. Strings are ordered and iterable, and support various methods for manipulating and transforming the data.
\item Booleans: This is a data type that can only take two values: \texttt{True} or \texttt{False}. Booleans are commonly used in conditional statements to make decisions in a program.
\item Tuples: This is an immutable ordered collection of elements. Once a tuple is created, its elements cannot be changed.
\item Lists: This is a mutable ordered collection of elements. Lists are similar to tuples, but elements can be added, removed, or modified after the list is created.
\item Dictionaries: This is a collection of key-value pairs, where each key is unique and maps to a value. Dictionaries are unordered, but are efficient for fast lookups of values based on keys.
\item Sets: This is an unordered collection of unique elements. Sets are useful for operations like union, intersection, and difference.
\end{itemize}

Each of these data types has its own properties and methods. For example:

\begin{itemize}
\item The \texttt{len()} function returns the number of elements in a collection.
\item The \texttt{sorted()} function returns a sorted version of a collection.
\item The \texttt{in} operator is used to check if an element is in a collection.
\item The \texttt{+=} operator is used to concatenate or append elements to a collection.
\end{itemize}

Additionally, some data types are iterable, meaning that you can loop over their elements one by one, ordered, meaning that their elements have a defined order, mutable, meaning that their elements can be changed, and hashable, meaning that they can be used as keys in dictionaries.

Here's an example of how you can use these data types and their properties in a program:

\begin{lstlisting}[language=Python, frame=single]
# Create a list
numbers = [1, 2, 3, 4, 5]

# Use the len() function
print(len(numbers)) # 5

# Use the sorted() function
print(sorted(numbers)) # [1, 2, 3, 4, 5]

# Use the in operator
print(3 in numbers) # True
print(6 in numbers) # False

# Use the += operator
numbers += [6, 7, 8]
print(numbers) # [1, 2, 3, 4, 5, 6, 7, 8]

# Create a tuple
person = ("John", "Doe", 32)

# Use the index operator
print(person[0]) # "John"

# Create a dictionary
person = {"first_name": "John", "last_name": "Doe", "age":40}
\end{lstlisting}

It's important to note that the basic data types in Python are not limited to the examples shown here. There are many more properties and methods associated with each data type, and you can even define your own custom data types using classes.

In conclusion, the basic data types in Python form the foundation for storing and manipulating data in your programs. Understanding these data types and their properties and methods is essential for building efficient and effective programs in Python.
\subsection{ Loops: for loop and while loop. Why using loops in Python might not be the best idea? }
In Python, there are two types of loops that are commonly used to execute a set of statements multiple times: the for loop and the while loop.

The for loop is used to iterate over a sequence (such as a list, tuple, or string) and execute a block of code for each item in the sequence. For example:

\begin{lstlisting}[language=Python, frame=single]

Using the for loop
fruits = ["apple", "banana", "cherry"]
for fruit in fruits:
print(fruit)
\end{lstlisting}

The while loop is used to execute a block of code repeatedly as long as a given condition is true. For example:

\begin{lstlisting}[language=Python, frame=single]

Using the while loop
count = 1
while count <= 5:
print(count)
count += 1
\end{lstlisting}

While loops and for loops can be useful for solving certain problems, it's important to note that using loops can lead to inefficiencies in your code, especially if the loop runs many times. This can result in long execution times and high memory usage, which can make your program slow and difficult to scale.

Therefore, it's often a good idea to avoid using loops when possible and instead use built-in functions and libraries that can perform the same operations more efficiently. Additionally, using list comprehensions and generator expressions can help you write cleaner and more efficient code, as they allow you to perform operations on collections of data without having to write explicit loops.

For example, instead of using a loop to compute the squares of a list of numbers, you can use a list comprehension:

\begin{lstlisting}[language=Python, frame=single]

Using a list comprehension
numbers = [1, 2, 3, 4, 5]
squared = [num**2 for num in numbers]
print(squared) # [1, 4, 9, 16, 25]
\end{lstlisting}
\subsection{ List, set, dictionary comprehensions. Iterators and generators }
List comprehensions, set comprehensions, and dictionary comprehensions are a compact and efficient way to create new collections based on existing collections in Python. They provide a concise syntax for transforming data and can be used as a substitute for traditional loops.

A list comprehension consists of an expression followed by a for clause, and optionally one or more if clauses. The expression defines the transformation to be performed on each element in the collection, and the for clause specifies the collection to be transformed. For example:

\begin{lstlisting}[language=Python, frame=single]

Using a list comprehension
numbers = [1, 2, 3, 4, 5]
squared = [num**2 for num in numbers]
print(squared) # [1, 4, 9, 16, 25]
\end{lstlisting}

A set comprehension is similar to a list comprehension, but instead of creating a list, it creates a set. For example:

\begin{lstlisting}[language=Python, frame=single]

Using a set comprehension
numbers = [1, 2, 3, 4, 5]
squared_set = {num**2 for num in numbers}
print(squared_set) # {1, 4, 9, 16, 25}
\end{lstlisting}

A dictionary comprehension consists of an expression followed by a for clause and a key: value clause. The expression defines the transformation to be performed on each key-value pair in the collection, and the for clause specifies the collection to be transformed. For example:

\begin{lstlisting}[language=Python, frame=single]

Using a dictionary comprehension
numbers = [1, 2, 3, 4, 5]
squared_dict = {num: num**2 for num in numbers}
print(squared_dict) # {1: 1, 2: 4, 3: 9, 4: 16, 5: 25}
\end{lstlisting}

Iterators and generators are a powerful way to represent sequences of values that can be iterated over one at a time. An iterator is an object that implements the iter method, which returns an object that implements the next method. A generator is a special type of iterator that uses a function to generate its values on the fly, instead of storing them in memory.

For example, the following generator function generates the squares of the numbers from 0 to 4:

\begin{lstlisting}[language=Python, frame=single]

Using a generator function
def squares(n):
for i in range(n):
yield i**2

gen = squares(5)
for num in gen:
print(num) # 0, 1, 4, 9, 16
\end{lstlisting}

Iterators and generators provide a flexible and efficient way to represent sequences of values, and are often used to process large amounts of data, as they allow you to generate values one at a time, without having to store all of the values in memory at once.

The yield keyword is used in Python to define a generator function. A generator function is a special type of function that returns an iterator, which can be used to generate a sequence of values.

The yield keyword is used to return a value from the generator function and pause its execution. The next time the generator function is called, it will resume from where it left off and continue to execute until it encounters another yield statement or reaches the end of the function.

For example, consider the following generator function:

\begin{lstlisting}[language=Python, frame=single]
def squares(n):
for i in range(n):
yield i**2

gen = squares(5)
for num in gen:
print(num) # 0, 1, 4, 9, 16
\end{lstlisting}

In this example, the generator function squares takes an argument n and returns a generator that generates the squares of the numbers from 0 to n-1. When the generator function is called, it starts executing and returns the first value generated by the yield statement. The function's execution is then paused and the control is returned to the caller. The next time the generator is called, it resumes its execution from where it left off, and continues to execute until it encounters another yield statement or reaches the end of the function. This process continues until all values have been generated.

The yield keyword is an important concept in Python and provides a flexible and efficient way to generate sequences of values, especially when dealing with large amounts of data.




\subsection{ Functions in Python. Function as an object. Lambda functions }
In Python, a function is a block of code that can be executed repeatedly. Functions are defined using the def keyword, followed by the function name and a set of parentheses that contain the function's arguments. The code inside the function is indented, and the function is terminated with the return keyword.

For example, consider the following function that takes two arguments a and b and returns their sum:

\begin{lstlisting}[language=Python, frame=single]
def add(a, b):
return a + b

result = add(2, 3)
print(result) # 5
\end{lstlisting}

Functions in Python are first-class objects, which means they can be assigned to variables, passed as arguments to other functions, and returned as values from functions.

Lambda functions, also known as anonymous functions, are a concise way to create small, one-line functions in Python. They are defined using the lambda keyword, followed by a set of parentheses that contain the function's arguments, a colon, and the function's return value.

For example, the following lambda function takes two arguments a and b and returns their sum:

\begin{lstlisting}[language=Python, frame=single]
add = lambda a, b: a + b

result = add(2, 3)
print(result) # 5
\end{lstlisting}

Lambda functions are often used as arguments to higher-order functions, such as map, filter, and reduce. They provide a convenient and concise way to write simple functions that can be passed as arguments to other functions.

\subsection{ Basic principles of OOP: encapsulation, polymorphism, inheritance. Magical methods. }
Object-Oriented Programming (OOP) is a programming paradigm that uses objects, which are instances of classes, to represent real-world concepts and to model the data and behavior of the system.

The basic principles of OOP are:

Encapsulation: This refers to the idea of wrapping data and behavior within a single unit or object. In OOP, an object's data is hidden from the outside world, and can only be accessed through the object's methods. This helps to ensure the consistency and integrity of the object's data.

Polymorphism: This refers to the ability of objects of different classes to respond to the same method call in different ways. In Python, polymorphism is achieved through method overloading and method overriding.

Inheritance: This refers to the ability of a new class to inherit the properties and behaviors of an existing class. The new class is called a subclass, and the existing class is called a superclass. Inheritance allows for the creation of new classes that are specialized versions of existing classes.

In Python, magic methods are special methods that have double underscores at the beginning and end of their names. Magic methods provide a way to customize the behavior of objects, such as the way they are compared, displayed, and manipulated.

For example, consider the following class definition that implements a simple point object:

\begin{lstlisting}[language=Python, frame=single]
class Point:
def init(self, x, y):
self.x = x
self.y = y

python
Copy code
def __str__(self):
    return f"({self.x}, {self.y})"
p = Point(2, 3)
print(p) # (2, 3)
\end{lstlisting}

In this example, the $__init__$ magic method is used to initialize the object's attributes, and the $__str__$ magic method is used to return a human-readable string representation of the object.

These basic principles and concepts of OOP provide a powerful and flexible way to organize and structure code in Python, making it easier to create and maintain complex systems.

self is a reference to the current instance of a class, and is used to access the attributes and methods of an object. In Python, self is the first argument that must be passed to any method defined within a class.

For example, consider the following class definition that implements a simple point object:

\begin{lstlisting}[language=Python, frame=single]
class Point:
def init(self, x, y):
self.x = x
self.y = y

python
Copy code
def __str__(self):
    return f"({self.x}, {self.y})"
p = Point(2, 3)
print(p) # (2, 3)
\end{lstlisting}

In this example, the self argument is used to access the object's x and y attributes, and to return a string representation of the object. By convention, self is usually the first argument in any method defined within a class, and it is mandatory to include it when calling the method.

The use of self allows objects of a class to maintain their own state and behavior, and provides a way to access and manipulate the object's data from within the object's methods. This is one of the fundamental concepts of Object-Oriented Programming (OOP), and it helps to ensure the separation of concerns and the encapsulation of data within objects.

\subsection{ O(n) notation. Search / insert / delete arrays in Python. Hash tables.   }
The O(n) notation is a way to describe the growth rate of an algorithm's running time with respect to the size of its input. The n in O(n) refers to the size of the input, and the O symbol is used to indicate that the running time grows linearly with n.

For example, consider an algorithm that processes each item in a list of n items, one by one. The running time of this algorithm would be proportional to n, so it would be described as having a running time of O(n).

In Python, lists and arrays can be used to implement basic data structures for searching, inserting, and deleting elements. However, the running times of these operations can be slow for large inputs, especially when the list needs to be sorted or reallocated in memory to accommodate new elements.

A hash table is a more efficient data structure for implementing the same operations, because it uses a hash function to map elements to indices in an array, allowing constant-time access to elements. When implemented correctly, the running times of hash table operations are usually O(1), meaning that they take a constant amount of time, regardless of the size of the input.

However, hash tables also have some disadvantages, such as the need to handle collisions (when two elements are mapped to the same index), and the need to choose a good hash function to ensure an even distribution of elements in the array.

In conclusion, choosing the right data structure is important for optimizing the performance of your code, and understanding the trade-offs between different data structures, such as arrays and hash tables, is an important aspect of algorithmic design and software engineering.

\section{Macroeconomics  }
\subsection{ Aggregate demand and supply }
Aggregate demand and aggregate supply are two concepts used in macroeconomics to understand the overall demand and supply in an economy.

Aggregate demand is the total demand for all goods and services in an economy. It is represented by the aggregate demand curve, which shows the relationship between the price level and the total amount of goods and services demanded in an economy. The aggregate demand curve slopes downward, which means that as the price level increases, the total demand for goods and services decreases. The aggregate demand curve can shift due to changes in consumer confidence, government spending, and changes in the money supply.

Aggregate supply is the total amount of goods and services that firms in an economy are willing and able to produce and sell. It is represented by the aggregate supply curve, which shows the relationship between the price level and the total amount of goods and services supplied in an economy. The aggregate supply curve slopes upward, which means that as the price level increases, the total supply of goods and services increases. The aggregate supply curve can shift due to changes in the cost of production, changes in technology, and changes in the availability of resources.

The intersection of the aggregate demand and aggregate supply curves represents the equilibrium price level and the corresponding quantity of goods and services demanded and supplied in an economy. Any changes in aggregate demand or aggregate supply can shift the curves and change the equilibrium price level and quantity.

Aggregate demand and aggregate supply play a crucial role in understanding the macroeconomic environment, as they can help explain the behavior of prices, output, and employment in an economy. They are used to analyze short-run and long-run macroeconomic phenomena, including inflation, recession, and economic growth.




\subsection{ Business cycles: expansion, peak, recession and trough }
Business cycles, also known as economic cycles or trade cycles, refer to the fluctuations in economic activity that occur over time. These cycles are characterized by four phases: expansion, peak, recession, and trough.

Expansion: The expansion phase is characterized by increased economic activity and growth, as well as rising employment, rising incomes, and increasing demand for goods and services. During this phase, the economy is considered to be in a state of growth, with a positive outlook for the future.

Peak: The peak is the highest point of the expansion phase, characterized by the highest levels of economic activity, employment, and demand. At this stage, the economy is considered to be operating at or near its maximum potential.

Recession: A recession is a period of economic contraction, characterized by a decline in economic activity, employment, and demand. During a recession, businesses are less confident, and consumers spend less, leading to a slowdown in the economy.

Trough: The trough is the lowest point of the recession, characterized by the lowest levels of economic activity, employment, and demand. After the trough, the economy begins to recover and enters the expansion phase once again.

The business cycle is a recurring phenomenon that has occurred throughout the history of capitalism. It is an important concept in macroeconomics because it helps to explain the fluctuations in economic activity that occur over time, and it provides insight into the factors that drive these fluctuations. By understanding the business cycle, economists can develop policies and strategies to help mitigate the negative effects of recessions and promote economic growth.




\subsection{ Examples of economic indicators: leading, coincident, lagging }
Economic indicators are statistical data that provide information about various aspects of the economy, such as economic growth, inflation, employment, and trade. There are three main types of economic indicators: leading, coincident, and lagging indicators.

Leading Indicators: Leading indicators are economic data that tend to change before the economy as a whole begins to change. These indicators are often used to predict future economic activity. Examples of leading indicators include stock prices, building permits, and the Purchasing Managers' Index (PMI).

Coincident Indicators: Coincident indicators are economic data that tend to change at the same time as the economy as a whole. These indicators provide a snapshot of the current state of the economy. Examples of coincident indicators include gross domestic product (GDP), industrial production, and employment.

Lagging Indicators: Lagging indicators are economic data that tend to change after the economy as a whole begins to change. These indicators provide information about the past state of the economy. Examples of lagging indicators include the Consumer Price Index (CPI), average weekly earnings, and the unemployment rate.

Economic indicators are used by policymakers, economists, and investors to analyze the state of the economy and make informed decisions. For example, a leading indicator like the PMI can signal an upcoming slowdown in the economy, allowing policymakers to take steps to mitigate its effects. Similarly, a lagging indicator like the unemployment rate can provide information about the success of previous policy decisions. The combination of different types of economic indicators provides a comprehensive picture of the economy and its current state, making it easier to understand and respond to economic conditions.
\subsection{ Types of unemployment: frictional, structural, cyclical }
Unemployment is a major economic challenge that affects individuals, communities, and entire countries. There are three main types of unemployment: frictional, structural, and cyclical.

Frictional Unemployment: Frictional unemployment occurs when workers are in between jobs and actively searching for new employment. This type of unemployment is considered normal and is a natural part of the labor market, as workers transition from one job to another.

Structural Unemployment: Structural unemployment occurs when workers are unable to find employment due to a mismatch between their skills and the skills required for available jobs. This type of unemployment can result from technological advancements, globalization, and changes in the demand for certain goods and services.

Cyclical Unemployment: Cyclical unemployment occurs during economic downturns and recessions, when there is not enough demand for goods and services. As a result, businesses reduce production, lay off workers, and cut back on hiring. This type of unemployment is related to the overall health of the economy and tends to rise and fall with the business cycle.

It is important to note that unemployment can be a combination of different types. For example, a worker who has been laid off due to a recession may experience both cyclical and frictional unemployment as they search for new employment. Understanding the different types of unemployment can help policymakers, economists, and individuals better understand the causes of unemployment and develop strategies to address it.
\subsection{ Types of inflation: hyperinflation, disinflation, deflation }
Inflation is the rate at which the general level of prices for goods and services is rising, and subsequently, purchasing power is falling. There are several different types of inflation, including hyperinflation, disinflation, and deflation.

Hyperinflation: Hyperinflation is a rapid and excessive increase in prices, typically defined as a rate of inflation over 50% per month. This type of inflation is often the result of a loss of confidence in the currency and the government, as well as an increase in the money supply. Hyperinflation can result in widespread economic disruption, as individuals and businesses struggle to keep up with rising prices.

Disinflation: Disinflation is a slowing in the rate of inflation, characterized by a decrease in the rate of increase in prices. Disinflation is often seen as a positive development, as it suggests that the economy is stabilizing and that inflationary pressures are easing.

Deflation: Deflation is a decrease in the general price level of goods and services, which results in an increase in the purchasing power of money. Deflation can be caused by a variety of factors, including a decrease in demand for goods and services, an increase in productivity, and a decrease in the money supply. Deflation can be a sign of a healthy economy, but it can also result in economic hardship, as individuals and businesses struggle to adjust to lower prices and reduced demand.

Understanding the different types of inflation is important for policymakers, economists, and individuals, as it can help to identify the causes of inflation and inform decisions about how to respond to it.
\subsection{ Quantity theory of money: types of demand for money (transaction, precautionary, speculative) }
The Quantity Theory of Money is a theory that explains the relationship between the quantity of money in circulation and the level of prices in an economy. The theory argues that, in the long run, an increase in the money supply will lead to a proportional increase in the general level of prices, and vice versa.

One of the key concepts in the Quantity Theory of Money is the distinction between three different types of demand for money: transaction demand, precautionary demand, and speculative demand.

Transaction demand: Transaction demand for money refers to the amount of money that individuals and businesses need to carry out everyday transactions, such as buying goods and services, paying bills, and making investments. This demand is driven by the need to make transactions and is largely independent of the level of prices.

Precautionary demand: Precautionary demand for money refers to the amount of money that individuals and businesses hold as a buffer against unexpected events, such as sudden changes in income, employment, or health. This demand is driven by the need for financial security and is generally larger when economic uncertainty is high.

Speculative demand: Speculative demand for money refers to the amount of money that individuals and businesses hold as a speculative investment, with the expectation that its value will increase over time. This demand is driven by the desire to profit from changes in the value of money and is generally larger when the rate of inflation is high or when expectations of future price changes are uncertain.

The Quantity Theory of Money argues that changes in the money supply will have a proportional impact on the level of prices in the long run, depending on the relative size of these different types of demand for money. By understanding the different types of demand for money, policymakers and economists can make more informed decisions about how to manage the money supply and respond to changes in the level of prices.
\subsection{ Fisher effect }
The Fisher effect is an economic theory that describes the relationship between inflation and interest rates. The theory states that the nominal interest rate, which is the rate of interest that is expressed in current dollars, is equal to the real interest rate, which is the rate of interest adjusted for inflation, plus the expected rate of inflation. The Fisher effect can be represented mathematically as follows:

Nominal interest rate = Real interest rate + Expected inflation rate

The Fisher effect is named after economist Irving Fisher, who first introduced the idea in 1930. The theory argues that, in a well-functioning economy, changes in inflation are reflected in changes in nominal interest rates. When inflation is expected to rise, nominal interest rates also rise, to compensate for the loss of purchasing power that comes with higher prices. When inflation is expected to fall, nominal interest rates fall, to reflect the increased purchasing power that comes with lower prices.

The Fisher effect is important because it provides a theoretical basis for understanding how changes in inflation and interest rates interact in the economy. By understanding the Fisher effect, economists and policymakers can make more informed decisions about how to manage inflation and interest rates, and respond to changes in the economy. Additionally, the Fisher effect can be used to calculate the real interest rate, which is a useful indicator of the cost of borrowing, taking into account the effects of inflation.




\subsection{ Monetary policy tools }
Monetary policy refers to the actions taken by a central bank, such as the Federal Reserve in the United States, to manage the money supply and control interest rates in order to achieve its macroeconomic goals, such as price stability, full employment, and economic growth. The central bank has a number of tools at its disposal to carry out monetary policy. Some of the most common monetary policy tools are:

Open market operations: This refers to the central bank's purchase or sale of government securities on the open market. By buying securities, the central bank increases the money supply and lowers interest rates, which stimulates economic activity. By selling securities, the central bank reduces the money supply and raises interest rates, which slows down economic activity.

Reserve requirements: The central bank can change the amount of reserves that banks are required to hold, which affects the amount of money that banks have available to lend. If the central bank increases reserve requirements, it reduces the money supply and raises interest rates. If it reduces reserve requirements, it increases the money supply and lowers interest rates.

Discount rate: The discount rate is the rate at which banks can borrow funds from the central bank. By changing the discount rate, the central bank can influence the overall level of interest rates in the economy.

Communication: The central bank can also influence interest rates and the economy through its public statements and forecasts. By communicating its views on the economy and its future monetary policy plans, the central bank can influence expectations and expectations can, in turn, influence economic outcomes.

These are some of the key monetary policy tools that central banks can use to influence the money supply and interest rates. The specific tools that a central bank uses will depend on its specific goals and the state of the economy. In general, however, the goal of monetary policy is to maintain price stability, promote full employment, and support economic growth.
monetary policy tools are used to control the money supply and interest rates in order to achieve the central bank's macroeconomic goals. In addition to the tools mentioned earlier, here are a few other monetary policy tools that are commonly used:

Term Auction Facility (TAF): The TAF is a lending program that allows banks to bid for funds from the central bank. The central bank sets the interest rate and auction participants submit bids for the desired amount of funds.

Repurchase agreements (repos): A repo is a short-term lending agreement between the central bank and a bank. The bank sells securities to the central bank with an agreement to buy them back at a later date for a higher price. This increases the money supply in the short term and helps to lower interest rates.

Interest on reserves: The central bank can pay interest on the reserves that banks hold, which can encourage banks to hold onto more reserves and reduce the amount of money they lend. This can help to reduce inflation and keep interest rates low.

Foreign exchange operations: Central banks can also use their foreign exchange reserves to influence the exchange rate and indirectly affect interest rates and economic activity. For example, the central bank can sell its domestic currency and buy a foreign currency, which can weaken the domestic currency and make exports more competitive.

These are some of the main monetary policy tools that central banks can use to influence the economy. The choice of tool will depend on the specific goals of the central bank and the state of the economy. Effective monetary policy can help to promote stability, support economic growth, and achieve the central bank's macroeconomic goals.

\subsection{ Fiscal policy tools }
Fiscal policy refers to the use of government spending and taxation to influence the economy. It is one of the main tools that policymakers use to stabilize the economy, reduce inflation and unemployment, and promote economic growth.

Some common fiscal policy tools are:

Government spending: The government can increase spending on infrastructure projects, transfer payments, and other programs to stimulate demand and encourage economic growth.

Taxation: The government can adjust tax rates and policies to affect consumer and business behavior. For example, lowering tax rates can increase consumer spending and business investment, while raising taxes can reduce demand and curb inflation.

Public debt: The government can issue bonds to finance its spending and stimulate demand in the economy.

Transfer payments: The government can increase or decrease transfer payments, such as unemployment benefits and social security payments, to influence the economy.

These tools can be used to counter the effects of a recession, stimulate economic growth, and control inflation. However, they also have potential drawbacks, such as increased government debt, reduced consumer confidence, and a shift in the balance between inflation and unemployment.




\subsection{ Use of inflation, interest rate, and exchange rate targeting by central banks }
central banks use inflation targeting, interest rate targeting, and exchange rate targeting as a way to achieve their macroeconomic goals. Let's take a closer look at each:

Inflation targeting: Inflation targeting is a monetary policy framework in which a central bank sets an explicit target for the rate of inflation and uses various monetary policy tools to achieve this target. The goal of inflation targeting is to maintain low and stable inflation, which can support economic growth and help to reduce uncertainty.

Interest rate targeting: Interest rate targeting involves the central bank setting an explicit target for the policy interest rate and using its monetary policy tools to achieve that target. The goal of interest rate targeting is to control the level of economic activity by influencing the cost and availability of credit.

Exchange rate targeting: Exchange rate targeting involves the central bank setting a target for the exchange rate and using its foreign exchange reserves to intervene in the market and achieve that target. The goal of exchange rate targeting is to maintain a stable exchange rate, which can help to reduce uncertainty and promote trade and investment.

These monetary policy frameworks are used by central banks around the world to achieve their macroeconomic goals. In practice, the central bank may use a combination of these frameworks, depending on the specific economic conditions and the goals of the central bank. The choice of framework will depend on various factors, such as the state of the economy, the level of inflation, the level of interest rates, and the exchange rate.
\subsection{ GDP calculation: income vs. expenditure vs. value-added approach}  
Gross Domestic Product (GDP) is a measure of a country's economic output. There are three main approaches to calculating GDP: the income approach, the expenditure approach, and the value-added approach.

The income approach: The income approach calculates GDP as the sum of all income generated in the economy, including wages and salaries, rent, and profits. The formula for the income approach is:
GDP = Wages + Rent + Profits + Net Interest + Net Indirect Taxes

Where:
Wages = the sum of all wages and salaries earned in the economy
Rent = the sum of all rent earned in the economy
Profits = the sum of all profits earned in the economy
Net Interest = the sum of all interest earned in the economy minus the interest paid
Net Indirect Taxes = the sum of all indirect taxes collected minus subsidies paid

The expenditure approach: The expenditure approach calculates GDP as the sum of all spending in the economy, including consumption, investment, government spending, and net exports. The formula for the expenditure approach is:
GDP = C + I + G + (X - M)

Where:
C = consumption
I = investment
G = government spending
X = exports
M = imports

The value-added approach: The value-added approach calculates GDP as the sum of the value added by each producer in the economy. The value added by a producer is equal to its output minus the cost of the inputs it has purchased from other producers. The formula for the value-added approach is:
GDP = ∑ Value added by all producers

In practice, the expenditure and the value-added approaches tend to give very similar results, and the income approach is used less often. However, all three approaches provide useful information about the economy and can be used in different contexts to understand the different sources of economic growth and to make economic forecasts.





\section{Finance  }
\subsection{ Time value of money }
The time value of money is the concept that a dollar today is worth more than a dollar in the future due to the opportunity cost of not having that money today. This means that the value of money changes over time, and it is important to account for this when making financial decisions.

The time value of money is a fundamental principle in finance and is used in a variety of financial applications, including investments, loans, and valuing future cash flows. In order to account for the time value of money, it is necessary to calculate the present value of future cash flows, which is the value of those cash flows today.

The present value of a future cash flow can be calculated using the following formula:

$$PV = \frac{FV}{(1 + r)^n}$$

where $PV$ is the present value, $FV$ is the future value, $r$ is the discount rate, and $n$ is the number of time periods.

The discount rate is a measure of the opportunity cost of not having the money today, and it takes into account the risk and inflation associated with the future cash flows.

The time value of money is also an important concept in the pricing of bonds. The price of a bond is determined by the present value of the future cash flows it will generate, including interest payments and the return of principal at maturity.

In conclusion, the time value of money is a fundamental concept in finance that is used to account for the opportunity cost of not having money today. It is used in a variety of financial applications and is important in determining the present value of future cash flows, including the pricing of bonds.




\subsection{ Option pricing }
Option pricing is the process of determining the fair value of an option based on various factors such as the underlying asset price, volatility, interest rates, and time to expiration. The pricing of options is important for both buyers and sellers of options as it helps to determine the expected return on an investment and the potential risk involved.

There are several models for pricing options, including the Black-Scholes model, the binomial model, and the Monte Carlo simulation model. Each model has its own strengths and weaknesses and is suited to different types of options and underlying assets.

The Black-Scholes model is a widely used model for pricing European options and is based on the assumption of a lognormal distribution of asset prices, a constant volatility, and a constant risk-free interest rate. The model uses the following formula to calculate the fair value of a European call option:

$$C = S\Phi(d_1) - Xe^{-rt}\Phi(d_2)$$

where $C$ is the price of the call option, $S$ is the price of the underlying asset, $X$ is the exercise price, $r$ is the risk-free interest rate, $t$ is the time to expiration, $\Phi$ is the cumulative distribution function of a standard normal random variable, and $d_1$ and $d_2$ are defined as:

$$d_1 = \frac{\ln(\frac{S}{X}) + (r + \frac{\sigma^2}{2})t}{\sigma \sqrt{t}}$$

$$d_2 = d_1 - \sigma \sqrt{t}$$

where $\sigma$ is the volatility of the underlying asset.

The binomial model, on the other hand, takes into account the possibility of early exercise and the impact of volatility on the underlying asset price and is used for pricing American options. The Monte Carlo simulation model uses simulations to estimate the probability of the underlying asset price reaching a certain level by the expiration date and is used for pricing options on complex and illiquid underlying assets.

In conclusion, option pricing is an important aspect of options trading and the choice of pricing model depends on the type of option and underlying asset being traded. The Black-Scholes model is widely used for European options, the binomial model is used for American options, and the Monte Carlo simulation model is used for complex and illiquid underlying assets.
\subsection{ Put-call parity }
Put-call parity is a relationship between the price of a put option and a call option on the same underlying asset with the same expiration date. The relationship states that the price of a put option plus the price of the underlying asset must equal the price of a call option plus the present value of the exercise price.

Mathematically, put-call parity can be expressed as follows:

$$C + PV(X) = P + S$$

where $C$ is the price of a call option, $P$ is the price of a put option, $S$ is the price of the underlying asset, $X$ is the exercise price of the option, and $PV(X)$ is the present value of the exercise price, calculated as $PV(X) = Xe^{-rt}$, where $r$ is the risk-free interest rate and $t$ is the time to expiration.

Put-call parity is an important concept in options trading as it provides a way to determine whether an option is overpriced or underpriced based on the prices of other options on the same underlying asset. If the relationship between the price of a put option and a call option is not in line with put-call parity, it may indicate an arbitrage opportunity.

In conclusion, put-call parity is a useful tool for options traders to determine the fairness of option prices and to identify potential arbitrage opportunities.


\subsection{ American vs. European options }
European options and American options are two types of options that are used to trade securities. The main difference between the two is the time period during which they can be exercised.

European options can only be exercised on the expiration date. For example, if an investor holds a European call option, they can only exercise their right to buy the underlying asset on the expiration date. This makes European options simpler to price and easier to understand compared to American options.

On the other hand, American options can be exercised at any time prior to the expiration date. This means that an investor holding an American option has the flexibility to exercise their right to buy or sell the underlying asset at any time before the expiration date. This added flexibility can make American options more valuable than European options.

It is important to note that American options are generally more difficult to price than European options due to the added flexibility in exercise. The Black-Scholes model, for example, can only be used to price European options, and alternative models, such as binomial or Monte Carlo simulations, are required to price American options.

In conclusion, European options are simpler to understand and price compared to American options, but American options offer greater flexibility in exercise. The choice between the two types of options will depend on the investor's specific needs and preferences.





\subsubsection{binomial option pricing }
The binomial option pricing model is a popular alternative to the Black-Scholes model for pricing American options. The binomial model takes into account the possibility of the underlying asset price moving up or down over a series of discrete time periods until the expiration date.

The binomial model is based on the concept of a lattice or tree, where each node represents a possible future state of the underlying asset price. The model calculates the option price by working backwards from the expiration date and considering all possible future states of the underlying asset price.

The binomial model is formulated as follows:

Let $u$ and $d$ be the up and down factors, respectively, and let $p$ be the probability of the asset price moving up. Then, the option price at each node in the lattice can be calculated as follows:

$$V_i = e^{-r\Delta t}[pV_{i+1} + (1-p)V_{i+2}]$$

where $V_i$ is the option price at time $i$, $r$ is the risk-free interest rate, $\Delta t$ is the time step, and $i$ is the number of time steps from the expiration date.

The binomial model is a useful alternative to the Black-Scholes model for pricing American options, as it takes into account the possibility of early exercise and the impact of volatility on the underlying asset price. However, it is important to note that the binomial model requires a large number of calculations to accurately price an option and may be less efficient than other models such as Monte Carlo simulations.




\subsection{ Black-Scholes-Merton differential equation and Black-Scholes formula }
The Black-Scholes-Merton (BSM) model is a widely used mathematical model for pricing options. The BSM model takes into account the underlying asset price, time to expiration, volatility, and risk-free interest rate to estimate the fair value of an option. The BSM model is based on the Black-Scholes differential equation, which describes how the option price changes over time.

The Black-Scholes differential equation is given by:

$$\frac{\partial C}{\partial t} + \frac{1}{2} \sigma^2 S^2 \frac{\partial^2 C}{\partial S^2} + rS \frac{\partial C}{\partial S} - rC = 0$$

where $C$ is the option price, $t$ is the time to expiration, $\sigma$ is the volatility of the underlying asset, $S$ is the price of the underlying asset, and $r$ is the risk-free interest rate.

The Black-Scholes formula is a closed-form solution to the Black-Scholes differential equation and is given by:

For a European call option:

$$C(S,t) = N(d_1)S - N(d_2)Ke^{-rt}$$

where

$$d_1 = \frac{\ln{\left(\frac{S}{K}\right)} + \left(r + \frac{\sigma^2}{2}\right)(T-t)}{\sigma \sqrt{T-t}}$$

$$d_2 = d_1 - \sigma \sqrt{T-t}$$

For a European put option:

$$P(S,t) = N(-d_2)Ke^{-rt} - N(-d_1)S$$

where $N(x)$ is the cumulative distribution function of a standard normal distribution, $T$ is the expiration date, $K$ is the strike price, and $t$ is the current time.

The Black-Scholes formula is widely used to price options, although it has several limitations and assumes a constant volatility and constant risk-free interest rate over time.

\subsection{ Greeks: delta, gamma, theta, vega, rho }
The Greeks are a set of risk measures that are used to quantify the sensitivity of an option's price to various underlying factors. The five main Greeks are:

Delta: Delta measures the change in the option price with respect to the change in the price of the underlying asset. Delta can be used to estimate the probability that an option will expire in the money. Delta is represented by the Greek letter $\Delta$.

Gamma: Gamma measures the rate of change of delta with respect to the price of the underlying asset. Gamma is used to estimate the rate at which the option price will change as the price of the underlying asset changes. Gamma is represented by the Greek letter $\Gamma$.

Theta: Theta measures the rate of change of the option price with respect to the passage of time. Theta is used to estimate the rate at which the option price will decline as expiration approaches. Theta is represented by the Greek letter $\Theta$.

Vega: Vega measures the change in the option price with respect to the change in the implied volatility of the underlying asset. Vega is used to estimate the sensitivity of the option price to changes in the implied volatility. Vega is represented by the Greek letter $\nu$.

Rho: Rho measures the change in the option price with respect to the change in the risk-free interest rate. Rho is used to estimate the sensitivity of the option price to changes in the risk-free interest rate. Rho is represented by the Greek letter $\rho$.

Each of the Greeks can be used to help estimate the potential outcomes of an option investment and to manage risk. Understanding the Greeks is an important part of options trading and investment strategy.

\subsubsection{formulas }
Here are the formulas for each of the five Greeks:

Delta: The delta of a European call option is given by:
$$\Delta = \frac{\partial C}{\partial S}$$

where $C$ is the call option price and $S$ is the price of the underlying asset. The delta of a European put option is given by:

$$\Delta = \frac{\partial P}{\partial S}$$

where $P$ is the put option price.

Gamma: The gamma of an option is given by:
$$\Gamma = \frac{\partial^2 C}{\partial S^2}$$

for a call option and

$$\Gamma = \frac{\partial^2 P}{\partial S^2}$$

for a put option.

Theta: The theta of a European call option is given by:
$$\Theta = -\frac{\partial C}{\partial t}$$

where $t$ is the time to expiration. The theta of a European put option is given by:

$$\Theta = -\frac{\partial P}{\partial t}$$

Vega: The vega of an option is given by:
$$\nu = \frac{\partial C}{\partial \sigma}$$

for a call option and

$$\nu = \frac{\partial P}{\partial \sigma}$$

for a put option, where $\sigma$ is the implied volatility of the underlying asset.

Rho: The rho of a European call option is given by:
$$\rho = \frac{\partial C}{\partial r}$$

where $r$ is the risk-free interest rate. The rho of a European put option is given by:

$$\rho = \frac{\partial P}{\partial r}$$


\subsection{ Option portfolios: bull spread, straddle, binary options, exchange options }
Option portfolios refer to a collection of options that are combined to create a single investment strategy. Some of the common types of option portfolios include:

Bull Spread: A bull spread is a portfolio that consists of a long call option and a short call option with a lower strike price. This strategy is used when the investor expects the underlying asset to increase in value, but is not willing to purchase the underlying asset directly. By using a bull spread, the investor can benefit from a potential increase in the value of the underlying asset while limiting the potential losses from an unexpected decline in the value of the underlying asset.

Straddle: A straddle is a portfolio that consists of a long call option and a long put option with the same strike price. This strategy is used when the investor expects a significant change in the value of the underlying asset, but is not sure whether the change will be positive or negative. By using a straddle, the investor can benefit from either an increase or a decrease in the value of the underlying asset.

Binary Options: Binary options are a type of option that offers a fixed payout if the underlying asset meets certain conditions. Binary options can be combined to create a portfolio that offers a range of investment opportunities and risk levels.

Exchange Options: Exchange options are options that are traded on an exchange, rather than over-the-counter. Exchange options can be combined to create a portfolio that offers a range of investment opportunities and risk levels.

Regardless of the type of option portfolio, it is important to understand the underlying assets, their risk characteristics, and the potential outcomes of the investment strategy before investing. Option portfolios can offer a range of investment opportunities, but they also come with inherent risks that must be understood and managed.




\subsection{ Portfolio optimization }
Portfolio optimization is the process of selecting a portfolio of investments that provides the highest expected return for a given level of risk or the lowest level of risk for a given level of expected return. The goal of portfolio optimization is to maximize the expected utility of the portfolio returns for an investor.

There are several approaches to portfolio optimization, including:

Markowitz Model: This is a mean-variance optimization model that assumes that investors are risk-averse and will seek to maximize the expected return of their portfolio while minimizing its risk. The Markowitz model is based on the following optimization problem:
$$\text{Maximize} \ \ \mu_p - \lambda \cdot \sigma_p^2$$

Where $\mu_p$ is the expected return of the portfolio, $\sigma_p^2$ is the variance of the portfolio returns, and $\lambda$ is a risk-aversion parameter that represents the trade-off between expected return and risk.

Black-Litterman Model: This is a Bayesian approach to portfolio optimization that allows for the incorporation of views and opinions of the investor into the portfolio optimization process. The Black-Litterman model is based on the following optimization problem:
$$\text{Maximize} \ \ \mu_p^T\mathbf{w} - \frac{1}{2} \mathbf{w}^T \Sigma \mathbf{w}$$

Where $\mu_p$ is the expected return of the portfolio, $\mathbf{w}$ is a vector of weights assigned to each asset in the portfolio, $\Sigma$ is the covariance matrix of the assets, and $\mathbf{w}^T \Sigma \mathbf{w}$ is a measure of the risk of the portfolio.

Risk Parity Model: This approach seeks to allocate assets in the portfolio so that the risk of each asset is equal, regardless of the expected return. The risk parity model is based on the following optimization problem:
$$\text{Minimize} \ \ \sum_{i=1}^{N} w_i^2 \sigma_i^2$$

Where $w_i$ is the weight assigned to asset $i$, $\sigma_i^2$ is the variance of returns for asset $i$, and $N$ is the number of assets in the portfolio.

Regardless of the approach used, portfolio optimization is a complex process that requires a thorough understanding of the assets in the portfolio and their dependencies, as well as an accurate estimation of their expected returns and risk characteristics. However, when done correctly, portfolio optimization can help investors make informed decisions about their portfolios and achieve their investment goals.
\subsection{ Value at risk }
Value at Risk (VaR) is a measure of the potential loss of an investment portfolio over a given time period and with a specified level of confidence. It is a widely used risk management tool that helps investors assess the potential risk of their portfolios and make informed investment decisions.

The formula for calculating VaR is as follows:

$$VaR = -\mu + Z\cdot \sigma$$

Where $VaR$ is the value at risk, $\mu$ is the mean return of the portfolio, $Z$ is the standard normal cumulative distribution function (CDF) with a confidence level of $1-\alpha$, and $\sigma$ is the standard deviation of the portfolio returns.

For example, if an investor has a portfolio with a mean return of 5% and a standard deviation of 10%, and they want to calculate the VaR with a confidence level of 95%, the formula would be as follows:

$$VaR = -0.05 + 1.65\cdot 0.1 = -0.05 + 0.165 = 0.115$$

This means that with a 95% confidence level, the investor's portfolio is expected to lose no more than 11.5% over the given time period.

It is important to note that VaR is not a perfect measure of risk as it only considers the potential loss at a specified confidence level and does not account for the potential for extreme losses beyond that level. However, it is still a useful tool for investors as it provides a quick and simple way to assess the risk of their portfolios and make informed investment decisions.
\subsubsection{Var using Copulas }
Value at Risk (VaR) calculations can also be performed using copulas, which are mathematical models that describe the dependence structure between different variables. Copulas allow for the calculation of VaR for portfolios with more complex dependencies and can provide a more accurate estimate of the potential risk of a portfolio.

The process of using copulas for VaR calculation typically involves three steps:

Model the marginal distributions of each asset in the portfolio.

Choose a copula function that best fits the dependence structure between the assets.

Use the copula to generate joint distributions of the assets and calculate the VaR based on these joint distributions.

One common copula function used in VaR calculations is the Gaussian copula, which assumes a normal distribution for each asset in the portfolio and a linear dependence structure between the assets. The formula for calculating VaR using a Gaussian copula is as follows:

$$VaR = -\mu + Z\cdot \sqrt{\text{cov}(\mathbf{x})}$$

Where $VaR$ is the value at risk, $\mu$ is the mean return of the portfolio, $Z$ is the standard normal cumulative distribution function (CDF) with a confidence level of $1-\alpha$, $\text{cov}(\mathbf{x})$ is the covariance matrix of the portfolio returns, and $\mathbf{x}$ is a vector of portfolio returns.

It is important to note that the choice of copula function will impact the accuracy of the VaR calculation, so it is important to choose a copula that accurately reflects the dependence structure of the portfolio. Additionally, the use of copulas for VaR calculations can be computationally intensive, so it may not be practical for large portfolios or portfolios with a high number of assets.



\subsection{ Duration and convexity }
Duration and convexity are important concepts in fixed income investing that measure the sensitivity of a bond's price to changes in interest rates. They are used by investors to help assess the risk of their investments and make more informed decisions about their portfolios.

Duration is a measure of a bond's sensitivity to changes in interest rates. It is defined as the weighted average time to receive the cash flows of a bond, with weights proportional to the present value of the cash flows. The formula for calculating the duration of a bond is as follows:

$$D = \frac{\sum_{t=1}^{n} t\cdot PV_t}{\sum_{t=1}^{n} PV_t}$$

Where $D$ is the duration of the bond, $t$ is the time period, $PV_t$ is the present value of the cash flow in period $t$, and $n$ is the number of periods.

Convexity, on the other hand, is a measure of the curvature of a bond's price-yield relationship. It measures the change in the duration of a bond as interest rates change. The formula for calculating the convexity of a bond is as follows:

$$C = \frac{\sum_{t=1}^{n} t^2\cdot PV_t}{\sum_{t=1}^{n} PV_t}$$

Where $C$ is the convexity of the bond, $t$ is the time period, $PV_t$ is the present value of the cash flow in period $t$, and $n$ is the number of periods.

Duration and convexity are useful metrics for bond investors as they provide a more complete picture of the risk associated with a bond. A bond with a high duration is more sensitive to changes in interest rates and therefore more risky, while a bond with a high convexity is less sensitive to changes in interest rates and therefore less risky. By considering both duration and convexity, investors can make more informed decisions about their investments and create more efficient portfolios.




\subsection{ Forward and futures }
Forward contracts and futures contracts are both financial instruments that allow investors to take a position on the future price of an asset. They are similar in many ways, but there are also some key differences between the two.

A forward contract is an agreement between two parties to buy or sell an asset at a specified price and date in the future. Forwards are customized contracts that are tailored to the specific needs of the two parties involved. They are typically used by corporations or individuals who want to lock in a future price for a commodity or currency.

Futures contracts, on the other hand, are standardized contracts that are traded on an exchange. Futures contracts specify the price, quantity, and delivery date of a particular asset. They are used by a wide range of market participants, including corporations, speculators, and hedgers.

The main difference between forwards and futures is that futures contracts are traded on centralized exchanges and are subject to regulation, while forward contracts are traded over-the-counter and are not subject to regulation. Because futures contracts are standardized and traded on exchanges, they are more liquid and transparent than forward contracts.

Another important difference between forwards and futures is that futures contracts typically involve margin requirements. This means that the trader must put up a portion of the contract value as collateral in order to trade the contract. The margin requirement helps to ensure that the trader has the financial resources to fulfill the contract if the price of the asset moves against them.

Overall, both forwards and futures are useful tools for investors who want to take a position on the future price of an asset. The choice between the two depends on a variety of factors, including the specific needs of the investor, the liquidity of the market, and the regulatory environment.
\subsubsection{Pricing}
The pricing of a forward contract or a futures contract is determined by a few key factors, including the spot price of the underlying asset, the time to expiration, and the risk-free interest rate. The following formulas can be used to calculate the price of a forward contract or a futures contract:

For a forward contract:

$$F = S\cdot e^{(r-q)T}$$

Where $F$ is the forward price, $S$ is the spot price of the underlying asset, $r$ is the risk-free interest rate, $q$ is the dividend rate, and $T$ is the time to expiration.

For a futures contract:

$$F = S\cdot e^{(r-q)T} - K\cdot e^{-rT}$$

Where $F$ is the futures price, $S$ is the spot price of the underlying asset, $K$ is the strike price, $r$ is the risk-free interest rate, $q$ is the dividend rate, and $T$ is the time to expiration.

It is important to note that these formulas are based on the assumptions of a risk-free interest rate and a constant dividend rate. In reality, these values may change over time and can impact the pricing of the contract. Additionally, the pricing of futures contracts can also be influenced by the volatility of the underlying asset and the margin requirements.

By using these formulas, investors can estimate the price of a forward contract or a futures contract and make more informed decisions about their investment strategies.
\subsection{ Interest rate models: Vasicek model }

The Vasicek model is a mathematical model used to describe the behavior of interest rates over time. It was first introduced by Oldrich Vasicek in 1977 and remains one of the most widely used models in finance today. The Vasicek model assumes that interest rates follow a mean-reverting process and that the speed at which they revert to their mean is constant.

The Vasicek model is based on the following equation:

$$dr(t) = (\theta - a r(t))dt + \sigma dW(t)$$

Where $r(t)$ is the interest rate at time $t$, $\theta$ is the long-run average interest rate, $a$ is the mean-reversion speed, $\sigma$ is the volatility of the interest rate, and $dW(t)$ is a standard Wiener process.

This equation can be used to simulate the evolution of interest rates over time, and to estimate the probability distribution of future interest rates. By incorporating additional information about the current level of interest rates and the volatility of the rate, it is possible to generate a forecast of interest rate changes.

One important question that arises when using the Vasicek model is how to determine the values of the parameters $\theta$, $a$, and $\sigma$. These parameters can be estimated using historical data and optimization algorithms, but there is no universally accepted method for determining their values. As a result, the results generated by the Vasicek model are subject to uncertainty and are not always accurate.

Despite its limitations, the Vasicek model remains a useful tool for financial professionals who need to make decisions about investments and risk management. By providing a simple and intuitive framework for understanding interest rate dynamics, it can help investors to make more informed decisions about the future behavior of interest rates.


\section{Additional}
\section{Optional / Extras  Probability Theory Games  }
\subsection{ Coin toss game }
Problem Statement:
A player tosses a fair coin 3 times. What is the probability of getting exactly 2 heads?

Solution:
There are 8 possible outcomes for 3 coin tosses (HHH, HHT, HTH, THH, THT, TTH, HTT, TTT), and exactly 2 heads can be obtained in 3 different ways (HHT, HTH, THH).
Therefore, the probability of getting exactly 2 heads is:

P(2 heads) = \frac{3}{8}


\subsection{ Card game }
Problem Statement:
A standard deck of cards has 52 cards, including 4 suits (hearts, diamonds, clubs, and spades) and 13 ranks (Ace, 2, 3, 4, 5, 6, 7, 8, 9, 10, Jack, Queen, and King). In this game, a player draws 3 cards randomly from a standard deck of cards without replacement. What is the probability of getting a King, a Queen, and a Jack?

Solution:
The total number of possible combinations of 3 cards from a standard deck of cards is 52 choose 3, which can be calculated as follows:

C(52, 3) = \frac{52!}{3!(52-3)!} = \frac{52 * 51 * 50}{3 * 2 * 1} = 132600

The number of ways to get a King, a Queen, and a Jack is 4 choose 1 for each of the 3 ranks, which can be calculated as follows:

C(4, 1) * C(4, 1) * C(4, 1) = 4 * 4 * 4 = 64

Therefore, the probability of getting a King, a Queen, and a Jack is:

P(King, Queen, Jack) = \frac{64}{132600} = \frac{1}{2080}

Latex Formulas:
C(52, 3) = \frac{52!}{3!(52-3)!} = \frac{52 * 51 * 50}{3 * 2 * 1} = 132600

C(4, 1) * C(4, 1) * C(4, 1) = 4 * 4 * 4 = 64

P(King, Queen, Jack) = \frac{64}{132600} = \frac{1}{2080}
\subsection{ Drunk passenger  }
Problem Statement:
A drunken passenger gets into a taxi at a random point on a 1-mile long straight road. The passenger then staggers either to the left or to the right, each with probability 0.5, until reaching the side of the road.
What is the expected distance the passenger will travel?

Solution:
Let X be the distance the passenger travels from the starting point. We can calculate the expected value of X as follows:

E(X) = 0.5 * (X + 1) + 0.5 * (1 - X)

E(X) = 0.5 * (X + 1) + 0.5

2 * E(X) = X + 1 + 1

2 * E(X) = X + 2

X = 2 * E(X) - 2

E(X) = 1

Therefore, the expected distance the passenger will travel is 1 mile.

Latex Formulas:
E(X) = 0.5 * (X + 1) + 0.5 * (1 - X)

2 * E(X) = X + 2

X = 2 * E(X) - 2

E(X) = 1
\subsection{ N points on a circle }
Problem Statement:
Suppose N points are randomly placed on a circle. What is the probability that at least two of the points are within a certain distance d of each other?

Solution:
The probability that at least two points are within a certain distance d of each other can be calculated as follows:

Let R be the radius of the circle. Then, the area of the circle is given by:

A = \pi R^2

Let's consider a small region of area a around a point on the circle. The probability that there is another point within distance d of this point is proportional to the area of this region:

p = \frac{a}{A} = \frac{a}{\pi R^2}

Since the probability of at least two points being within distance d of each other is 1 minus the probability that no points are within distance d of each other, we can write:

P = 1 - (1 - p)^N

Note that the above formula assumes that the points are randomly placed on the circle. If the points are placed according to a different distribution, the formula may need to be adjusted accordingly. Additionally, the above formula assumes that the distance d is small relative to the size of the circle. If the distance d is large relative to the size of the circle, a different formula may be required to calculate the probability of at least two points being within distance d of each other.

Large Distance d in N Points on a Circle Problem

Problem Statement:
Suppose N points are randomly placed on a circle. What is the probability that at least two of the points are within a certain distance d of each other, where d is large relative to the size of the circle?

Solution:
In the case where the distance d is large relative to the size of the circle, a different approach is needed to calculate the probability that at least two points are within distance d of each other.

One approach is to use geometric probability. Let's consider a region on the circle of radius d. The area of this region is proportional to the probability that a randomly placed point on the circle will fall within this region:

a = 2 \pi d

The probability that a randomly placed point on the circle will fall within this region is given by:

p = \frac{a}{A} = \frac{2 \pi d}{\pi R^2} = \frac{2d}{R}

Since the probability of at least two points being within distance d of each other is 1 minus the probability that no points are within distance d of each other, we can write:

P = 1 - (1 - p)^{N-1}

Note that the above formula assumes that the points are randomly placed on the circle. If the points are placed according to a different distribution, the formula may need to be adjusted accordingly.
\subsection{ Poker hands }
Problem Statement:
In a game of poker, what is the probability of being dealt a specific hand, such as a full house or a straight flush?

Solution:
The probability of being dealt a specific hand in poker depends on the number of cards in the deck and the number of cards needed for the specific hand. For example, a full house requires three cards of one rank and two cards of another rank, while a straight flush requires five cards of consecutive rank and of the same suit.

To calculate the probability of being dealt a specific hand, we need to determine the number of ways that the specific hand can be formed and divide by the total number of possible hands.

For example, the number of ways to form a full house with a deck of 52 cards is given by the formula:

C_{13}^{1} \cdot C_{4}^{3} \cdot C_{12}^{1} \cdot C_{4}^{2}

where C_{n}^{k} is the number of combinations of n items taken k at a time.

Dividing this number by the total number of possible hands, which is C_{52}^{5}, gives the probability of being dealt a full house.

Similarly, the probability of being dealt a straight flush can be calculated by determining the number of ways to form a straight flush and dividing by the total number of possible hands.

Note that these probabilities are only estimates, as the actual probability of being dealt a specific hand will depend on the order in which the cards are dealt.
\subsection{ Hopping rabbit }
Problem Statement:
A rabbit is hopping randomly on a number line, starting at the origin. At each step, the rabbit either moves one unit to the right or one unit to the left. What is the probability that the rabbit will reach a certain point x after n steps?

Solution:
The Hopping Rabbit problem is a classic example of a random walk problem. The probability that the rabbit will reach a certain point x after n steps can be calculated by counting the number of ways the rabbit can reach that point after n steps.

Since the rabbit can only move one unit to the right or one unit to the left at each step, the total number of ways the rabbit can reach a certain point x after n steps is given by the sum of the number of ways the rabbit can reach x after n-1 steps and the number of ways the rabbit can reach x-1 after n-1 steps.

Using this logic, we can write a recursive formula for the probability P_{n,x} that the rabbit will reach a point x after n steps:

P_{n,x} = \frac{1}{2}(P_{n-1,x-1} + P_{n-1,x+1})

The initial conditions are P_{0,0} = 1 and P_{n,x} = 0 for all x not equal to 0 and all n less than 0.

Using this recursive formula, we can calculate the probability that the rabbit will reach a certain point x after n steps for any values of n and x. Note that this formula assumes that the rabbit starts at the origin and makes exactly n steps, and does not account for any additional constraints or conditions that may affect the movement of the rabbit.
\subsection{ Screwy pirates 2 }
Problem Statement:
You and n-1 other pirates have taken over a ship and plan to divide the loot. To determine the order in which the pirates can divide the loot, you draw numbers from a hat with the numbers 1 to n. The pirate with the highest number goes first, the pirate with the second-highest number goes second, and so on. You draw a number and find that it is the kth-largest number. What is the probability that you will go first (i.e., have the highest number)?

Solution:
The solution to the Screwy Pirates problem, Part 2 involves finding the probability that your number is the largest among all the n numbers. The number of ways in which you can have the largest number is n-1 (all other numbers are less than your number). The total number of ways in which n numbers can be drawn from the hat is n!, so the probability of you having the largest number is given by:

P = \frac{n-1}{n!}

Note that this formula assumes that each number has an equal chance of being drawn and that the order in which the numbers are drawn does not affect the outcome.

In the case of you having the kth largest number, we can generalize the formula to:

P = \frac{n-k+1}{n!}

This means that the probability that you will go first depends on the value of k. The higher the value of k, the smaller the probability that you will go first, and vice versa.




\subsection{ Chess tournament }
Problem Statement:
In a chess tournament, each player plays one game with every other player. Assuming that the outcome of each game is either a win, loss, or draw, what is the probability that a given player will win exactly k games?

Solution:
The number of games that a player can win in a tournament is dependent on the number of players in the tournament. Let's consider a tournament with n players. In this tournament, a player can win n-1 games, lose n-1 games, or draw n-2 games. The number of ways in which the player can win exactly k games is given by the binomial coefficient, C(n-1, k), where C(n,k) = n!/(k!(n-k)!).

The probability of the player winning exactly k games is given by:

P = \frac{C(n-1,k)}{3^{n-1}}

where 3^{n-1} is the total number of possible outcomes (win, loss, or draw) for each game.

This means that the probability of a player winning exactly k games in a tournament with n players depends on the number of players and the number of wins desired. The more players in the tournament, the smaller the probability of winning exactly k games, and the fewer the number of wins desired, the smaller the probability.

\subsection{ Application letters }
Problem Statement:
Suppose you are applying for a job and you have to send an application letter to a company. What is the probability that your application will be accepted given the number of applications the company receives and their acceptance rate?

Solution:
This problem can be approached using the concept of probability mass function (PMF) of a discrete random variable. Let X be the number of accepted applications and N be the total number of applications received by the company. Then, X follows a binomial distribution with parameters N and p, where p is the probability of a single application being accepted.

The PMF of X can be calculated as follows:

P(X = k) = (N choose k) * p^k * (1-p)^(N-k)

Where (N choose k) is the number of ways to choose k successful outcomes from N trials, given by the binomial coefficient:

(N choose k) = N! / (k! (N-k)!).

Using the PMF, one can find the expected number of accepted applications and various other probabilities of interest.

\subsection{ Birthday problem }
Problem Statement:
Suppose there are n people in a room. What is the probability that at least two of them have the same birthday?

Solution:
The birthday problem is a classic example of a probability puzzle. The problem can be solved by finding the probability that all n people have different birthdays, and subtracting it from 1 to find the probability that at least two people have the same birthday.

The probability that all n people have different birthdays can be calculated as follows:

P(all different) = 365/365 * 364/365 * ... * (365-n+1)/365

This expression gives the probability that the first person has a certain birthday, the second person has a different birthday, and so on, until the nth person has a different birthday from all the others.

The probability that at least two people have the same birthday is then given by:

P(at least 2 same) = 1 - P(all different)

It can be shown that as n increases, the probability of two people having the same birthday approaches 1. For example, if there are 23 people in the room, the probability that at least two of them have the same birthday is over 50%. This surprising result highlights the importance of considering the finite number of possible birthdays when calculating probabilities in real-world scenarios.

\subsection{ 100th digit }
Problem Statement:
Consider a sequence of real numbers that are formed by concatenating the positive integers. For example, the first few terms of the sequence are: 1, 2, 3, 4, 5, 6, 7, 8, 9, 10, 11, 12, ... Find the 100th digit after the decimal point in this sequence.

Solution:
This problem can be solved using mathematical calculations and not by brute force. To find the 100th digit, we need to determine the number in the sequence that contains the 100th digit, and then determine the position of the 100th digit within that number.

The number of digits in the nth positive integer is given by the logarithm function: log10(n) + 1. Thus, to find the number that contains the 100th digit, we need to find the smallest n such that the sum of log10(i) + 1 for i = 1 to n is greater than or equal to 100.

Once we have found the number that contains the 100th digit, we can calculate the position of the digit within the number. The position of the digit is given by 100 minus the sum of the digits of the smaller integers.

By solving this problem, we can see that the 100th digit is 7.
\subsection{ Cubic of integer }
Problem Statement:
Find the smallest positive integer whose cube ends in 888.

Solution:
This problem can be solved by testing integers and calculating their cubes until we find the one that ends in 888.

Since the number must be a positive integer, we can start by testing the smallest integer, 1. The cube of 1 is 1, which does not end in 888, so we move on to the next integer, 2. The cube of 2 is 8, which also does not end in 888. We keep testing the integers until we find the one that satisfies the condition.

By testing the integers, we can see that the smallest positive integer whose cube ends in 888 is 11.
\subsection{ Boys and girls }
Problem Statement:
In a random family of 2 children, what is the probability that both children are boys?

Solution:
The number of possible outcomes when a family of 2 children is formed can be represented by the sample space, S = {BB, BG, GB, GG}, where B represents a boy and G represents a girl.

The probability of having both children as boys is given by P(BB) = 1/4.

So, the answer is P(BB) = 1/4, which means that in a random family of 2 children, there is a 1 in 4 chance that both children will be boys.
\subsection{ All-girl world }
Problem Statement:
Suppose in a world of families with two children, it is known that all children born are girls. What is the probability that the next family to have a boy?

Solution:
The sample space, S = {GG, BG, GB}, consists of

all possible combinations of families with 2 children, where B represents a boy and G represents a girl. Given that all children born are girls, the event of interest is the birth of a boy in the next family, which can be represented as the event B. The probability of event B happening is 1/3, as each combination of family with 2 children (GG, BG, and GB) has equal chance of happening.

Therefore, the probability that the next family to have a boy is 1/3.
\subsection{ Unfair coin }
Problem Statement:
Suppose we have a coin that lands heads with probability p and tails with probability (1-p). What is the probability of getting exactly k heads in n tosses of the coin?

Solution:
The number of ways to get exactly k heads in n tosses of the unfair coin is given by the binomial coefficient, C(n,k), where C(n,k) = n!/(k!(n-k)!). The probability of getting exactly k heads in n tosses is given by:

P = C(n,k) * p^k * (1-p)^{n-k}

where p^k is the probability of getting k heads, and (1-p)^{n-k} is the probability of getting n-k tails.

So, the probability of getting exactly k heads in n tosses of an unfair coin depends on the number of tosses and the probability of getting heads. The more tosses, the more likely it is to get close to the expected number of heads, and the more biased the coin, the more likely it is to get more or less heads than the expected number.

\subsection{ Fair probability from an unfair coin }
Problem Statement:
Suppose we have a coin that lands heads with probability p and tails with probability (1-p), and we want to estimate the probability of heads in n tosses of the coin. What is the maximum likelihood estimator of the probability of heads?

Solution:
The maximum likelihood estimator (MLE) of the probability of heads is given by the number of heads divided by the number of tosses. In other words, the MLE of the probability of heads is given by:

p_{MLE} = \frac{number\ of\ heads}{number\ of\ tosses} = \frac{k}{n}

where k is the number of heads in n tosses.

This means that the MLE of the probability of heads is simply the fraction of heads in the sample, and it is a unbiased estimator for the true probability of heads. The more tosses, the more accurate the estimate will be, and the closer the estimate will be to the true probability.
\subsection{ Dart game }
Problem Statement:
Suppose you are playing a dart game where you have to hit a target of radius 1. The darts are thrown randomly and independently, and the probability of hitting the target is proportional to the area of the target. What is the probability of hitting the target exactly k times in n throws?

Solution:
The probability of hitting the target exactly k times in n throws is given by a binomial distribution, with success probability equal to the ratio of the area of the target to the area of the surrounding square of side 2.

The area of the target is given by π, and the area of the square is 4, so the probability of hitting the target is π/4. Hence, the probability of hitting the target exactly k times in n throws is given by:

P = C(n,k) * (π/4)^k * (1-π/4)^{n-k}

where C(n,k) = n!/(k!(n-k)!) is the binomial coefficient and (1-π/4)^{n-k} is the probability of missing the target n-k times.

The more throws, the more likely it is to get close to the expected number of hits, and the larger the target, the more likely it is to hit the target.
\subsection{ Birthday line }
Problem Statement:
Suppose you have a line of n people and you want to find the probability that no two people have the same birthday.

Solution:
The probability that no two people have the same birthday can be found by assuming that each person can have any birthday with equal probability, and then multiplying the probabilities that each person has a different birthday.

The probability that the first person has a different birthday than the others is 1. The probability that the second person has a different birthday than the first and the others is (364/365), since there are 364 other possible birthdays for the second person. Continuing in this manner, we get the probability that no two people have the same birthday as:

P = \prod_{i=1}^{n} \frac{365-i+1}{365} = \frac{365!}{365^n(365-n)!}

This expression approaches 0 as n approaches 365, meaning that as the number of people approaches the number of days in a year, the probability of no two people having the same birthday becomes very small. This is the famous "birthday paradox."
\subsection{ Dice order }
Problem Statement:
Suppose you have n dice, and you want to find the probability that the numbers on the dice appear in ascending order.

Solution:
The number of ways to arrange the numbers on the dice can be found by multiplying the number of possible outcomes for each die. Since each die has 6 sides, there are 6^n possible arrangements of the numbers on the dice.

To find the number of arrangements that are in ascending order, we can use the idea of permutations with restrictions. For the first die, there are 6 possible numbers. For the second die, there are 5 possible numbers that are greater than the number on the first die, and so on. Thus, the number of arrangements that are in ascending order is:

A = \prod_{i=1}^{n} (6-i+1) = 6! / (6-n)!

The probability of getting an ascending order of numbers on the dice is then given by:

P = \frac{A}{6^n}

This expression approaches 0 as n increases, meaning that as the number of dice increases, the probability of getting an ascending order becomes very small.





\subsection{ Monty Hall problem }
Problem Statement:
Suppose you are on a game show and you are presented with three doors. Behind one door is a car, and behind the other two are goats. You are asked to choose a door. After you have made your choice, the host opens one of the doors you did not choose to reveal a goat. You are then given the opportunity to switch your choice to the other unopened door, or to stick with your original choice. Which choice gives you the best chance of winning the car?

Solution:
The best strategy is to switch your choice. The probability of choosing the correct door initially is 1/3. After the host reveals a goat, the probability of the car being behind the door you originally chose is reduced to 1/2, while the probability of the car being behind the door you did not originally choose is increased to 1/2. Thus, by switching, you have a probability of 1/2 of winning the car, which is better than the probability of 1/3 of winning by sticking with your original choice.

This result may seem counterintuitive, but it is a result of the way the problem is constructed. The key is to recognize that the host's decision to reveal a goat gives you new information about the location of the car, and that this information should be used to update your beliefs about where the car is.
\subsection{ Amoeba population }
Problem Statement:
Consider an amoeba population that grows exponentially. At time t=0, there is only one amoeba, and each amoeba splits into two amoebae every hour. Suppose that an amoeba population N(t) at time t is given by the formula N(t) = 2^t. What is the expected number of amoebae after 5 hours?

Solution:
The expected number of amoebae after 5 hours is 2^5 = 32. The population grows exponentially with time, so the number of amoebae doubles every hour. Hence, the expected number of amoebae after 5 hours is 2^5, which is 32.

Note that this result assumes that the growth rate of the amoeba population is constant and that the rate of splitting is not affected by the size of the population. In reality, the growth rate of the population may be influenced by various factors, such as the availability of food and the presence of predators, which would affect the actual number of amoebae.
\subsection{ Candies in a jar }
Problem Statement:
A jar contains N candies, and you are allowed to take out candies one by one. The candies are indistinguishable, meaning that you cannot tell which candy you have taken out. You have to find the expected number of candies you have to take out before you have taken out all N candies.

Solution:
Let X be the number of candies you have to take out. Then X is a random variable with the discrete uniform distribution over the set {1,2,...,N}. Hence, its expected value can be found as follows:

E(X) = (1 + 2 + ... + N)/N = (N + 1)/2

So, the expected number of candies you have to take out before you have taken all N candies is (N + 1)/2. This result applies to any number of candies N in the jar.
\subsection{ Russian roulette series }
Problem Statement:
Russian Roulette is a deadly game in which a player spins a six-chamber revolver, puts a bullet into one of the chambers, and then pulls the trigger. If the chamber with the bullet is aligned with the barrel, the player dies. Suppose that you and your friend take turns playing the game, and the game continues until one of you dies. What is the probability that you survive?

Solution:
Let X be the number of turns it takes for one of you to die. Then X is a geometric random variable with parameter 1/6, since on each turn, the probability of dying is 1/6.

The probability that you survive can be found by finding the expected number of turns it takes for your friend to die, and then subtracting that from the total number of turns. Let Y be the number of turns it takes for your friend to die. Then Y is a geometric random variable with parameter 1/6. Hence,

E(Y) = 1/ (1/6) = 6

Since you and your friend take turns playing the game, the expected number of turns it takes for one of you to die is E(X) = E(Y) / 2 = 3. Hence, the probability that you survive is 1 - 1/3 = 2/3.

Note that this solution assumes that the bullet is equally likely to be in any of the six chambers and that the spins are independent. In reality, the spins may not be perfectly independent, which would affect the actual probability of surviving.




\subsection{ Aces }

???
recheck


Problem Statement:
In a standard deck of 52 playing cards, there are 4 aces. Two players each draw a card from the deck, and the player who draws the ace wins. What is the probability that the first player wins?

Solution:
The probability that the first player draws an ace is 4/52. Given that the first player has drawn an ace, the probability that the second player does not draw an ace is 48/51. Hence, the probability that the first player wins is:

P(first player wins) = 4/52 * 48/51 = 192/2652 = 4/59

So, the probability that the first player wins is 4/59, or about 6.78%.
\subsection{ Gambler’s ruin problem }
Problem Statement:
A gambler starts with an initial stake of x dollars and plays a fair game of chance, where each game has a probability of p of winning and a probability of 1-p of losing. The gambler continues to play until he either reaches a goal of y dollars or is ruined with zero dollars. What is the probability that the gambler reaches the goal before being ruined?

Solution:
Let q = 1 - p, the probability of losing. The probability that the gambler reaches the goal of y dollars before being ruined can be expressed as:

P = p^n / (p^n + q^n)

where n is the number of games required to reach the goal, which can be found using the formula:

n = (y - x) / (p - q)

Hence, the probability that the gambler reaches the goal before being ruined is given by the formula above, where n and p can be calculated from the initial stake, goal, and the probabilities of winning and losing.
\subsection{ Basketball scores }
Problem Statement:
In a basketball game, each basket made by a player has a probability of p of being worth 2 points, and a probability of 1-p of being worth 3 points. Find the expected number of points a player scores in a given number of shots.

Solution:
Let x be the number of baskets worth 2 points and y be the number of baskets worth 3 points. The expected number of points a player scores can be found using the formula for expected value:

E(Points) = 2p x + 3(1-p) y

Since x + y = the total number of shots, we have:

E(Points) = 2p (total shots - y) + 3y (1-p)
E(Points) = 2p (total shots) - 2p y + 3y - 3y(p)
E(Points) = 2p (total shots) + y (1 - 3p)

So the expected number of points a player scores is a linear function of the total number of shots and p, the probability of making a 2-point basket.
\subsection{ Cars on road }
Problem Statement:
In a basketball game, each basket made by a player has a probability of p of being worth 2 points, and a probability of 1-p of being worth 3 points. Find the expected number of points a player scores in a given number of shots.

Solution:
Let x be the number of baskets worth 2 points and y be the number of baskets worth 3 points. The expected number of points a player scores can be found using the formula for expected value:

E(Points) = 2p x + 3(1-p) y

Since x + y = the total number of shots, we have:

E(Points) = 2p (total shots - y) + 3y (1-p)
E(Points) = 2p (total shots) - 2p y + 3y - 3y(p)
E(Points) = 2p (total shots) + y (1 - 3p)

So the expected number of points a player scores is a linear function of the total number of shots and p, the probability of making a 2-point basket.
\subsection{ Meeting probability }
Look there https://stats.stackexchange.com/questions/15448/probability-of-two-people-meeting

Problem Statement:
Two people, A and B, are to arrive at the same location some time between a and b hours. They both arrive at random times within the hour and only stay for x minutes. Since they only go one time, what is the probability that they will meet during that hour?

Solution:
Let's denote the arrival time of person A as T_A and the arrival time of person B as T_B. The arrival times of both individuals are uniformly distributed within the interval [a, b].

The probability that A and B will meet can be calculated as the probability that the difference between their arrival times is less than x minutes:

P(|T_A - T_B| <= x/60) = \frac{2x}{(b-a) * 60}

Note that the above formula assumes that both individuals arrive at different times and that they stay for exactly x minutes. If they arrive at the same time or if they stay for a different amount of time, a different formula may be required to calculate the probability of meeting.

\subsubsection{ANother formulation}

Problem Statement:
Suppose two individuals, A and B, are randomly walking in a 2-dimensional plane. What is the probability that they will meet?

Solution:
The probability that two individuals, A and B, will meet can be calculated as follows:

Let T be the time it takes for A and B to meet. Then, T has an exponential distribution with parameter 2 (since the rate at which they move is equal and constant). The probability that A and B will meet within a time t is given by:

P(T <= t) = 1 - e^(-2t)

Note that this formula assumes that A and B are walking in a 2-dimensional plane and that their paths do not intersect. If the paths of A and B intersect, the meeting time will be the same as the time it takes for them to cross each other's paths. If the paths of A and B are parallel, they will never meet.

Note: The above formula can be generalized to more than two individuals moving in a 2-dimensional plane. The formula can also be extended to higher dimensions by considering the probability that the individuals will meet within a given distance in a 3-dimensional plane, for example.
\subsection{ Probability of triangle }
Problem Statement:
Suppose we have n points randomly distributed in a 2-dimensional plane. What is the probability that three of these points will form a triangle?

Solution:
The probability that three randomly chosen points in a 2-dimensional plane will form a triangle can be calculated as follows:

P(triangle) = 1 - P(no triangle)

where P(no triangle) is the probability that no triangle is formed.

To calculate P(no triangle), we consider the different ways in which three points can fail to form a triangle. There are two cases:

Collinear points: Three points are collinear if they lie on a single line. The probability of this happening can be calculated as follows:
P(collinear) = C(n, 3) * (\frac{1}{2})^3

where C(n, 3) is the number of combinations of n things taken 3 at a time.

Non-collinear points: Three points are non-collinear if they do not lie on a single line. The probability of this happening can be calculated as follows:
P(non-collinear) = C(n, 3) * (1 - \frac{1}{2})^3

Finally, P(no triangle) is given by:

P(no triangle) = P(collinear) + P(non-collinear)

P(triangle) = 1 - P(no triangle)

Note: This formula assumes that each point is equally likely to be chosen. If the points have different probabilities of being chosen, a different formula may be required to calculate the probability of triangle formation.

May be not accurate: https://www.youtube.com/watch?v=xFS4xpYQ82w&ab_channel=MITOpenCourseWare
\subsection{ Property of Poisson process }
A Poisson process is a type of discrete-time stochastic process that models the number of events that occur in a fixed interval of time. The following are some important properties of a Poisson process:

Memoryless property: The number of events that have occurred in a given time interval has no effect on the number of events that will occur in a future time interval. This means that the number of events in any time interval is independent of the number of events that have occurred in the past.

Constant rate of occurrence: The rate at which events occur is constant over time. This means that the average number of events per unit time remains the same, regardless of the time elapsed.

Stationary increments: The number of events that occur in any given time interval is the same, regardless of the starting time of the interval.

Independence of increments: The number of events that occur in two disjoint time intervals are independent of each other.

Poisson distribution: The number of events that occur in a given time interval follows a Poisson distribution.

Note: These properties hold for a true Poisson process. In practice, many processes that are modeled as Poisson processes may not strictly satisfy all of these properties.
\subsection{ Moments of normal distribution }
Problem Statement:
Suppose X is a random variable with a normal distribution, with mean (mu) and variance (sigma^2). What are the first and second moments (mean and variance) of the normal distribution?

Solution:
First Moment (Mean):
The mean (mu) of a normal distribution is given by:

\mu = E(X)

where E(X) is the expected value of X.

Second Moment (Variance):
The variance (sigma^2) of a normal distribution is given by:

\sigma^2 = Var(X) = E((X - \mu)^2)

where Var(X) is the variance of X, and E((X - \mu)^2) is the expected value of the squared difference between X and its mean.

Note: This formula provides the mean and variance for a normal distribution. The normal distribution is a continuous probability distribution that is symmetrical and bell-shaped, with the mean (mu) and variance (sigma^2) as its parameters.




\subsection{ Connecting noodles }
Problem Statement:
Suppose we have n noodles, and we want to connect them end-to-end to form a single long noodle. What is the expected length of the final long noodle?

Solution:
The expected length of the final long noodle can be calculated as follows:

E(L) = n * \frac{l_1 + l_2 + ... + l_n}{n}

where l_1, l_2, ..., l_n are the lengths of the individual noodles. This formula states that the expected length of the final long noodle is equal to the number of noodles times the average length of an individual noodle.

Note: This formula provides the expected length of the final long noodle when the individual noodles are equally likely to be chosen. If the individual noodles have different probabilities of being chosen, a different formula may be required to calculate the expected length of the final long noodle.
\subsection{ Optimal hedge ratio }
Problem Statement:
In financial mathematics, the optimal hedge ratio is the ratio that minimizes the variance of the portfolio returns. Given two assets, X and Y, what is the optimal hedge ratio that minimizes the portfolio returns?

Solution:
The optimal hedge ratio can be calculated as the ratio of the covariance between the two assets to the variance of one of the assets. The formula is given by:

\text{Optimal Hedge Ratio} = \frac{Cov(X, Y)}{Var(Y)}

This formula states that the optimal hedge ratio is equal to the ratio of the covariance between the two assets to the variance of one of the assets. The optimal hedge ratio can be used to create a portfolio that has a lower risk and higher expected return than either of the individual assets.

Note: This formula provides the optimal hedge ratio for two assets X and Y. If more than two assets are involved, a more complex formula may be required to calculate the optimal hedge ratio.
\subsection{ Sum of random variables }
Problem Statement:
Let X and Y be two random variables with expected values E(X) and E(Y) and variances Var(X) and Var(Y) respectively. What is the expected value and variance of the sum Z = X + Y?

Solution:
Expected Value of Sum:
The expected value of the sum Z = X + Y can be calculated as follows:

E(Z) = E(X + Y) = E(X) + E(Y)

Variance of Sum:
The variance of the sum Z = X + Y can be calculated as follows:

Var(Z) = Var(X + Y) = Var(X) + Var(Y) + 2 * Cov(X, Y)

Note: This formula assumes that X and Y are uncorrelated. If X and Y are correlated, the covariance term Cov(X, Y) must be included in the formula for the variance of the sum.
\subsection{ Coupon collection }
Problem Statement:
Suppose there are n different coupons, and each coupon is equally likely to be collected after each purchase. What is the expected number of purchases needed to collect all n coupons?

Solution:
The expected number of purchases needed to collect all n coupons can be calculated as follows:

E = n * (1 + \frac{1}{2} + \frac{1}{3} + ... + \frac{1}{n})

This formula states that the expected number of purchases is equal to the number of coupons times the harmonic series. The harmonic series is a mathematical series that approaches infinity as n approaches infinity.

Note: This formula provides the expected number of purchases for collecting all n coupons when each coupon is equally likely to be collected after each purchase. If the coupons are not equally likely, a different formula may be required to calculate the expected number of purchases.
\subsection{ Joint default probability }
Problem Statement:
Suppose we have two companies, A and B, with default probabilities of p_A and p_B respectively. What is the joint default probability, i.e., the probability that both companies default at the same time?

Solution:
The joint default probability can be calculated as the product of the individual default probabilities:

P(\text{joint default}) = p_A * p_B

Note: This formula assumes that the defaults of the two companies are independent events, i.e., the default of one company does not affect the default of the other company. If the defaults are dependent, a more complex formula may be required to calculate the joint default probability.
\subsection{ Order statistics: expected value of max and min; correlation of max and min }
Problem Statement:
Let X1, X2, ..., Xn be a sample of independent and identically distributed random variables with common cumulative distribution function F(x). Find the expected value of the maximum and minimum, and the correlation of the maximum and minimum.

Solution:
Expected Value of Maximum:
The expected value of the maximum can be calculated as follows:

E(X_{max}) = \int_{-\infty}^{\infty} x * [F(x)]^{n} * f(x) dx

Expected Value of Minimum:
The expected value of the minimum can be calculated as follows:

E(X_{min}) = \int_{-\infty}^{\infty} x * [1 - F(x)]^{n} * f(x) dx

Correlation of Maximum and Minimum:
The covariance of the maximum and minimum can be calculated as follows:

Cov(X_{max}, X_{min}) = E(X_{max}X_{min}) - E(X_{max})E(X_{min})

The correlation of the maximum and minimum can be calculated as follows:

\rho = \frac{Cov(X_{max}, X_{min})}{\sqrt{Var(X_{max})Var(X_{min})}}

Where Var(X) is the variance of X.

Note: These formulas provide general expressions for the expected values of the maximum and minimum and the correlation between them. To calculate these quantities for specific distributions, the cumulative distribution function F(x) and the probability density function f(x) must be known.

\subsubsection{Order Statistics: Expected Value of Max and Min for Poisson Distribution}
Problem Statement:
Let X1, X2, ..., Xn be a sample of independent and identically distributed Poisson random variables with mean λ. Find the expected value of the maximum and minimum.

Solution:
Expected Value of Maximum:
For a Poisson distribution, the cumulative distribution function F(x) is given by:

F(x) = 1 - e^{-\lambda x}

The expected value of the maximum can be calculated as follows:

E(X_{max}) = \int_{0}^{\infty} x * [1 - e^{-\lambda x}]^{n} * \frac{\lambda^{x}e^{-\lambda}}{x!} dx

Expected Value of Minimum:
The expected value of the minimum can be calculated as follows:

E(X_{min}) = \int_{0}^{\infty} x * [e^{-\lambda x}]^{n} * \frac{\lambda^{x}e^{-\lambda}}{x!} dx

Note: These formulas provide the expected values of the maximum and minimum for a Poisson distribution. To calculate these quantities for other distributions, the cumulative distribution function F(x) and the probability density function f(x) must be known.
\subsection{ Random ants}
Problem Statement:
Suppose that 10 ants are randomly placed on a number line from 0 to 10. What is the probability that at least two ants are at the same point?

Solution:
The number of ways to place 10 ants on a number line from 0 to 10 is 10^10.
The number of ways to place 10 ants such that none are at the same point is 10!.
Therefore, the probability that at least two ants are at the same point is:

P(\text{at least two ants at same point}) = 1 - \frac{10!}{10^{10}}

\section{Brain Teasers  }
\subsection{ Screwy pirates }
\subsection{ Tiger and sheep }
\subsection{ River crossing }
\subsection{ Birthday problem }
\subsection{ Burning ropes }
\subsection{ Card game }
\subsection{ Defective ball }
\subsection{ Trailing zeros }
\subsection{ Horse race }
\subsection{ Infinite sequence }
\subsection{ Box packing }
\subsection{ Calendar cubes }
\end{document}
